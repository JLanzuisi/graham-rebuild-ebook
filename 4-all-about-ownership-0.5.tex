\chapter{The free company}
\addcontentsline{toc}{chapterdescription}{The myths around what a company is only maintain our dysfunctional status quo. Like apartheid in South Africa, typical company incorporation forms exclude most stakeholders from governance and wealth sharing. The tragedy of the commons isn’t what we imagine. Rather, done well the commons is a solution, and it’s time for the FairShares Commons company or equivalent commons company to rise to our global challenges and build a better world.}
\addcontentsline{toc}{chapterdescription}{\pagebreak}
\label{chapter:ownership}




\begin{chapterquotation}
Freedom is a heavy load, a great and strange burden for the spirit to undertake. It is not easy. It is not a gift given, but a choice made, and the choice may be a hard one. The road goes upward towards the light; but the laden traveler may never reach the end of it.\\
\raggedleft\textemdash Ursula K. Le Guin\index{Le Guin, Ursula K.}


\centering
Designing institutions to force (or nudge) entirely self-interested individuals to achieve better outcomes has been the major goal posited by policy analysts for governments to accomplish for much of the past half century. Extensive empirical research leads me to argue that instead, a core goal of public policy should be to facilitate the development of institutions that bring out the best in humans.\\
\raggedleft\textemdash Elinor Ostrom at her 2009 Nobel lecture.\index{Ostrom, Elinor}
\end{chapterquotation}




I (Graham) grew up in apartheid South Africa\index{South Africa!apartheid}. I never felt safe, and struggled to see any path out of the morally wrong and socially unstable system to a viable future. I felt powerless to do anything, hopeless about the future, and angry at how South Africa was.


Are you angry about climate change, and other injustices in today’s world, but feel powerless to change the system? Have you joined Extinction Rebellion\index{Extinction Rebellion}, or the school strikes, and want to know what comes next? Do you want to build a new system that works? This chapter is a call to action, and describes an essential building block for building a system that works.


Looking back at South Africa, I can now see clearly how apartheid was like a hidden force, a gravity, pulling at all South Africans. Anyone who tried to act differently was pulled back into the apartheid swamp and the binary choice of either defending or attacking apartheid, rather than creating a viable future. Apartheid\index{South Africa!apartheid} began as a library of stories\index{stories}, and these stories then shaped the reality everyone experienced. Few recognised that the stories were only that, stories, with no deeper reality. But by believing them, all the structures and processes of apartheid were built. 


In 2008 I decided to leave my career in the middle management of Procter and Gamble\index{Procter and Gamble} to focus on figuring out how to harness the power of business for the greater good. I also left because my work no longer felt safe. Not safe for me in the moment, and the consequences were certainly not safe for future generations. Even though P\&G’s clearly stated purpose was, and is, to improve lives, I observed that too many choices were made to improve shareholder returns at the cost of the quality of life of staff today and all of us in the future. 


You will read more about the critical importance of psychological safety\index{psychological safety} in section~\ref{section:growing-living-organisation}. Even without reading the definitions, what is your gut feeling? Do you feel psychologically safe at work? Do you feel psychologically safe in the society we have today? Do you trust your boss with your future? Do you trust the shareholders to vote for what will be good for you, your children, and your grandchildren, even if that diminishes their return on investment this quarter?


Greater inequality drives safety down. We can measure inequality with the Gini coefficient\index{Gini coefficient} developed by the Italian sociologist Corrado Gini. It is often used to measure the gap between the wealthiest and the poorest within nations or regions. A Gini coefficient of 1 means total inequality, one person in the country has all the wealth, the rest have none; 0 means perfect equality, everyone has the same. The UN estimates 0.4 as the threshold for risk of social unrest, so any country with a Gini coefficient above 0.4 ought to be worried.


Economists\index{economists} are also seeing that the greater the Gini coefficient, the lower the economy’s resilience, which limits individual wellbeing, growth and happiness. For an economy to go beyond resilient into antifragile the Gini coefficient needs to be near zero.


I propose we consider the Gini coefficient within each company\footnote{I hope that a researcher picks this up. I would love to see hard data on the Gini coefficients across all stakeholders, and compared to companies failing because they lost their social licence to operate.}, across all stakeholders, as an indicator of corporate viability. What is your company’s Gini coefficient? How large is the inequality between the total wealth of the C-suite and investors, and the salaried staff? How big is it across all stakeholders? 


Companies are first and foremost social constructions that society has invented to do specific jobs that society needs doing. A Gini coefficient \index{Gini coefficient} above 0.4 in your company is a cause for concern that you no longer have the social licence to operate. So the motivations for a 300x gap between the CEO and the lowest-paid member of staff are flawed, because they lead to the company losing its social licence to operate. 


In fact, there is data that having a flat and public pay structure at all levels raises company performance\cite{kristof-gravity-70minwage}. What if you \emph{boosted} your company's success by shrinking the remuneration gap to between 1 and 25; chances are you will.


My experiences in P\&G\index{Procter and Gamble}, and hearing the very similar experiences of my friends in other companies, led me to deeply question the fundamentals: what are the structural foundations of trust, psychological safety, and fiduciary vs. moral responsibility in business? What might the hidden gravities be\textemdash gravities pulling well-intentioned people into the swamp of behaviour contrary to their values? 


The stories that make meaning for us are the gravity\index{gravity} making our world less psychologically safe by the day, despite the best intentions of so many people. Much of the polarised politics of today is us trying to feel safe again.


Keep Picasso\index{Picasso, Pablo} in mind throughout this chapter. You may have looked through one lens all your life and see very clearly through it; but that does not mean you have seen clearly what actually is, and all that is, just as Picasso showed that there was far more to represent in a painting than traditional perspective painting could ever manage. Ask yourself what the lenses (i.e., stories\index{stories}) you are using might prevent you from seeing. What don't you even imagine might be invisible to you through the lenses\index{lens} you are using?
\section{Incorporation and apartheid}
The foundation of South Africa’s apartheid\index{South Africa!apartheid} is also hidden in the foundations of today’s business, and is one root cause of the global problems we are facing. This common foundation is the legal construct restricting the power\index{power} to govern, and the right to a share of the wealth generated, to only the investing shareholders\index{shareholders}, rather than to all stakeholders\index{stakeholders}.


The word apartheid is built of two elements, with the basic meaning: \emph{a context (heid) of separation (apart)}. Such a context of separation was the gravity\index{gravity} driving the behaviours that harmed all South Africans, and a context of separation is the gravity driving the behaviours that are harming us all today. 


Apartheid itself, and everything that apartheid means, grew out of a context, or set of stories\index{stories} and lenses that saw having a white skin as the reason for someone to have better capabilities and to deserve more. So only those who had a white skin were eligible to vote for politicians, own land, and enjoy most of the country’s wealth. In reality, it’s luck, not cause and effect. A century ago having white skin just meant that you came from Europe, which had just been lucky enough to be on a random upswing vs. other cultures to greater power and wealth through the Industrial Revolution.


Apartheid disenfranchised all other stakeholders\index{stakeholders}, denying them any role in steering the country into the future, and denying them a share of the wealth generated from their past labour. All the emotional and physical injustices, the lives lived far short of their potential; 
and the protests inside and outside South Africa, were inevitable consequences of the construct of South African apartheid\index{South Africa!apartheid}, combined with the inevitable human biases we all share, the stages of meaning\hyp{}making we all go through, and a low fluidity in transformational thinking.


\begin{longstoryblock}
On my most recent flight to South Africa, I (Jack) sat next to a white South African woman living in the US, who valued how much more creative she found the music of today's South Africa than the music she grew up with.
\end{longstoryblock}


Stories, lenses, shape experienced reality and the constructs we build, including legal constructs. These legal constructs are just stories made concrete in law. 


The insight that the N{\"u}remberg\index{N{\"u}remberg} trials of the Nazi leadership after WWII established was that even when a story is written into law, that law is not a justification for actions contrary to the overarching laws of humanity. This means that you can be tried and convicted for actions that were fully within the law at the time if any reasonable person ought to have recognised that those actions were against humanity's moral laws. You can think of these trials as establishing the primacy of actuality over reality (Section~\ref{section:stories}).


Does having more money, or financial power, mean that you are always more capable of making better choices in the annual general meeting of a multinational than any other stakeholder? Better against all frames of reference?


This is the equivalent meaning\hyp{}making story\index{meaning-making stories} in today's world to the story that having a white skin meant you took better governance decisions for society and nations. A story that emphasised one point of difference between people, hiding other sources of difference, and more importantly hiding our huge common ground. These stories are just lenses that highlight one small part of all that is, and hide the rest. Mistaking the distorted and limited view you see through one lens for the whole is deeply flawed, as Picasso\index{Picasso, Pablo} spent his life illustrating. 


The institution of business today comes from a meaning\hyp{}making story emphasising one difference, money, and hiding both other differences and our huge common ground. This meaning\hyp{}making story\index{meaning-making stories} then shapes our reality, and turns into the legal constructs we have today. Constructs separating stakeholders from the company, allowing only the investor group the right to vote and a share of the wealth generated. (In Chapter~\ref{chapter:gen-th-economies-flaws} we describe in more detail how meaning\hyp{}making stories shape our economy.)


We have invented this context of separation of stakeholders, this apartheid\index{South Africa!apartheid} in business. We took the meaning in the stories as a given, created our reality to match these meaning\hyp{}making stories, then concretised them in the laws we have written, as well as the norms, cultures and practices of business. Ewan McGaughey\index{McGaughey, Ewan} has shown quite clearly that company law in some countries~\cite{mcgaughey} does not restrict voting to investors, and the summary of commonly believed myths about the company from a global consortium~\cite{veldman-modern-economics} that you’ll read more about in Section~\ref{section:myths-of-incorporation} open even more space for change. 


This separation prevents all but the investors from contributing their perspective, their insight into future risks and opportunities, to steer our companies into creating a viable future; and reserves most of the benefits for those investors, demotivating all others from investing discretionary effort, creativity etc.


We seldom make it a habit to become truly aware of our context, whether it’s physical gravity\index{gravity} or some other kind of gravity pulling at us, shaping the space in which we move (see the appendix); and we seldom recognise how this pull shapes everyone’s behaviour, nor our ability to change it. I hope that throughout this book you are becoming steadily more aware of how your meaning\hyp{}making stories\index{meaning-making stories} shape the reality that you experience, and that you can begin changing it. 


\section{Excluding stakeholders causes crises}
\index{investor|(}
\emph{What is a business for?}\textemdash a Harvard Business Review article\cite{handy-whats-a-business-for} from Charles Handy\textemdash \index{Handy, Charles} accelerated my journey, around 2009, to understanding how a context of separation is part of creating and anchoring many, if not all, of our challenges today.


I believe that climate change, likely everything threatening us with extinction, is an inevitable consequence of the apartheid-like shareholder context exacerbating our inevitable and normal mix of human biases, stages of meaning\hyp{}making and fluidity of transformational thinking. Excluding most stakeholders\textemdash human and non-human\textemdash from the key decisions that create our shared reality produces a gravity pulling us deeper into the swamp of extinction.


When investors\index{investor} have all the decision power in a multinational, and enough take as a given that a company is for \emph{“maximising shareholder value / Total Shareholder Return (TSR)”}, \index{Total Shareholder Return} then decisions will be taken based on the worldview, emotional drives, and cognitive biases of investors. 


There is nothing wrong with shareholder returns, i.e., the financial investor's\index{investor} return on investment\index{RoI}. You need to keep a careful eye on paying back the money people have invested in your company, and enough to reward them for taking the risk. Generating a healthy level of wealth is an important role of business, and one of the best indicators that your strategies are good and being executed with excellence. But maximising TSR \index{Total Shareholder Return} is certainly not the sole, not even a primary, objective. Paine’s recent HBR article\cite{paine-governance}, looking at how Covid-19 has fundamentally changed our business context, shows how systemic inclusion of stakeholders is better for all, including investors.


Certainly I am against acquiring more money today by burning up the very natural capitals\footnote{I use capitals here to emphasise the multiplicity of capitals in nature; elsewhere I use capital where I choose to not emphasise the multiplicity.} \index{capitals!financial}we ought to be stewarding in trust for our children, and their children. Today’s harmful outcomes are an unavoidable consequence of the apartheid-like business context our dominant meaning\hyp{}making stories\index{meaning-making stories} have generated.


This apartheid-like separation even creates split personalities in individuals. 


Someone may be a wonderfully loving mother at home with her children, and then sit in an annual general meeting (AGM) representing the pension fund employing her, making choices that will harm her grandchildren. In the AGM everyone is inside the context of separation, deep in their roles, the stories and structures creating a narrow reality\textemdash that of the financial stakeholders, i.e., the shareholders. No other stakeholders are present, none can voice with power the needs of other stakeholder groups, and none have the power to vote.


As well as all the power\index{power}, the financial investors also get all the rise in the company’s capital value and all the dividends, i.e., surplus cash. Of course, many decisions that lead to how much dividend is issued and the rise in the company’s capital value are made at an annual general meeting by those investors. Conflict of interest anyone?


In the general meetings of a typical company today there is no stakeholder class representing the needs of the children, grandchildren, or even the insects we all depend on for life; no class representing the needs of the staff, customers, suppliers, or any other stakeholders; and with an equitable level of voting power to balance the shareholder class. No one can say \emph{“Sorry, that option costs our stakeholder group too much, even though it benefits your stakeholder group enormously. No.”}


Is it any wonder then, that, when faced with a choice between two or more options, all too often the negative consequences are borne by those stakeholders excluded from the decision in the AGM? The staff, customers, suppliers, or future generations? 


Now imagine two centuries of decisions in millions of companies. Each one small and insignificant, but together adding up to the crises threatening humanity today. I have concluded over the past ten years that the apartheid-like separation of stakeholders\index{stakeholders} is the common factor in South Africa and in business. This story of separation sets the stage for decisions that are contrary to the needs of all the non-financial stakeholders. Climate change and the need for the sustainable development goals is an outcome of this story of separation.


I have hope because I believe we can reapply in business today what worked in the transformation of apartheid South Africa\index{South Africa!apartheid} before. Rewriting the stories\index{stories} that shape our reality will give us a better reality, and probably actuality too.




\section{Incorporation and ownership}
\label{section:slavery}
\begin{longstoryblock}
I (Jack) was surprised and initially worried when I learned that copies of one of my books were being downloaded and printed in Singapore without my knowledge or involvement. Looking through the lens of property and my need for money, I saw theft.


But, I wrote the book as a tool to do the job of disseminating insight into economics, from where it was abundant (in me) to where it was scarce (in economics students). Through this lens, these people were very kind souls giving me their time and effort increasing the power of my book to do the job I wanted it to do.


Given my perspective on how the economy is changing, and the need for multiple lenses to look at the economy through (not just the neoclassical lens) my initial trigger quickly turned into pleasure. This is what the rapidly emerging copyleft\index{copyleft} legal framework brings.


And I recognized that knowledge generates wealth when it multiplies through sharing.
\end{longstoryblock}


Property law\index{property!law} in nature\index{nature} gives us an interesting comparison. There are many solitary animals that can be regarded as owning their territory. If any other animal that competes with them for the same prey enters their territory, they go to war. Compare this with the story of ownership used by ants. All regard themselves as members of the colony, no ant claims individual ownership of anything. The colony is the living entity. 


If we want a viable planet at 8 billion people (vs. only viable at under 1 billion) we need an economy based on all human beings, even all life on earth, as one living entity. We are now at a point where either all humans learn to thrive together, or we all experience a collapse. 


The reality you are experiencing today is a consequence of the stories\index{stories} of ownership\index{ownership}, coupled with the power to enforce them\cite{vant-institutions}, inappropriately applied. It's time that we all brought these hidden stories into the light of day, and looked critically at them. Do they fit the dominant stories of our era? Are they shaping realities that enable you to thrive? You'll learn more about how to rewrite stories in Chapter~\ref{chapter:who-am-i-meaning}. 


In most countries in the world, concepts of ownership can no longer be legally applied to you as a human legal person. In law, global, and in almost every country in the world, slavery is illegal. Slavery\index{slavery} is defined today by the ICC\cite{allain-slavery} following the original Slavery convention of 1926 as:


\begin{quote}
“Slavery” means, as defined in the Slavery Convention of 1926, the status or condition of a person over whom any or all of the powers attaching to the right of ownership are exercised, and “slave” means a person in such condition or status.
\end{quote}


Enslavement of any human legal person is illegal in practically every country\footnote{Despite that, we are still dealing with a global problem with human slavery\index{slavery}. In 2016, on any given day, 40.3 million\cite{ilo-slavery} people were classified as enslaved globally. The actual number may be higher today. The majority (24.9 million) were used as forced labour; 4.8 million in sex work, and the rest on farms and in factories. Forced marriage accounted for another 15.4 million. Over the previous five years a total of 89 million people had experienced modern slavery for some period of time. You have quite likely bought something partly manufactured, harvested, or built by someone enslaved.}. But we are still applying the concepts of property to the abstract group of human beings we call a company.


In Section~\ref{section:institutions} gave you compelling evidence that property law \index{property!law} is certainly not always the best way of maximising the value\index{value} of something. In almost all countries in the world, children are no longer classed as the property of their parents. Instead, a completely different meaning\hyp{}making story is the basis for the institution of childhood. This meaning\hyp{}making story tells us that being a child has the same basis in freedom as any other legal person, But recognises that this free person lacks in certain cases the capacity to take certain decisions. Parents, whether biological or appointed to take the function of a parent, then take the decision on behalf of the child’s best interests. Not themselves. 


Imagine, for a moment, that you had been born into a country where you are legally the property of your parents. The dominant meaning\hyp{}making story was that by anchoring in law you as the property of your mum and dad would be the best way to maximise your value. All the rights of property law, giving them sole authority over what to do with you, including the decision to sell you to whoever they choose for whatever price they negotiate. And at that point losing any further right to have any further say in your life. 


Even from a purely economic perspective, we already have compelling evidence that wealth generation, even in a capitalist economy, is maximised through your freedom\index{freedom}. So we all see owning someone as dumb from a purely economic, wealth maximisation frame of reference, regardless of any judgements against a moral frame of reference. 


In fact, the end of slavery was largely driven by the advent of cheap fossil fuel engines, and the subsequent realisation that the economy was better off if everyone was paid a salary for their work, which they then used in the economy to progressively raise their standard of living, after many decades of unsuccessful campaigning against slavery based on moral motivations anchored in the unique value of each human being as a meaning\hyp{}making being. I believe the economic, ‘what’s in it for me’ case will lead to faster, broader action than the moral case, because it is inclusive of all, and does not create divisive blame-defence games.


Since companies are in law classed as non-human legal persons, in 2010 I (Graham) began asking myself how the story of the company as property might be shaping the reality that we experience in the world today. You can read more about how to apply the concepts of freedom fully in creating a free company in the white paper that I wrote between 2010 and 2013\cite{boyd-msc-msv}. Another recent paper supporting much of this has just been published by Common Wealth UK~\cite{lawrence-commoning-the-company, deakin-corporation-as-commons}.


You will read in Section~\ref{section:growing-living-organisation} my perspective that seeing an organisation\index{organisation} as a living being, with its own meaning\hyp{}making capacity, is an essential lens\index{lens} and frame of reference\index{frames of reference} to use in grasping what your business truly is, and to guide your choices for it to be fully productive. How on earth could we ever have believed that applying the meaning\hyp{}making stories of property, rather than the meaning\hyp{}making stories \index{meaning-making stories}of freedom\index{freedom}, could be the basis for businesses, the second fundamental building block of a capitalist economy? (You are the first fundamental building block of a capitalist economy. You as a worker, consumer, and investor.)


Treating companies as property leads directly to many people to burn out at work, become disengaged, or spend more than 50\% of their effort at work protecting themselves, at huge cost. At the global level, climate change and other injustices built into our economy are a consequence of too.


I came to the conclusion, the more I read about this and the more I thought about this, that applying the concepts of ownership to non-human legal persons was having significant  consequences on everyone.


Think about a company where all the staff, from the CEO down to the most recent and junior recruit, decide to embark on a 20-year plan to become a net positive, regenerative business. The current company owners, perhaps a few wealthy and globally minded individuals, agree. But, five years into the plan, a couple of these owners sell, and another passes away, their children inheriting ownership.


A pension fund\index{pension}, private equity fund, or almost any other kind of investor buys all the shares. They have also bought the dominant governance power over the company. They have full power to rip out all regenerative decisions and replace them with decisions designed to extract the maximum possible “rent”\index{rent} from their property in the short term. In the worst case, they may deliberately strip out all value from the company over the following quarters, leaving a hollow shell.


None of the staff, nor any of the suppliers, customers, or cities that supported the company's success through direct and indirect services, nor even the natural ecosystems that enabled the company to generate financial wealth, had any governance power in that decision. When the company is sold, the staff are sold with it. 


Of course, in principle the staff could all walk away. But in practice, some of them may have been required to sign legally binding contracts forcing them to stay. This is especially true in modern knowledge-based companies. Other staff may have had financial commitments, including families to support, and live in towns where this is the only company with the kind of work that they are qualified to do, that made it impossible for them to find any other source of work and income.


These consequences carry with them shadows of the concept of property\index{property} to the human legal persons that are part of the company.


All of this is only possible because companies today are structured in a way that applies some or all of the concepts of property to non-human legal persons. And as you read in Chapter~\ref{section:living-organisation}, regarding a company as a living being, with its own meaning\hyp{}making capacity, makes a lot of sense and gives us a very effective way of understanding how an organisation can thrive, engage in society, and truly take accountability for the consequences of its decisions and actions. And so it seems only natural that the next stage of emancipation that has, over the past centuries, emancipated all human legal persons, is now extended to groups of humans and the non-human legal persons that they are part of.


Think of a scenario where the dominant story shaping our reality is that companies are the property of the investors providing the financial capital. Where only these investors have the right to buy or sell the company, and take decisions about the company as a whole? Now, can you see any antifragile\cite{taleb-antifragile} adaptive ways of structuring businesses and our economy in this scenario that avoid outcomes like our climate crisis and all the other injustices of our economy? I cannot.


So long as only the representatives of the needs of financial capital hold all the power in the company general meetings, without any equitable balance from all the other capitals, I do not see any way that individual businesses will take decisions capable of regenerating all the capitals that human life depends on. And, if the company as a whole is also not free of property law, I do not see any way of having a truly antifragile adaptive economy\index{economy!adaptive} that would do the job of provisioning\index{provisioning} for all, regardless of how our context and global needs evolve.


Think about how stories\index{stories} are what we use to make meaning, and thereby shape the reality that we experience out of the actual raw material. Think about how even a very small shift in a story can be amplified down through a chain of stories, leading to truly horrendous outcomes.


Whilst it may never be provable mathematically, I believe it's plausible that most of the 40.3 million people enslaved in 2016 would be free, and living in an economy that did the job of provisioning properly for them as well, if no part of property law was applied at all to businesses anywhere in the world. 


Explore these ideas for a while. I believe that we cannot address the adaptive challenges of climate change and the SDGs if we continue to apply any or all the concepts of property to our businesses.


One last thought here. Is it possible to use the stories of property to deal with some of the problems that emerge from the stories of property? Neoclassical economists claim that this will  completely address all the root causes of our global crises. If the atmosphere is owned, if nature as a whole is owned, that will enable us to take care of the atmosphere and stop pollution. I doubt it. 


This reminds me of physicists holding on to classical mechanics, attempting in the decades before quantum mechanics to get it to work for subatomic particles. They could get the equations to work, for a while, with a lot of effort. The equations became more and more fragile,  more and more limited to special cases. The new quantum mechanics paradigm transcended and integrated all the old stories of classical mechanics, and so gave physics antifragile, simple, elegant ways of grasping what we could say about how the world works.


We have enough evidence for the usefulness  of property in some domains as part of capitalism\index{capitalism} (the experiments of the USSR, Eastern Europe, China, etc. demonstrate the value). So I am not, in any way, arguing for the elimination of the meaning\hyp{}making story of property, just as no physicist would ever claim that quantum mechanics\index{mechanics!quantum} argues against the use of particles. Quantum mechanics does not say physical particles are never useful, and that everything should be replaced by waves. Physicists\index{physicists} use both particles and waves without any argument, choosing which one delivers the best results on a case-by-case basis.


The lesson from quantum physics is that property and freedom are complementary pairs, a dual story, not opposite ends of a single story. Somewhat like the truth square on Page~\pageref{figure:truth-square}, the meaning\hyp{}making story of freedom is very different from that of property, not the opposite story. Not owned is very different to freedom, because not owned just means that no one has yet claimed ownership. Like particles and waves in physics are a complementary pair, not simple opposites in a binary choice. In a regenerative economy we will have figured out how to use both meaning\hyp{}making stories simultaneously.


The new case that we are confronted with today is the global adaptive challenge of our climate crisis and the 17 SDGs\index{Sustainable Development Goals, UN 17}. I don't believe that only using the stories of property and the laws they create is capable of giving us regenerative businesses and an economy\index{economy} that does the job of provisioning\index{provisioning} for all. For example, if property law is applied to you, i.e., you own yourself, maybe that will eliminate slavery and give you freedom. Using a story and laws of property\index{property} to protect freedom, capitals, etc. seems to me to be a fragile and unworkably complicated way. Even if the owners include the next seven generations, or mountains, streams, and oceans. 


The recent decision in New Zealand that Te Urewera National Park, Mount Taranaki and Whanganui River have full legal personhood, including freedom, recognises that property law\index{property!law}, even if you are deemed to own yourself, falls short.


However, given the dominance of the stories of property in our reality today, hacking property law in defence of freedom can be a very good bridge. Until we recognise companies as living beings, as Mount Taranaki is by the Maori, and extend our stories and laws around freedom to non-human legal persons.


The Purpose Foundation’s \index{Purpose Foundation}excellent work in pushing for the incorporation of companies that own themselves is a very necessary and excellent step forward. The FairShares Commons goes beyond this, yet still works in existing company law, shaped as it is by stories of property, to protect a company’s commons. Wherever you can use existing stories and their laws in novel ways to build bridges to the future as it emerges, do so.


You may be thinking that this is all unachievable utopian thinking. So did most in the old slave-powered economy, as they could not see the transformation that cheap fossil-fuel powered machine labour was bringing, because they had insufficient fluidity in transformational thinking. We are now entering the era of zero marginal cost\index{marginal cost, zero} energy, not just cheap energy\index{energy}\cite{rifkin-zero-mc}\index{Rifkin, Jeremy}; maybe the pattern is about to repeat itself, now at the scale of non-human legal persons.




\section{Myths}
\label{section:myths-of-incorporation}\index{myth|(}
Myths are prevalent in the worlds of business and economics. Like all other stories, these myths shape the reality that we end up experiencing. They are meaning\hyp{}making stories in economics and business. In some countries they are no more than beliefs with no legal solidity, in others they have some legal concrete. So the experienced reality they create for us is more like clouds forming castles in the sky that we simply accept as given, not physical buildings we can safely thrive in. ( Section~\ref{section:stories} covers what I mean here.)


The original Greek Mythos referred to fantasies about the gods, used by the powerful rulers to shape individual behaviour, and society as a whole, maintaining social cohesion. These myths stated the origin of everything, and were inherently unquestionable.


It is time to question the myths around incorporation.


Perhaps the biggest myth is that the state avoids entrepreneurial innovation and risk, leaving it to private investors and businesses; actually, many of the innovations we depend on began with public sector investments, not private, as described well in Mariana Mazzucato’s\index{Mazzucato, Mariana} book \emph{The entrepreneurial state}\cite{mazzucato-entrepreneurial-state}. Just take a look at your smartphone: the key elements were all government-funded, as was the web, invented at CERN to make global physics collaborations easier, not by a company to make money through you sharing gifs of cats. 


I summarise below some of the myths. You can read more about these myths across most of continental Europe, the UK nations, the US states, and almost all other jurisdictions, and find all the legal references, in the statement signed by a number of leading academics\cite{veldman-modern-law} and in Lynn Stout's\index{Stout, Lynn} book\cite{stout-shareholder}.


\begin{itemize}
\item Shareholders\index{shareholders} own the company and its assets. Not true.


All that a shareholder owns is an intangible bundle of rights and obligations\cite{brown-bbaequity} (See Chapter~\ref{chapter:who-is-your-organisation-incorporation} for details), represented by a piece of paper, or a string of zeros and ones held on some hard disk in some storage centre, that gives the shareholder various rights, for example the right to a share of any dividends the company decides to distribute. In most jurisdictions, though, the shareholder has no right to mandate a dividend payment. The company directors decide whether there is sufficient surplus to pay out a dividend, and at most the shareholders vote collectively yes or no to that dividend.


The company, as a distinct legal person, owns its own assets, is responsible for its own liabilities; and anyone acting as if the company were theirs can be held legally liable. A company’s assets are locked in, precisely so that the company can follow long-term business strategies; shareholders have no right to demand a company give them its assets.


\item Directors of the company are there to serve the shareholders. Not true.


Company directors are legally bound to serve the company’s needs to the best of their ability, they are not the agents of the shareholders. And because every company is a legal person (a clear legal actuality, not a fiction) distinct from any other legal persons like you or me associated with the company, including any shareholder, the company directors are not bound in any legally enforceable sense to maximise TSR. This “business judgement rule” exists in every major jurisdiction, and can, in principle, span all time horizons, from short to long. 


The belief that a company exists to maximise TSR is a consequence of neoclassical economists’ lobbying, and pressures from markets, some investors, potential hostile takeovers, and executive incentive schemes. If you look in the company law of England and Wales, and the company laws of different US states, you will find that the law as written is generically about the company’s purpose is to achieve its stated objectives or purpose. Sadly many founders omit to define a very clear set of objects or purposes for the company, and so the neoclassical interpretation and pressures win when things come to court.


\item Buying shares is the only way for a stakeholder\index{stakeholders} to gain the right to vote in a general meeting, and voting rights must be tied to investment. Not true.


UK law, for example, refers to the members of the company, not the shareholders. So anyone who is entered into the register of voting members is eligible to vote\cite{mcgaughey}. One route to being entered in the register of members of a company lies in buying shares. Equally, the company may at a general meeting declare that any individual satisfying certain qualifying criteria is entered into the register of members and acquires the right to vote.


Most jurisdictions also have routes for investors to invest without acquiring voting rights, and for voting rights to be acquired with no investment, or minimal investment. If direct entry in the register of members is not possible, a company can easily place criteria on a specific class of voting shares based on all kinds of capitals invested by all kinds of stakeholders.


\item The company’s sole objective, and the focus of the AGM, is to maximise TSR. False.


Earlier in this chapter I wrote about this; I’ll add here that this shift in meaning\hyp{}making story, from the earlier ones seeing a company as very much part of the broader social fabric, has directly led to short-term thinking in companies, long-term wealth destruction for most stakeholders, the rising gap between the wealthy and the rest of society, and especially hides the non-financial risk-capitals invested by all stakeholders. 


\item The market plays a big role in funding company operations, and is the best judge of efficient use of resources. False.


Most of the capital that companies raise is prior to listing, not on the market after listing. The market enables those first investors to exit (so it does have a vital enabling role). We need it, like a dog needs a tail to wag; but today we are letting the tail wag the dog.


This myth became dominant in the second half of the last century, and has led to fatal flaws in our economy. Especially, it has taken us into a blind, lemming-like rush to the cliff-edge of our current degenerative economy. All direct stakeholders with commitment to the company, not the investors on the stock market lacking commitment (especially not the automatic trading algorithms holding shares for a few minutes), are the best judges of what is in the interests of the company.


Equally harmful, this myth has led to the emergence of so-called professional managers who stay with a company for such a short period of time that they are gone before the consequences of their decisions are seen, 


\item The company is a set of abstract contracts between owners. False.


Or, at least misleadingly incomplete. Recall, the word company itself comes from the Latin roots meaning \emph{with bread} and is first and foremost a group of people who bond together into a community, building strong social relationships, in order to improve their capacity to thrive. The intangible social organisation that a company is, is far more important to business results than any tangible bundle of contracts. (see Chapter~\ref{chapter:who-is-your-organisation-base}.)


\item Money\index{money} is neutral, and the only currency\index{currency} to use for all kinds of value in business. False. 


Money, defined as positive interest bank debt, is one of many currencies we can define and use (See Section~\ref{section:money-vs-currency}.) Each currency distorts and biases decisions if it is used inappropriately; and money is only appropriate for financial and some manufactured capital goods, not for natural or human capitals. We need to use other currencies there. Even a net present value calculation gives different decisions if you merely change the currency used from money to, say, the Terra\cite{lietaer-money-and-sustainability}.


\end{itemize}


If you believe these meaning\hyp{}making stories, you will experience a reality\index{reality} that concretises them. If directors believe in the myth that their duty is to maximise TSR, forgetting that the legal personhood of the company puts it onto an equal footing to themselves and the investors as legal persons, then they will act exactly as if it were an absolute truth.


There are feedback loops in the power flows that lead to the same outcomes as if these myths were true. For example, if only the voting shareholders choose who is a director, and the directors need to earn their money or status through this role, and the incentives are designed to bolster these myths, then the shareholders can use this power to get their way.


These myths are so deeply woven into the meaning\hyp{}making stories active in business and economics that they have become invisible. These stories form one of the invisible hands shaping the reality you experience today in your economy. Harnessing the power of business to rise to the challenges all of you reading this book (and everyone else connected with you) are facing, requires all of us to uncover these hidden stories that shape our reality, and to then pull away what is giving us these crises, keeping what is still useful in overcoming them.


The most important to keep is the meaning\hyp{}making story of communities of people collaborating in a larger endeavour acquires itself the rights in law of a person, to become a non-human legal person, like you, a human legal person. 


Inventing this story, and then for society to incorporate this story\index{stories} into the larger set of stories that shape our collective reality, has been a huge step forwards. Imagine that you were alive 400 years ago, perhaps making wagon wheels, and somebody tried to tell you that you should separate your wagon wheel business as an independent fictitious legal non-human person from yourself as the legal human person doing the artisanal work. You probably would have been completely unable to grasp how the company could possibly be an independent entity to you, and yet still fully dependent on you to speak on its behalf.


Just as quantum physics recognised that there was value in both particle and wave physics, and that the leap forward was to transcend and include in a radically new way the value of both, so too do we need to transcend and include here. As you read in the section on capital growth in a commons\index{commons}, a highly functioning commons takes care to grow the capital base. Maximising TSR has value and is a dumb idea, as Jack Welch said.


Integrate the complementary pairs: we need to keep the powerful capacity of business to multiply capital; \emph{but must expand that to multiplying all capitals}\index{capitals}. We need to keep the power of business to maximise TSR\index{Total Shareholder Return}, \emph{and expand shareholder to include all stakeholders}, from you and everyone like you, through to the planet's natural ecosystems as a stakeholder. 


Then the tools of business for multiplying financial capital\index{capitals!financial} are put to work regenerating all other capitals. Such a regenerative economy will have high adaptive capacity, will be antifragile, and can give us what we need to address our 17~SDGs\index{Sustainable Development Goals, UN 17}, including the climate emergency.
\index{myth|)}


\section{Steps in the right direction}
There are numerous excellent initiatives underway to transform business into a force for good for all, many of which began decades ago. Yet they have changed business far too little far too little for us to have hope that we can address our global crises in time if we just continue trying harder to do what we have been doing.


An early, and vital, step in this direction has been freedom in the software world. Also called open source software, the big idea is that intellectual capital multiplies when it is shared freely.  Today's Creative Commons licences, Wikipedia\index{Wikipedia}, Android, Linux, and much more have their roots in the realisation of people like Richard Stallman\cite{williams-free} of the need to legally protect freedom. This led to copyleft laws which protect the freedom to use (as opposed to copyright laws which restrict the freedom to use), such as the GNU public licence, and Creative Commons. Richard Stallman started the GNU project to create a completely free computer operating system in September 1983, and in 1985 he started the Free Software Foundation to protect intellectual freedom. 


The Triple Bottom Line\cite{elkington-triple}\index{triple bottom line}, Reporting3.0, the Integrated Reporting Initiative, Economy for the Common Good, Benefit Corporations, Purpose Companies, Transition Towns, b-corps and all the others are essential first steps towards the new reality we need. Equally, the trend towards developing leadership ethics, alternate ways of organising like Holacracy\index{Holacracy}\cite{robertson-holacracy}, sociocracy\cite{rau-sociocracy}\index{Sociocracy}, and Deliberately Developmental Organisations\cite{kegan-everyone} are essential components.


There are a whole range of excellent steps at the intersection of governance, society, non-profit and (social) enterprise, such as the associative democracy of Paul Hirst, or the work of Graham Smith and Simon Teasdale~\cite{smith-teasdale-associative}.


But even the most responsible leader using the most powerful of these tools can, at best, make small, local, and fragile changes. The apartheid-like context of separation in the legal constructs of our current incorporation leaves the door wide open for narrow self-interest to pull these improvements back to our current status quo, because only investors have legally mandated decision power, and only the investors have the majority benefit from those decisions; all others are separated from power and benefit.


Even though other stakeholders\index{stakeholders} carry many of the costs of those decisions, sometimes for many generations into the future.


We need gentle ways of changing, gentle enough to keep society stable; but these will all get pulled back to the status quo, unless we also remove the root cause of the gravity pulling us back. Pressure is growing to use forceful change, even though it will destroy society and even more of nature, because we are not seeing clearly enough the difference between myth and actuality. 


A growing number of people are now seeing this. Evolutesix (Graham) has starting up a startup creator and accelerator to grow startups using the Adaptive Organisation\index{Adaptive Organisation} composed of the FairShares Commons\index{FairShares Commons} legal framework, with self-organising, and developmental approaches  (see Part~\ref{part:organisations}).


South Africans dismantled their constitution, the source of their gravity\index{gravity} shaping everyone's behaviour, in five years. There is nothing to stop us rewriting equally fast the foundations we build our companies on.


This will enable us to transform from the capitalism\index{capitalism} we have today to a fully regenerative capitalism, one that multiplies all capitals, from the entire planet’s natural capital to your own self. For this to work, we need to clarify why there is no tragedy of the commons, so long as you construct it consistently.


\section{Commons}
\label{section:commons}\index{commons|(}


In this book we call a common resource a commons if:


\begin{itemize}
\item everyone who benefits from it is also part of taking care of it;
\item governance is broad, inclusive, where all relevant voices and perspectives are taken into account in a decentralised, non-hierarchical way;
        \item there are mechanisms to protect the commons against predation.
\end{itemize}


A commons doesn’t need anything external for it to work well; everything it needs is internal (except perhaps solar energy). 
\subsection{Tragedy of the commons}
In the original article by Garrett Hardin\index{Hardin, Garrett} on the \emph{Tragedy of the Commons}\cite{hardin-tragedy}\index{Tragedy of the Commons} the key heading was \emph{The tragedy of freedom in the commons}; and mis-perceived open access and the commons as the same. 


They are not. In a highly functioning commons, only the members of the community caring for the commons can access its benefits. This has functioned well across most of human history, because of various social norms, such as the biases you will read about in section~\ref{section:biases}.


Tragedy there means something inevitable, and harmful. Yet there's more than enough evidence that if all the characteristics of a well-functioning commons are in place, there is no inevitable harm. In fact, the climate emergency is evidence that you are likely to have inevitable harm when the stories of property are applied.


Hardin's intent was to show how self-interest without any boundaries\index{boundaries} to freedom would lead to a tragedy. Very true, and one of this book’s central themes. 


But with boundaries, with appropriate legal and social protection of the commons, protection as strong as property law and social norms are today, a commons based on reciprocal freedom and custodial obligations is a more viable approach to regulating individual self-interest\index{self-interest} in order to build an economy that does a better job of provisioning\index{provisioning}. It is a more viable approach than property is to meet the different needs of each individual across the community as a whole, and use the different capacities of each individual to protect and grow the wealth of all.


\subsection{A highly functional commons}
Elinor Ostrom's\index{Ostrom, Elinor} work on highly functioning commons\cite{ostrom-governing} and the recent book by Michel Bauwens\cite{bauwens-peer-to-peer}\index{Bauwens, Michel} describe the elements critical to an antifragile\index{antifragile}, highly functioning commons. 
\subsubsection{Highly functioning commons: governance}
Perhaps the most important capacity of a highly functioning commons is self\hyp{}policing. 


Governing a commons for the good of all of its members, including future members as yet unborn, needs to be institutionalised. The norms of interaction, of caring for a commons, of decision-making, become part of the overarching stories that every member of the community stewarding and benefiting from the commons adopts as part of their own personal stories shaping their reality.


In a highly functioning commons everyone benefits from the surplus it generates, everyone cares for it, and everyone polices it.


There is no need for any outside regulation, because all the necessary stakeholders are already part of the shared benefit, caring, and policing. 


Typically, any attempt at outside regulation weakens or even destroys the capacity of a commons to function. You can look at a commons as a superb example of how a highly functioning economy with zero external government regulation can exist. A commons can please both left and right views in politics.




\subsubsection{Highly functional commons: external defence}
For a commons to thrive over long periods of time, it needs sufficient defensive power to protect itself against invasion. Once this meant castles and knights; today this mostly means law, lawyers, and accountants. Although sometimes it still means physical power.


For a company to function as a FairShares Commons\index{FairShares Commons}, and groups of such companies\index{ecosystems!companies} to function as a commons ecosystem, it's mostly about carefully crafting your articles of incorporation. Making sure the company is structured sufficiently clearly so it can defend itself against being enslaved.


The commons across North America functioned well for centuries, in part because of power balances across all the communities that cared for and benefitted from each regional commons. Then came large sailing ships, muskets that could kill at unprecedented distances, cannons that could demolish fortifications, and diseases new to their immune systems. The stewards of a commons had no power strong enough to defend their commons against a hostile takeover. Much as you may lack the power to defend yourself from a mugger.


They also came with stories about ownership,\index{ownership} lenses that saw only the weakness of a commons, not the advantages; and judged it as inherently primitive and inferior. So the combination of the stories shaping the reality of the Europeans arriving in the Americas, along with their firepower, created the reality we have today. 


But this is a consequence of what the lenses cannot see, not absolute evidence that the stories of property create better realities than the stories of a commons in all situations, everywhere, at any time, and for all people.


\subsubsection{Highly functioning commons: fiduciary duty}
Just as a well-functioning business has very clearly defined expectations on the executives carrying out their fiduciary duties with excellence, with various mechanisms to enforce it, so too do those people acting in executive-like roles in a highly functioning commons have clearly defined fiduciary duties\index{fiduciary duty}. These duties apply to the full breadth of all the capitals that are part of the commons, and to all the stakeholders, whether they are alive today, yet to be born, and even those who have passed on and have become ancestral members.


In a small commons pretty much all roles are simultaneously filled by all of its members. Everybody is caring for the commons in all the ways that it can be cared for without much specialisation. This is very much the same as you'll find in many small start-ups or successful long-running small cooperatives.


As any organisation scales\index{scaling} beyond the size of ancient family units (20 to 40 people) and then beyond the size of villages (200 to 300), new structures, hierarchies, and differentiations need to come in at each scale. Most organisations around you today use only one of the many ways of differentiating and layering. (Part~\ref{part:organisations}) But scaling is seldom linear, and in organisations completely new phenomena can emerge at each scale that need to be dealt with in fundamentally different ways\cite{west-scale}. 


The traditional management accountability hierarchy is not very suitable for a larger commons, especially not a global one that is capable of rising to the challenges we face today, such as the climate crisis. However, the human capacity hierarchy, combined with a functional accountability hierarchy and consent-based ways of filling specific roles of Part~\ref{part:organisations} can be used to build a commons that will work at all scales.
\subsubsection{Highly functioning commons: capital growth}
Typically a commons comes into existence because some resource, valuable to all of its members, is below some threshold of supply. So it makes sense to work together so that everybody is provided with an adequate amount, nobody has too much at the expense of somebody else having too little, and that that resource continues to be available with sufficient abundance for future generations.


A highly functioning commons explicitly takes the needs of future generations into account, and it’s usually had a long enough existence to also give some idea of the kinds of natural fluctuations that it is subject to. So highly functioning commons put effort into multiplying the capital. That way, even if they have a run of bad years and the capital drops to much less than what it had been, over the long run the amount of capital in the commons will steadily grow.


This is no different to how modern businesses are designed to multiply financial capital. Any modern business can run perfectly well as a commons and continue to multiply financial capital well.


But today, we need new kinds of commons. We need new kinds of businesses, that multiply all capitals. We need all-capitals commons, where all capitals are multiplied. And we need part of a nested hierarchy of ecosystems\index{ecosystems} of commons producing an economy that does the job of provisioning for all.
\subsubsection{Highly functioning commons: benefit}
There are many different ways at each scale to decide how members of a commons benefit from the wealth it generates. Typically, though, a highly functional commons will use some kind of collective decision-making process that ensures the valid needs of all members are adequately met. A process that strives to be fair to all, and that if fairness is unachievable, at a minimum is equally unfair to all.


In particular, the decision is not taken by a small stakeholder group that happens to have power for some reason disconnected from the commons and the other stakeholders.
\index{commons|)}
\section{Free FairShares Commons}
Buckminster Fuller\index{Fuller, Buckminister} said \emph{“You never change things by fighting the existing reality. To change something, build a new paradigm that makes the existing paradigm obsolete.”} The legal construct we use today is an expression of an old reality, created by a story well past its sell-by date. This story is about the culture, beliefs and actual scarcity of money in the past. High time we used a modern story matching today’s scarcity: the resource scarcity of natural capital, human innovation and attention. A story creating the reality that we need to thrive within the constraint of the interdependence of life on planet Earth.


Fully adaptive organisations\index{Adaptive Organisation} based on Free companies or FairShares Commons\index{FairShares Commons}, with at least self-governing for roles and tasks, and with developmental human practices, may well be the minimum viable solution. Let’s look at what pragmatic actions you can already take for another step towards this minimum viable solution. What can you do to build a new business paradigm that addresses climate emergency\index{climate!emergency} and other injustices?


Central is adopting inclusive legal incorporation constructs. Those of the Purpose Stiftung (a German and Swiss foundation promoting the for-purpose company form, which is also a form preventing sale of the company), and even more so those of our FairShares Commons\index{FairShares Commons}, are sufficiently proven and ready to be adopted by those forward-looking companies that want to show the way.


These build on proven approaches used for decades, for example by Carl Zeiss, Robert Bosch, Mondragon, etc. Even Visa\index{Visa Corporation}, in its first decade, was able to transform the completely dysfunctional relationships between players in the emerging credit card system because it included all in a democratic, citizenship-based governance. And its first CEO, Dee Hock\index{Hock, Dee}, is clear that this way of incorporating was the major reason why Visa was so successful.


You can build a FairShares Commons now in most countries. At least one that is good enough for now. You do not need to agitate for a change in company law. Using existing company law you can easily create distributed, equitable governance and wealth sharing, such as the FairShares Commons Incorporation does.


In a FairShares company, investors still get voting rights and a share of wealth. Anything that separates investors from power and wealth is just another kind of separation, and will likely lead to harmful consequences. Few investors are fundamentally bad, most have value to contribute, but are pulled down by the system’s hidden gravity\index{gravity}, just as everyone in South Africa was pulled down by the gravity of apartheid\index{South Africa!apartheid}. For any new approach to work, investors must have just the right amount of power and reward.


In a FairShares company, staff, suppliers, customers, any relevant stakeholder group, even perhaps the city and country the company is based in, can all qualify for voting rights in the general meetings, and a share of the wealth generated. These rights are not based only on buying shares with money\index{money}. Rather, these rights are earned through engagement, and the investment of all capitals: financial, human and natural.


The big idea here is the same as has underpinned our modern approach to governing countries: citizenship. In the Free company, anyone that satisfies the criteria of “a citizen of good standing” has the right to engage in governance. Few today would want to go back to the days when kings and queens used countries, along with the people in them, as their personal property; the same will be true in the future for companies. So every stakeholder group has enough power to keep the costs and benefits of decisions balanced across all stakeholders\index{stakeholders}, including future generations.


You may wonder how you can run a company if so many people have a vote. No differently to today. All operating decisions are still made by the executives, not the shareholders. But big decisions, like selling the company, or shifting to renewable energy, are taken by all stakeholders in the AGM or via the board.


Experience over the centuries in cities and nations shows, without any doubt, that when all are citizens, when all have rights to freedom, and an equitable share of governance and wealth is generated, then all are most likely to prosper long term. The transformation of city and national governance from an aristocratic elite to a citizenship democracy over the past centuries in much of the world has addressed historical injustices and created much good in our lives at the citizen, city and national scales. To rise to our new challenges and build a better world we need to repeat what has worked before, now at the non-human legal person, worker, investor, etc. levels.


Of course, not all old injustices are done and dusted; there are many that we still need to address. And globalisation within today's flavour of capitalism has led to new injustices. 


Just as the ending of apartheid was the essential step of this trend to freedom in South Africa\index{South Africa}, I believe we now need to take the next step in companies if we are to address social injustices like climate change.


This step brings you to freedom\index{freedom}, the freedom to be yourself in your full strength, the freedom to develop yourself into who you can become\cite{sen-development}, as the pragmatic foundation of an economy free of the injustices of climate change, etc. It also brings freedom to the companies that are the building blocks of our economy.


When everything participates in governance and wealth sharing, including cities, nations and the natural environment, then it’s a small step to see the company as a common good over multiple generations. Which requires freedom for the company itself. 


Free of ownership, free to develop to its full potential as an element of society. Compare with nations and cities today; they work better because they can no longer be sold by a king to another king. Yet we still have structures allowing the shareholders of a company to sell the company.


Freedom also means the company is free to die when it is too old, or no longer has a niche in the social ecosystem. Which means that healthy evolution can work in the business ecosystems, not the harmful culling you see today.


Some ask me \emph{“does freedom mean eliminating regulation of business?”} 


Certainly not; what freedom does is shift most regulation from the slow, uninformed, indirect, external, macro regulation of today to fast, informed, direct, internal, micro regulation. 


Once all stakeholders (from investors, staff, etc. to entire regions) affected by a general meeting decision are in the general meeting with power, there is far less need for the slow, clumsy regulation of state bodies. All benefit from immediate, internal micro regulation.


Ending apartheid in South Africa could only really begin once all South Africans were free and included in governance. (South Africa still has a long way to go in addressing all injustices inherited from apartheid. Governance power is still not yet well distributed, and the share of wealth even less so.)


We only begin ending the global injustices of today’s capitalism once all of a company’s stakeholders are included in governance and wealth sharing; and the legal person we call a company is as free as the natural persons you and I are. 


Even if all companies were FairShares Commons tomorrow we would have a long way to go before the unjust consequences of our current economic and political choices are addressed. But we’d be going a lot faster than we can now, and humanity needs us to go way faster than we are now.


The FairShares Commons\index{FairShares Commons} will enable us to rise to the global challenges we are all facing, because such a free company gives us the power to stand on the shoulders of giants, instead of reinventing the wheel\cite{williams-free}. 
\section{Legal and economic foundations}
I (Graham) realised around 2010 that the barrier I was hitting trying to making Holacracy work well in my startup lay in both the human and the incorporation dimensions shown later in the book in Figure~\ref{figure:three-axes} on Page~\pageref{figure:three-axes}. I realised we needed a way beyond the paradigm of ownership, and looked at legal research I discovered empty voting (the separation of voting power from financial risk), vote trading, hidden ownership, and how these could turn competitive stock markets into collaborative stock markets. This showed I was on the right track; all that was needed was to take these practices all the way, and structure them to enable us to create ecosystems of regenerative commons companies.


You get empty voting where an investor has voting power without the matching financial risk; for example, the investor buys both voting shares in a company and derivatives protecting them against any loss if the share price drops; or even paying out more than they invested. This means that first asset stripping, and then destroying the company, can be in the investor’s financial interests. It usually at least opens up significant potential risks and costs for all stakeholders, leading to calls for regulation\cite{ringe-europe}.


Equally it can lead to more efficient corporate governance\cite{brav-empty-voting}, along with greater and broader shareholder democracy\cite{yermack-governance}. A lot of attention is paid to executive values and their decisions; but investor / shareholder values and decisions, and how effective they are, are equally important\cite{schouten-mechanisms}. After all, the AGM vote is used by investors to decide on board membership, executive pay, mergers and acquisitions, etc. Which leads to both the consent decision making process and the inclusion of diverse categories of stakeholders in the FairShares Commons, as a way of improving the decisions of shareholders.


The conflict between Telus, a Canadian company, and Mason Capital, a US hedge fund, described by Ringe~\cite{ringe-revisited} is worth reading. Ringe describes well how the decoupling of voting power from financial risk distorts the traditional incentives. Whilst most cases so far have benefited investors at the cost of other stakeholders, a few changes as described in this book turn the same decoupling into a new way of building regenerative ecosystems. 


How eliminating information asymmetry (the FairShares Commons does this) to empty voting and hidden ownership creates a new kind of economy, where social welfare and better market outcomes emerge, has been described by Barry et al.\cite{barry-social-welfare}. Julie Battilana et al. give evidence that hybrid organisations perform better\cite{battilana-hybrid}.


Iwai has described how, much as in quantum physics, all perspectives on what a company is have validity\cite{iwai-persons}, and can be integrated as complementary pairs.


\index{investor|)}