\chapter{How do we know our economy?}
\addcontentsline{toc}{chapterdescription}{An economy is a tool that provisions across multiple capitals, and so nature also has economies. But our current economy is failing us, because of the meaning\hyp{}making stories that have created it. We need an economy that’s fit for our current and future challenges. Yet we cannot control our economy; rather we need to facilitate generative emergence by transforming our existing approaches to freedom, stakeholders, property, money, and capitalism.}
%\addcontentsline{toc}{chapterdescription}{\pagebreak}
\label{chapter:know-our-economy}


\begin{chapterquotation}
Classic economic theory, based as it is on an inadequate theory of human motivation, could be revolutionised by accepting the reality of higher human needs, including the impulse to self-actualisation and the love for the highest values.\\
\raggedleft\textemdash Abraham Maslow\index{Maslow, Abraham}


\centering
If you jump, you might fall on the wrong side of the rope. But if you are not willing to take the risk of breaking your neck what good is it? You don't jump at all. You have to wake people up. To revolutionise their way of identifying things, you've got to create images they won't accept.\\
\raggedleft\textemdash Pablo Picasso\index{Picasso, Pablo}\cite{gardner-creating-minds}
\end{chapterquotation}


The economy\index{economy} is not some distant philosophical entity, belonging to somebody else. It is your economy, it's with you wherever you are, awake or asleep, 24 hours a day. Your economy is Ubuntu\index{Ubuntu} at a global level. You shape your economy, your economy shapes you, and once you've added it all up, 7.9 billion people are shaping each other through the economy.


When you work, you contribute to production; when you buy or get given products or services, you are contributing directly or indirectly to the flow of goods and services in the economy. 


Your economy is nebulous and emergent, changing every time you or anybody else does anything in it. You can understand each part in perfect detail and still have no idea of how it really works. You can only understand the economy as a whole, just as you as a living being can only be understood as a whole. No heart surgeon can ever understand you, no matter how well they understand your physical heart. 


Just like in quantum physics\index{physics!quantum}, there are always inherently unknowable, emergent areas in the economy.


Also, just as in Picasso's\index{Picasso, Pablo} art, many perspectives are needed to understand the economy, even perspectives that appear to be inconsistent with each other. (This is thought form T7 in Chapter~\ref{chapter:who-am-i-sense}.) Really understanding the economy requires a high fluidity in all 28 transformational thought forms \index{thought forms (28)} of Chapter~\ref{chapter:who-am-i-sense}.


Your economy starts with stories,\index{stories} not unambiguous facts. It is something that society has invented and given meaning to by bringing the dominant stories to life. So to understand your economy, not only do you need to have fluidity in sense-making through using all 28 transformational thought forms, you need to have a thorough grasp of how individuals, communities, and societies make meaning through stories that they are aware of and unaware of (Chapters\ref{chapter:who-am-i-base} to~\ref{chapter:who-am-i-meaning}.)


The statements by a large number of leading economists\index{economists} and change agents on economics and business\cite{veldman-modern-economics,veldman-modern-law} show clearly how our current economic reality has been shaped by a set of stories that we have believed. These stories are now so hidden in our norms of behaviour and in our institutions that we've forgotten where they came from. 


Just as physicists learnt a hundred years ago that physics is the study of what we can \emph{say}, not the study of what \emph{is}, so I see economics as the study of what we can \emph{say} about the economy, e.g. how valuable goods and services flow within society from where they are abundant to where they are needed, not what the economy \emph{is}. Central to that is the word “valuable”; value is our stories shaping the reality we experience. Gold, wine, the US dollar, all have value because of the meaning that society attributes to each of them.






\section{An economy does the job of provisioning}
\label{section:economy-provisioning}
If you need food, housing, medical care, banking services, clean air to breathe, or clean water to drink, you judge your economy to be working when it does the job of providing you with them. Your economy is failing you if it cannot provide you with all the fundamental resources you need to at least survive.


In business, especially in disruptive innovation, you ask what the job is that needs to be done, and what tool will do the job\cite{christensen-dna}. The job you might need doing is getting an internet connection from your home office, where you work, to your living room, so you can relax in the evenings. One tool to do the job is a long ethernet cable and holes in your wall. And for many years this was the only tool to do the job, until somebody realised that you could dispense with both the holes and the cable by inventing Wi-Fi. (We use ‘tool’ in this book to mean any way of solving a problem or achieving an outcome.)


The economy is a tool that society has invented to do the job of provisioning\index{provisioning}.


The economy concretises society's stories, giving more value to some things and less to others. These stories are anchored in our underlying human meaning\hyp{}making and biases (section~\ref{section:biases}), creating an inherently subjective reality. 


Since our economy\index{economy} is a tool we have invented, it can become whatever we need it to be\textemdash within the constraints of physics, of course!


The three fundamental building blocks in an economy that we focus on in this chapter are capitals, currencies, and marketplaces. 
\subsection{Multiple capitals, multiple currencies}
\label{section:six-capitals}
\index{capitals!six|(}


A capital\index{capitals!six} is a store of anything that has value to a living entity or organisation of people. We attribute value to it because it supports a living being's capacity to thrive. A currency is anything that enables or represents anything of value when it is put to work, i.e., when it flows from where it is stored to where it is needed and put to work, thereby supporting life.


Often there is a two-way flow, of two different capitals flowing in opposite directions. The marketplace facilitates the agreement on how much of one capital flowing one way is a fair return for a different capital flowing the other way.


The multiple capitals of the Integrated Reporting Initiative are widely recognised as a minimum.
These capitals are all affected, transformed, or even created or destroyed by a company's activities. Each needs one or more currencies to represent the capital when it is stored, moving from one store to another, or being transformed, created, or destroyed. Some will only have a currency of attribution, not one that can be traded or used to store value\textemdash for example, your reputation as a buyer is only a currency of attribution. The currencies will have floating exchange rates between each other, representing the relative value at that point of time in a certain context.


The six capitals, and possible associated currencies\index{complementary currencies!six capitals} in brackets, are:
 
\begin{itemize}
\item Natural (Terra)\index{complementary currencies!Terra};
        \item Human (Time, Energy, Quality Adjusted Life Year\textemdash QALY)\index{complementary currencies!time}\index{complementary currencies!QALY};
\item Social and Relational (Reputation, e.g. on LinkedIn or eBay, connections)\index{complementary currencies!Reputation};
\item Intellectual (Number and ranking of citations, patents);
\item Manufactured (Money, WIR, embodied energy)\index{complementary currencies!WIR}\index{complementary currencies!Energy};
\item Financial (Money, WIR, local money-based currencies).
\end{itemize}


A systemic cause of the problems we have today is our attempt to attribute value to all the capitals using money\index{money} as the only currency\index{currency}, rather than the currency intrinsic to each capital.


Today financial capital is relatively plentiful; it is environmental and social capital that is scarce. In many elements of environmental capital, we are so deep in overdraft that if we don't begin paying the capital back rapidly we will exceed our capacity to even pay the interest. The climate emergency, plastic pollution of our oceans, microparticle air pollution, and many more are all symptoms of this overdraft. To pay them back, we need businesses designed to multiply all capitals, not just financial capital.


All the human capitals are also scarce. Human time, attention and creativity, and human capacity to rise to adaptive challenges are all in shorter supply then we need to be able to pay back our environmental overdraft. The same is true for everything else that is put into the company across the other capitals.\index{capitals!six|)}




\subsection{Nature is an economy}
You can understand your economy by comparing it to nature’s economy. Nature\index{nature} has run the best example of a sustainable economy that we can ever hope to see. The central capital of nature is energy\index{energy}\footnote{This is the one aspect of nature's economy that is open. Every day, energy streams into our planet from the sun in a supply that is effectively endless compared to even nature's longest life cycles.}. 


The difference between a living entity and anything else lies in energy\index{energy}. Living entities harness energy to increase their capacity to thrive. They convert energy into specific types of work in a highly efficient way. Energy is the most important capital in all of nature, including human society and its subset, our economy.


Throughout this book we will look at nested energy economies, from your personal inner energy economy, through your organisation's energy economy, to all of nature as an energy economy. Energy is a vital capital, and fortunately it is one that is renewable for as long as the sun shines. 


The associated biological currencies of storage and trade are primarily fats, carbohydrates, and proteins. These currencies are highly tradable across a broad swathe of nature because nature's economy uses common metabolic pathways and DNA\index{DNA}. In other words, nature uses an optimum balance of diversity and simplicity in the number of global energy currencies. It does a superb job of provisioning for all life and steadily increasing the capitals that life depends on. 


Energy is valued by nature in days of life. For example, the value of a cup of oil is approximately one week of your life. Think of this next time you fill up your car with petrol or diesel. The true value of each litre that you use to drive away from the filling station is one month of your life, not the money you paid\footnote{Our back-of-an-envelope calculation of the energy content of oil, and the average energy needed per day for a human to live.}.


Nature's interconnectedness\index{interconnectedness} is powered by several different kinds of markets, with the energy market being the most significant. Trades in this market range from highly collaborative, such as aphids providing honeydew to ants and the ants then taking the aphids into their nests at night for protection, to the one-way brutal extreme, such as the lion eating the buck in order possess the energy stored in the buck’s fat and muscle.


The mycelium of a mushroom is very much nature's equivalent to the Visa Corporation\index{Visa Corporation}. They transfer various nutrients from one end of the forest to the other. Each nutrient is a capital, and moves in the form of some kind of currency. No surprise that the largest and oldest living organisms on earth are the mycelia of mushrooms in the old forests.


Of course, these markets look quite different from the typical market we have invented in society. But what our markets look like is a consequence of how our stories create the reality we experience. 


Nature\index{nature} is regenerative for all capitals\index{capitals} except energy\index{energy}. Energy comes into nature's economy in a fixed daily amount, is cascaded down the value chain of nature's economy, and ejected from the planet as low-value waste heat at the end of that value chain. All other capitals grow in value through nature's efforts. Higher value soil is created from lower value rock, higher-value freshwater is purified from saltwater, nitrogen is released from rocks, etc. 


Almost all forms of life on the planet use either DNA\index{DNA} or RNA as their common language to capture their stories. DNA and RNA are very much analogous to your stories that you use to figure out who you are and what you should do for whom.


So if we want to rebuild society's economy in a way that cannot only repair the damage that our economy has caused up to now, but can also enhance the capacity of life on earth to thrive, we can do no better than to mimic nature. This means creating businesses that are as deeply interwoven as all different forms of life are in nature, because they use common DNA and RNA, because they regenerate multiple capitals, and because they use common metabolic pathways. 


The central message here is that economies and marketplaces occur naturally throughout life on Earth, if you look at an economy\index{economy} as any mechanism where valuable capital flows, via a currency\index{currency}, from where it is abundant to where it is scarce.
\subsection{The Western economy is failing}
\label{section:western-economy-failing}


The recent film \emph{I, Daniel Blake}\index{I, Daniel Blake} by Ken Loach\index{Loach, Ken} illustrates quite clearly where our Western economies are failing. Whilst it is set in the UK, it's far from only in the UK that valuable people who have bad luck are failed by our economy.


All Western economies are failing, because they are detached from the stories that create most people’s realities. This is even more visible now than when we first wrote this paragraph, in how badly most countries have responded to the Covid-19\index{Covid-19} pandemic; with the USA and the UK as prime examples. The starting point to reshape the economy\index{economy} as a tool to do the job provisioning is to ask a couple of questions. What needs do you have that your economy is poorly meeting, or not meeting at all? What problems do you have that can't be adequately, inclusively solved?


Depending on your specific needs and what you find meaningful, you may say clean air, security, healthy food, leaving a planet that your great-grandchildren will thrive on. Whatever you say, chances are you will find many areas where your economy is failing you. You know it's failing. The climate emergency\index{climate!emergency}, the 60 years of harvest left in much of the soil around the world, and more, is clear evidence of how your economy is failing you.


Are you invested in property in a city like London, New York, Shanghai, Dubai, Singapore or Manila, either directly or via your pension fund? 


Most of the property value in these cities depends on land that is less than 15m above sea level. The ice on Greenland alone is sufficient to raise the sea level by 6m, even if it just slides onto the sea as icebergs, rather than melting completely. The loss of glacial ice is accelerating too. Since 1980 we have lost a total equivalent to slicing a 22m thick layer off the top of each glacier\cite{glacier-loss}. We won't even begin talking about the ice that is melting and sliding into the oceans on the Antarctic continent.


Property in these cities will be worthless if we fail to act globally now. Which means any investment your pension fund has made into such properties may well be worthless before you retire, as the value in this future begins to be factored into property prices.


The old lenses we've used to look at the economy distort it, just as the Mercator projection distorts land area in such a way that very few people realise how much smaller the USA is than, say, the north-western corner of Africa. These lenses hold us back from seeing clearly the trends in society and the natural ecosystems that human life depends completely on. Because we fail to see actuality clearly, we construct a poor reality, and so too many are still failing miserably to see the imperative to act now. This book shows you how to recognise the lenses you are using, and how to construct additional ones 


\subsection{Introducing the precariat}
\label{section:introducing-the-precariat}\index{precariat|(}


Our failed economy\index{economy} is also visible in a new socio-economic class: the precariat. Precariat is a portmanteau word combining precarious and proletariat, covered in detail in Guy Standing's\index{Standing, Guy} book\cite{standing-precariat}. They hold down jobs, spend as much time as is humanly possible earning money, but live a precarious existence. 


The precariat lacks security in many ways. If you are spending a significant number of hours on unpaid tasks that you have to do to have a few hours paid; and the amount you are paid does not average out to a healthy wage, you are part of the precariat. If you lack a company pension fund, health insurance, unemployment protection, if you lack a voice and decision power in your work, you are part of the precariat.


Many of you reading this book are either already part of the precariat, or will become part of it at some point in your life. Even if you are not, because you have a salaried job for life, the fear that pervades the precarious existence of the precariat is spreading to everybody at work, influencing your decisions.


The emergence of this new class is a clear sign that the Western economy, including our traditional investor-centric ways of incorporating, is failing to do the job of provisioning that \emph{we} need doing. 


There are also benefits that many people enjoy in the freedoms that the \index{economy!gig} brings. The precariat is not an automatic consequence of these freedoms, but rather a consequence of the economy and how we incorporate failing to adapt fast enough. FairShares Commons\index{FairShares Commons} incorporation, which we describe in Chapter~\ref{chapter:ownership}, and unconditional basic income (UBI), in Section~\ref{section:ubi} are solutions.\index{precariat|)}


\subsection{Provisioning all of us}
The core job of an economy\index{economy} is provisioning\index{provisioning}, yet our economy is failing to do this job for many of us. You've just read above how nature does the job of provisioning, and the core role that markets play in nature in transferring valuable resources from stores of value (capital) to where the capital is needed and can be put to work. 


We all need an economy that does the job of provisioning for all of us, even if you find that your economy as you see it is doing the job sufficiently well for you. Your future is insecure because the economy is not doing the job sufficiently well for many, and is creating the climate crisis. 


This means, much as a physicist did in developing a quantum physics\index{physics!quantum} that worked by integrating the opposites of classical physics\index{physics!classical} (like particles and waves) into one, that we need to do the same with capitals, currencies, and markets. We must integrate, transcend, and include in ways that go beyond the stories that run today's economic reality. Just as nature has multiple capitals (not just energy) with multiple currencies for each, and multiple marketplaces, your economy will only work for you and everyone else if your economy is highly profitable on an all-capitals, all-currencies, and all-markets basis.




\section{The stories that define our economy}
\label{section:stories-define-economy}
Your economy\index{economy} starts with stories\index{stories}, which then shape the structures and your reality as you interact with your economy. These stories do that in part because they are the lenses you use to look through. This is an emerging field in economics, called narrative economics, and the topic of a very recent book of that name by Robert Schiller\cite{shiller-narrative-economics}. At last, the lessons from psychology of individuals\index{psychology} and business performance, that the dominant meaning\hyp{}making stories shape the reality that is experienced, are entering economics. \index{economics}


The other important lens\index{lens} is whatever frame of reference\index{frames of reference} you use when you evaluate something as good or bad, right or wrong, beneficial or harmful. In this section I apply to our economy what we cover in detail in Chapters~\ref{chapter:who-am-i-base} to~\ref{chapter:who-am-i-one}, about how to recognise and then transform the stories that shape the meaning you make. Choose for yourself, depending on your preferred learning style, whether to read those chapters on the how and what first, and then come back here for the why; or to read first this chapter on why, and then the how and what.


(I wish we could write the way Picasso\index{Picasso, Pablo} painted: showing everything simultaneously.)


Each story is like the lens\index{lens} you use. If you look through the lens of property\index{property}, you see an economy based on property. If you look through the lens of interactivity, at an economy based on property, you may see very little interactivity. To grasp your economy, and later to change it, you must look through a wide variety of lenses.


Very much like Picasso and his realisation that art had reached the end of the road and could not generate anything new and worthwhile, so long as it only allowed one single perspective on nature.  He realised (at least implicitly)  that every lens biases what you see, making some aspects visible while hiding and distorting other aspects. By using multiple lenses and capturing all these perspectives on canvas, Picasso was using some of the 28 transformational thought forms\index{thought forms (28)} of Chapter~\ref{chapter:who-am-i-sense}. 


Next time you're talking with someone about what is good or bad about the economy (or anything else for that matter), remember to first be clear on which lens(es) you are each using to gather data, which internal frames of reference you are each comparing to, and which meaning\hyp{}making stories you are using. (How to do this is in Chapters~\ref{chapter:who-am-i-meaning} to~\ref{chapter:who-am-i-one}.)


\subsection{Your economy's identity}
Your economy\index{economy} has an identity\index{identity}, just as you have. A central meaning\hyp{}making story\index{meaning-making stories} defining the USA is the story of free individuals, free of government interference. The reality of your experience in the USA emerges from that, along with all the other stories that shape the economy into one where many lack a stable job, and the money to pay for their basic needs of food, housing, health care etc.; but conversely where the shops are filled with all kinds of innovative products. 


Compare that with the meaning\hyp{}making story defining the USSR, creating a reality where every individual had a job, and all their basic needs covered, but where often the shops lacked what people wanted to buy, and business innovation was at a slower pace.


\begin{longstoryblock}
Jack’s story: I had the privilege of teaching in the former Soviet Union. It was a fascinating time for me, yet quite apprehensive for Soviet citizens, especially the young people, knowing that socialism was not working, and that sometime in their future (most assumed later rather than sooner) there would be a reckoning, a transition to an economic system better able to provision, a transition which was expected to be violent. 


Freedom was differently defined in the USSR, emphasizing freedom from economic want, and the freedom to work; but at the expense of constantly waiting in queues, unsure what would be available on any given day, along with shoddy products and a dearth of innovation. I remember one afternoon, traveling on the road with a Soviet colleague, eagerly telling him my innovative teaching and research plans for the University, when he waved his hands laughing, “Jack, this is the Soviet Union.” (A well-known adage during the Soviet era was: the taller blade of grass gets cut.)


I also had the privilege of teaching in the USSR during the tumultuous year of 1991, when the 15 Republics broke away and the USSR no longer existed; and then teaching in the various Republics for many years afterward. The frames of reference that worked in the old USSR no longer worked for the new. That was obvious to all. But those who looked at either the past, the present, or the future with the wrong lens, the wrong frame of reference, were frustrated and saw nothing but failure. But the young people understood the need for new lenses, new frames of references, which gave them hope and the ability to construct a new economic system, a process which is still ongoing. 


The transition from one type of economy to another is all about constructing and then learning how to use the right frames of references. The latter task is arduous and can never be underestimated. This is where listening, dialoguing, patience, and empathy all come into play. 
\end{longstoryblock}


Both these economies led to power in the hands of a few, and they benefitted. Which one is better depends on which frame of reference you compare with to decide on better or worse.


Today we are entering a new meaning\hyp{}making story, and so can build a new reality.\index{reality} One with an economy that has the best of both worlds: high levels of entrepreneurial innovation, high regeneration of all our capitals, and security for all that your basic needs will be met because you have meaningful gainful work. 


A bit like the old joke: “Capitalism is a system where man [sic] exploits man, and socialism is just the opposite.” For too long we have been stuck in flatland, either following a USA story or a USSR story. Like the shift from classical physics to quantum physics, it is high time we left this either / or flatland and entered the quantum world of complementary pairs.


\subsection{Complementary pairs}\index{complementary pairs|(}
One reason why your economy is failing to do the jobs you need done today, so that you can thrive today and build a better future for you and the generations to come, is because neoclassical economics\index{economics!neoclassical} is much more like classical physics rather than modern quantum physics\index{physics!quantum}. Even the analogy to classical physics I find a stretch, because of the paucity of solid data gathered using double-blind studies with reliable control groups.


It's time for economics to become much more like relativistic quantum physics. Recognising that the vacuum, the interactivity between all the parts, the effect of the observer, and the inherently unknowable nebulous nature of much of the economy\index{economy} is what matters. The consequence is giving up all hope of predicting most specifics. All that you can work with is other probabilities. It means understanding what is unknowable, and the two components of complementary pairs are simultaneously both one and distinct.


As we mention in Chapter~\ref{chapter:emergence-einstein-picasso}, physicists used to believe that physical entities were either particles or waves. Light was a wave, an electron was a particle. Waves and particles were either / or opposites. Then they learned that entities transcended such simple either / or divisions. Light was a wave and a particle, depending on which lenses you looked at it through. 


Capitalism\index{capitalism} and socialism\index{socialism} are usually regarded as distinct opposites. But what if they are actually complementary pairs? That if you look at a society through the lens of capitalism, you will see a capitalistic economy and create that as the reality that most people experience. And through that, suppress the socialism that is an inherent part of a functioning economy.


Equally, if you look at society and its economy through the lens of socialism, you will see socialism and build a socialist economic reality that suppresses the essential capitalism needed for an economy to function. 


The story, and the institution that concretises it into reality, form a complementary pair. We cannot build a regenerative economy that works for all without fully understanding these both as one complementary pair and as two distinct elements.


Another complementary pair is the consumer and the investor. Perhaps it's a complementary triangle of consumer, investor, and worker. And yet every worker contributes to the economy at least through their work and their consumption of food and housing. Many also contribute to the economy as investors, through the pension contributions that they pay into a pension fund. That I wrote many, and not all, is another example of how our economy is failing to do the job of provisioning.


\begin{longstoryblock}
I (Graham) was driving down the street in Tzaneen, South Africa\index{South Africa!Tzaneen}, back to Granny Dot’s\index{South Africa!Granny Dot’s} and was pulled over by a traffic policeman at one of the regular roadblocks to check drivers’ licences and roadworthiness. He introduced himself as Mike, we asked each other how we were, and then he asked for my licence. I showed him my German driver's licence, and he looked at me with a big smile and said \begin{quote}you're a German brother.\end{quote} I valued how, despite decades of decidedly unbrotherly relationships between the peoples of South Africa, I was greeted with brother; a recognition that we are all one, regardless of the superficial differences.


We started chatting, as one does sometimes in South Africa, and he asked me where I came from, what I was doing. 


\begin{quote}What do you think, brother, about business possibilities between Germany and South Africa? Take someone like me, working for the state. I know I'll never get rich on a state salary, but sometimes I don't even know if I will get enough money to pay for all my bills this month, let alone save up enough to send my children to university and have a comfortable retirement.\end{quote}


I didn't have a good answer then, and I still don't. All I could think was that this was another example of how our economy is failing to do the job that society needs it to do, provisioning for all. In particular, I thought about how it's failing to do the job of provisioning for those of you in many of the essential roles that develop each of you as individuals and all of us as a stable, functioning society.
\end{longstoryblock}\index{complementary pairs|)}


\subsection{Maximising shareholder value}
We describe more of the meaning\hyp{}making stories\index{meaning-making stories} that are creating the economic reality you are experiencing today in section~\ref{section:myths-of-incorporation}. But the most powerful story creating today's economic reality is the story that a company exists to maximise shareholder value\cite{veldman-modern-economics}.


This story is one of the biggest myths\index{myth} that's taken root in the recent past. Compare Barclays bank, or Cadbury's chocolate, today and when they were founded by Quakers based on Christian values and the balanced importance of society and the individual. How visible in their results today are their founding values?


Originally, the stories of individual shareholder benefit only and the story of a company as part of your community, to be of service to all of you, were both around. Many companies were relatively small, and deeply embedded over decades in all aspects of their community. In the 1950s, companies were seen as legal persons with a role to benefit society overall.


It was during the 1960s that the stories of conservative neoclassical economists\index{economics!neoclassical} and others began to dominate. The story of shareholder primacy, that the company existed to serve the interests of investors, that investors of financial capital were the only people taking a risk that was not adequately reflected by a market remuneration; and that investors had the right to exercise control over the company, began to create the reality that we experience today.


This story failed to see the company as it is legally defined, a non-human legal person, and instead saw it as simply a collection of assets and contracts between the owners of those assets. By believing that, a meaning\hyp{}making story of property ownership going all the way back to Locke\index{Locke, John}, investors and many others created the economic reality we have today. Step-by-step as the story was concretising into institutions\cite{veldman-modern-economics, stout-shareholder}. Many today have lost sight of the original story as just a story, and can only see the institutions as if they were unchangeable.


\subsection{Biases in economics}
It is impossible to talk usefully about economics\index{economics} without including the biases\index{biases} and meaning\hyp{}making stories of the economists\index{economists} that have shaped the neoclassical and pluralist economics\index{economics!pluralist} used to talk about our economy. 


There are two cognitive biases from section~\ref{section:biases} that I believe play a significant role in shaping the economy you experience today: uncertainty avoidance bias, and authority bias.


Uncertainty avoidance bias leads almost everyone, and politicians and economists are no exception, to prefer an explanation that has no uncertainty, regardless of whether it has any validity at all. However, because your economy\index{economy} is first and foremost a social tool that everyone has collectively constructed, and is constantly under construction (Chapter~\ref{chapter:who-am-i-sense}, thought form P5), it always will be filled with uncertainty. Any valid economic theory describing what we can say about an economy can only be a theory of probabilities and uncertainties, just as quantum physics\index{physics!quantum} is.


Exacerbating this is authority bias. Those few who declare themselves to be authorities, and especially those who derive their self-identity from their authority and expertise, are themselves subject to authority bias and trigger authority bias in everyone else. In any situation where you are listening and believing what somebody says because of their position, rather than because you know and have checked the validity of what they are saying, authority bias is in action.


As you've learned from the beginning, the stories\index{stories} that everyone uses to make meaning, are stories that at least shape, if not completely create, the reality that you experience. These biases are nothing other than a set of deeply entrenched stories that almost all of you share. The reality of the economy that you experience is not the only possible economic reality. The reality of capitalism\index{capitalism} that you experience is not the only possible capitalist reality.
\subsection{Stock markets and the myth of consumer sovereignty} 
\index{consumer sovereignty} One central meaning\hyp{}making story\index{meaning-making stories} in economics is that the company is an ownable good. And therefore can simply be bought and sold on the stock market\index{stock market}, just as if you are buying and selling any commodity. However, this is in stark contrast with the other central meaning\hyp{}making story in company law, which is that a company is a non-human legal person. 


There is a second story, which is actually an enabling myth\index{myth!enabling}, that of consumer sovereignty. This myth\index{myth} claims that consumers are in charge of companies because they can withhold purchase. This myth prevents us from seeing clearly the real meaning\hyp{}making story of investor primacy. If all companies behave similarly, shifting our purchase from one to another makes no difference. 


You can read much more about the consequences of integrating these stories\index{stories} to write a new, powerful meaning\hyp{}making story fit for a regenerative economy in Chapter~\ref{chapter:ownership}. 
\section{Your current and future challenges}
We have many current and future challenges that are going to transform society. A number carry significant risks of social upheaval and pain for many if our economy fails to adapt fast enough to do the job of provisioning for all. And it can only do this job of provisioning if we recognise and work with the rapid changes in the dominant stories that shape the reality of our society and your needs. These include:


\begin{itemize}
\item The imperative driven by the climate emergency\index{climate!emergency} to rapidly shift all our energy supply out of oil and coal, where USD 20 trillion of the economy is tied up\cite{birch-business-and-society}. 


\item The dramatic narrowing of specialisation driven by the increase in total knowledge, meaning that in the near future the average knowledge worker will know less than 1\% of what they need to do their job, compared to 75\% in 1985.


\item The emergence of renewable energy\index{energy!renewable} and artificial intelligence\index{artificial intelligence} is shifting energy, transport, manufacturing, and knowledge rapidly towards zero marginal cost\cite{rifkin-zero-mc}. 


\item The growing scarcity of freshwater around the world, both for you to drink and to irrigate the fields that grow your food. (70\% of our usage of freshwater is for your food.) Especially the rapid disappearance of the glaciers providing freshwater for the Himalayan regions\footnote{\url{https://ourworldindata.org/water-use-stress}}.


\item The Gulfstream shutting down if the Arctic\index{Arctic} ice cap shrinks much more. 


\item The potential in the internet of things, big data, and global interactivity to open up completely new adaptive capacities we have never had before. Will these enable us to rise to the challenge?


\item The rise of artificial intelligence\index{artificial intelligence} is bringing with it a social transformation significantly bigger than any other shifts, from the Industrial Revolution onwards. Today most of the economy depends on knowledge workers. In the very near future many of the skilled jobs that we have today will be automated. Over the past few months, my (Graham) administrative assistant has been Amy of x.ai. All I do is send an email asking her (an artificial intelligence system) to arrange a call with five people, then accept the invitation that she has arranged at a time that suits everybody.


\item In a decade or two, many things that people are currently being paid for will be done by software or physical robots. At that point, our economy\index{economy} will need us all to do what cannot be automated, which is entrepreneurs creating businesses, human beings caring for human beings, and creative innovation. All this (and note that, importantly, economics is not prepared to measure this activity or even to prioritise such goals) will be especially important over the next decades as we need these to deal with the consequences of climate change and all the other problems that we've created on the planet through our economy. 


As these jobs disappear, and we require human beings to put more time and effort into less predictable work; where you cannot predict at the point you start, whether what you will have produced over the next five years will be the one in 1,000 innovation that is the next Facebook\index{Facebook}, or the 999 others that needed to be done to find out that they were not the next Facebook. To ensure that all thousand can be maximally creative so that we get the one Facebook, all thousand people at least need to have their basic income taken care of so that there is no disincentive to trying.


\item Greater need for rapid changes in jobs, careers, and for entrepreneurs. Society needs the economy to remove obstacles to your taking risks, being entrepreneurial and innovative. And the biggest obstacles are the ones of fear around survival, in the very first days of innovation on an entrepreneur’s journey, where you have an initial idea of what might work but no evidence yet that it will. 


Innovators and entrepreneurs have always had some way of having their survival needs taken care of during those early days. Once, it was a wealthy patron that took care of the artists and scientists that our society and economy is today based on. In the spring of 1696, Newton\index{Newton, Isaac} was taken care of by being made Warden of the Royal Mint. Leonardo da Vinci’s patrons included the Medici, Ludovico Sforza, Cesare Borgia, and King Francis I of France.
\end{itemize}


Humanity has now reached a point where the individual can no longer be seen as the primary element of reality. And so we can no longer say that it's enough to have all entrepreneurial innovation coming from a few people supported by a few patrons.


We now need a fundamental shift of our entire society and economy to one where everybody is as entrepreneurial and innovative as possible, which means that the entrepreneurial and innovation characteristic shifts from just a few to becoming the reality of all. That also means the role of the patron shifts from a few individuals into the whole of society, i.e, the UBI.


Knowledge has become less and less a competitive edge. The competitive edge emerging today is interactivity, creativity, and what you do with that knowledge. You could call this the marriage of individual wisdom with social oneness.


Speak to anyone who understands and works with individuals and communities, and they will tell you that nothing kills off innovation, wisdom, and collaboration faster than fear. We urgently need our economy to be one where wisdom and interactivity, where individuals standing fully in their uniqueness and together in our common oneness, is the primary engine for wealth creation.


Two elements of that that we believe are essential are the unconditional basic income (UBI) \index{unconditional basic income}and FairShares Commons\index{FairShares Commons} businesses.


This will eliminate most of the fear that people have around artificial intelligence and augmented reality. The wealth generated by all these businesses will be fairly shared amongst all stakeholders, which includes those whose businesses and professions disappear as a consequence of the artificial intelligence\index{artificial intelligence} revolution. Because this is done directly within the economy, rather than through government handouts, with everyone participating in governance decisions, everyone will be empowered, not disempowered.


Our needs for mastery, autonomy, and making a difference, will be the primary drivers in this economy for people finding ways of contributing value to themselves, to their peers, and through that, to the economy as a whole. (More in Chapter~\ref{chapter:who-am-i-nature}.)


\section{The economy cannot be controlled}
The economy\index{economy} can’t be controlled, nor can it even be fully understood by anyone, not even those working on it, the business leaders, economists, and politicians. No politician, bureaucrat, or economist has any hope of ruling the economy.


You have probably heard or read many times over that you cannot solve a problem with the same thinking that created it (Einstein)\index{Einstein, Albert}. In the language of this book, you cannot solve an adaptive challenge (Section~\ref{section:adaptive-technical}) with the same Size of Person \index{Size of Person}(Chapter~\ref{chapter:who-am-i-one}) that created it.


The adaptive challenges\index{challenge!adaptive} we are facing are bigger than any challenges any human being has ever faced before, and require a new economy to address them. These challenges have a physical size as big as the planet, time horizons of centuries, and are inherently nebulous, volatile, uncertain, complex, and ambiguous. 


If you are, in any way, called to step up and do something, and whatever your role, the bigger you are, the better you will rise to the challenge. Whilst especially true for economists, leaders of all sorts, and politicians, this section applies equally to all of you reading this book.


The better you know the stories\index{stories} that you use to shape your reality, and the better you are at then rewriting those stories, the more likely you are to shape a future reality that addresses the challenge.


The greater your fluidity in the 28 transformational thought forms\index{thought forms (28)} of Chap\-ter~\ref{chapter:who-am-i-sense} the more likely you are to be able to work with the fully nebulous, complex, or even chaotic inherent nature of the challenge.


The more subtlety you can use in rising above your unique nature and set of biases\index{biases}, the more likely you are to address the challenge according to its nature, rather than your own nature.


In fact, I will go so far as to say that we would not have crossed the edge into our current climate crisis, nor any of the other existential crises humanity faces over the coming century, if the people who had made political, economic, and business decisions over the past century had been bigger than the decisions they made.


You need to be big to rise to today's challenges, precisely because an economy is inherently nebulous, and cannot be controlled, nor even predicted in specifics. 


Our economy, and perhaps any economy, is inherently on the edge between complex and chaos in the Cynefin framework\cite{snowden-cynefin, cynefin}, and quite often deeper in chaos because of misguided attempts at command and control. 


The Cynefin framework\index{Cynefin} is essential for anyone working on the macroeconomics of our society; without familiarity with it, you cannot deliver useful results in macroeconomics today as an economist, politician, or business leader. It summarises the different nature of, and ways of dealing with, the five different conditions.


\begin{description}
\item[Simple] Under these conditions you can get far more data than you need to understand fully what is happening. The data changes slowly or not at all, and there are no hidden, unknown interconnections. Best practice is clearly defined and validated. So you gather data, categorise, and apply best practice. 


\item[Complicated] Here you can just get the data you need in time, you can draw on experience, but a little too much is new or unknown to apply best practice. So you gather data, analyse, and apply good practice.


\item[Complex] Here there are too many variables for the data you can gather, the situation is changing too fast, and there are significant hidden inter-connections. By the time you have gathered data, the conditions have already changed so far as to render the data inadequate. Now you need to take a small action, observe what happens, and then react.


\item[Chaos]\index{chaos} Here there is just way too much happening, way too fast. So you need to immediately act on a large scale, see what happens, and react fast and big. However, there are still underlying patterns allowing some courses of action. Our climate emergency and all else in the 17 SDGs\index{Sustainable Development Goals, UN 17} are driving us deep into this quadrant.


\item[Disorder]\index{disorder} This is complete disorder; there is little you can do, other than hope\index{hope}.
        \label{list:cynefin}
\end{description} 


As we’ve said, an economy\index{economy} is filled with irreducible ambiguity, complexity, and is inherently nebulous. It is partly created by the meaning\hyp{}making of people with power over others. Since many of these factors are hidden, or change as we look at them, with deeply hidden inter-connectedness, our economy is at best Cynefin complex, and often Cynefin chaos.


Of course, our uncertainty avoiding bias leads us to want to believe that our economy\index{economy} is Cynefin simple or complicated, predictable and controllable. Our authority bias then leads us to follow the advice of anyone wearing the badge of authority. Not questioning whether we are following them just because of the badge, or because what they are saying is actually valid. 


\section{Institutions}
\label{section:institutions}\index{institutions|(}
For society to function well though, we need institutions. (Institution here refers to established norms applicable to large groups of people, not organisations like the United Nations.) These give us protocols, and include biases. As you will read in Section~\ref{section:biases} your biases can serve both you and society very well, and so become concretised into worldviews and institutions that make life very simple for us all, until they pass their sell-by date. They make our life efficient and simple because mostly we then only need to use linear logic and Type~1, or fast thinking (Chapter~\ref{chapter:who-am-i-sense} and Kahneman\cite{kahneman-thinking}), to act.


Think of how much effort it would cost you if, every time you met a stranger, you needed to engage in an intense negotiation to work out how to greet them. In some languages and cultures, such as the Japanese, there are a large number of words to simply say hello to someone. Which word to use to avoid offending depends on which pre-existing relationship you and they slot into. Which is why, in such cultures, the institution of exchanging business cards first plays such an important role in even greeting a stranger.


Other institutions that have served us all well across thousands of years are institutions like marriage, trade, democracy, and many more. A highly functioning economy, that does the job of provisioning you need, depends on highly functioning institutions that are right for the time.


So to construct such an economy, like the Economy of the Free\index{Economy of the Free} I describe in Chapter~\ref{chapter:economy-of-the-free}, the first step is to understand how to recognise the hidden meaning\hyp{}making stories\index{meaning-making stories} that have become concretised into our institutions, and the reality that these concretised stories are now creating. Both the individual parts and how they combine into a whole that is qualitatively and quantitatively different from the mere sum of the parts. You need to be fluid in applying the seven \emph{process} thought forms of Chapter~\ref{chapter:who-am-i-sense} to grasp how these institutions themselves change over time and thus build a reality that changes over time. 


You then need to be fluid in the seven \emph{relationship} thought forms to understand how pre-existing relationships may have shaped the individual parts and yourself as a part of your economy.


Finally, to fundamentally transform your economy\index{economy} into one that does the job you need it to do, you need fluidity in the seven \emph{transformational} thought forms.
\subsection{Institution: freedom}\index{freedom}
The institution of freedom is crucial to our modern society, and hence to our economy. It is a huge, multifaceted concept. Each of you reading this likely has a different meaning\hyp{}making story attributing meaning to freedom, so each of you means something different with freedom.


We use freedom in its simplest sense: free to move within any relevant space. 


So freedom of physical movement means being able to travel from A to B. It might mean walking to the shops, it might mean travelling by train to another country, or travelling to the moon. 


It can also mean freedom to move within socio-economic space. The freedom to move from your birth in poverty to becoming CEO of a multinational, the freedom to move from one meaning\hyp{}making story to another, the freedom to move from one religion to another. 


Freedom also includes the freedom to move through your life along a path that is best suited to you, your nature, and your development potential. The freedom to grow into who you can become.


Boundaries place limits on the extent of freedom and are essential for freedom to function well. For example, if you want to walk from A to B in London, you are no longer free to move in all possible ways. The road and tube system places boundaries on your freedom, you can't simply walk in a straight line from A to all possible Bs; and it enables your freedom too. A complementary pair again. 


However, these boundaries\index{boundaries} maximise the freedom that we all have to get from all possible As to all possible Bs with all variations of speed and distance, and to have buildings to live and work in, plus everything else we want our cities to provide us with. Freedom to move in human society is a negotiation across all freedoms of all stakeholders at all scales across all spaces to move within.


This ought to include, across time, to include stakeholders yet to be born. Burning fossil fuel today in your car, to have more freedom of movement today, takes the freedom to thrive in a comfortable climate from future generations. 




\subsection{Institution: stakeholders}
\label{section-stakeholder-institution}\index{stakeholders|(}


Ubuntu\index{Ubuntu} fully recognises that each of us is who we are because of who we all are, and hence the necessity of including all stakeholders (for how this has created you, and how you can transform yourself, see Section~\ref{section:ubuntu}). 


If you are not actively and visibly including certain stakeholders, you lose most of the power\index{power} you would otherwise have to consciously use the adaptive capacity\index{adaptive capacity} that they bring, because you drive underground the role that they play in shaping the reality experienced.


We define the institution of stakeholder as an entity with an interest in the existence of another entity. These stakeholders can be human legal persons, non-human legal persons, alive today or yet to be born, or even already passed away; communities, interest groups, the natural environment; in fact, anything with a stake.


For example, imagine that you have founded a new kind of restaurant, following your intuition into the criteria gourmets and the Michelin guide will have in 10 years. You are a founder; you have bootstrapped the company’s initial funding needs, so you also belong to the investor category. You will be doing a lot of the cooking in the kitchen so you belong to the employee category, and soon you will be managing the work of the staff you hire, so you belong to the executive category. You will quite likely eat in the restaurant, so you also belong to the customer category. You may find that you need to grow certain otherwise unobtainable ingredients yourself, making yourself a supplier.


When the organisation\index{organisation} is still very small, you and a few family members and friends are co-investors and co-workers, much like an embryo with a few cells. There is no need for differentiation. Each of you shows up as your full selves, fully caring for yourselves, each other as human beings, and your business as a whole. Each shows up as a whole person integrating all the stakeholder categories they belong to and united by the common oneness of being part of one restaurant.


Stakeholders do not have hard boundaries\index{boundaries}, but flow into each other. It is time for the same kind of shift in how we talk about society, especially what the subsets of economics and business say, that came with quantum physics. 


As we mentioned in previous chapters, in classical physics\index{physics!classical}, a proton is clearly a solid particle, as is an electron and a neutron, and each only belongs to its own category of particles. Light is clearly a wave, only a wave, and has nothing in common with protons, electrons, neutrons. Then a hundred years ago quantum physics \index{physics!quantum} broke through and for a couple of decades war raged between the new quantum physicists and the old classical physicists. 


Eventually this war settled down when it became clear that the classical mechanics could continue doing their mechanics with perfect accuracy in certain situations, but in other situations you needed the quantum mechanic to do her magic.


Central to this was the complete disappearance of the hard distinction in categories. Once physicists\index{physicists} had become comfortable with the unarguable fact that light was inherently and irreducibly both wave and particle (called a photon), and that the known particles (e.g., the electron, proton, and neutron) were also inherently and irreducibly both particle and wave, physicists had to become comfortable with two inescapable consequences.


First of all, it became clear that as light moved from one place to another, whatever the photon was, it could not just be light. It was in constant flux, turning into countless pairs of particles and antiparticles, such as electrons and positrons. These had such a fleeting existence over such small dimensions that they were hidden inside Heisenberg's\index{Heisenberg, Walter} uncertainty\index{uncertainty!principle}. Equally, as an electron moved along, it was surrounded by a cloud of photons, and other electron-positron pairs.


In short, the hard distinction is not something that nature\index{nature} itself recognises. In nature the distinction is soft, fluid; everything is in constant flux shaping and even morphing into everything else. The hard distinction is something that human beings have imposed so that we can talk about nature with our limited language and concepts, to then build our houses and bridges in ways that they do what we need them to do.


The second transformational consequence of this, in how physicists realised they needed to describe nature, was the realisation that the vacuum was a fully active part of every single particle. The vacuum\index{vacuum} around an electron is a seething mass of photons that the electron pulls out of the vacuum,\index{vacuum} and the electron-positron pairs that each photon then transforms into and is recreated out of. 


This led to a couple of frantic decades for physicists, because figuring out what an electron would do now requires you to include everything from here to infinity. Which naturally gave ridiculous answers, somewhat akin to each electron being as infinitely heavy as the entire universe might be. Eventually they figured out that something called renormalisation was the missing ingredient to give sensible answers out of this nebulous mix of everything.\index{stakeholders|)}




\subsection{Institution: property}
\index{property|(}
The institution of property, or the story of what it means to own something, will continue to be an important building block of your economy\index{economy} and the future. But, I will provide what I hope is compelling evidence in which situations it is downright harmful in section~\ref{section:slavery} and in the subsection below.


John Locke\index{Locke, John} wrote a description of one possible story of an economy, where if you performed work on public land you had the right to claim ownership of that land. This was the basis for capturing the USA from the trustees of the original “Land Trusts”. Ownership of land in the USA had, before the arrival of European settlers, been regarded as held in trust for future generations and the lower orders of nature itself. Humans were here as trustees or stewards accountable for protecting the land. Typically, they regarded taking into account the needs of the next seven generations to be adequate. 


Contrast this with Locke\cite{locke-two-treatises} describing, in a time where agriculture was the primary source of wealth generation for many, that you could only own land if you were deemed by yourself or your peers to be highly able to generate wealth. Otherwise you are excluded from freely using the land, but instead need to pay the landowners rent. And to make it worse, your right to challenge these decisions by engaging in governance was only possible if you were a landowner.


Two different lenses\index{lens} or stories\index{stories} creating incompatible realities, coupled with a power mismatch and human nature, gave us today's reality in the USA. If the original “trustees” of land in the USA had had more power to enforce their stories, i.e., their laws, what do you think might have happened? Climate emergency \index{climate!emergency} or no climate emergency? Instead, Locke's ownership story became the most influential globally.


This story of property, that you as an individual human legal person or as an individual non-human legal person could own, was, and still is, inconceivable in many old cultures and religions.


The reality you are experiencing today is a consequence of the stories of ownership coupled with the power to enforce them. Both direct physical force and indirect force through laws; and their associated systems of reward and punishment\cite{vant-institutions}. 


There are many simply different stories of ownership\cite{vant-institutions} spread out across the spectrum. No one owns it, and you are there both as a steward and a beneficiary of a commons that no one owns; it could be owned by everyone; it could be owned by a trust; it could be owned by the state; or it could be privately owned.


So everything around private property begins as a story giving meaning to something as mine, and not yours. It's a story of separation and exclusion. Over time, the stories get concretised into law. Remember, you get a lot of value by concretising stories into institutions. Everyone can run on highly efficient type~1 thinking, you have predictability, you can take much bigger risks because the risks are only within small domains. We could not function in groups bigger than about two people without these. And the meaning\hyp{}making stories around property are amongst the cornerstones of any functioning society. 


But they still are nothing other than commonly accepted meaning\hyp{}making stories\index{meaning-making stories}. No matter how well the story fitted the needs of your parents and grandparents, if the story does not fit your needs today, there is no reason to stay with the story. I see no way of rising to the adaptive challenge\index{challenge!adaptive} of climate change without surfacing the meaning\hyp{}making stories of property\index{property}, and evaluating them very carefully for effectiveness in adapting to the challenge of our climate emergency \index{climate!emergency}.






\subsection{Institution: Money vs. currency}
\label{section:money-vs-currency}


One meaning\hyp{}making story creating the reality you are experiencing today is that money\index{money} is a neutral background. Neoclassical economics\index{economics!neoclassical} sees money the way classical physics treated the vacuum\index{vacuum}, as an inert background. A growing body of research into money\cite{leander-phd,lietaer-money-and-sustainability,lietaer-rethinking}, versus all the other types of currencies\index{currency}, suggest that economics\index{economics} as a whole is on the edge of the same revolution quantum mechanics\index{mechanics!quantum} brought, in seeing that the vacuum was dynamically shaping particle behaviour. 


So too are we beginning to see how money is far from being neutral in our economic calculations. In the near future, the indications are that economics will shift into the next phase of economic theory, one where the nature of money (regardless of the amount) is recognised as the primary shaper of business decisions and their consequences. Just as in quantum physics today, where the vacuum dresses the electrons and is responsible for the bulk of an electron’s mass.


Money\index{money} is created through debt\index{debt}, at a positive interest rate, so it's technically called positive interest bank debt\index{positive interest bank debt}. Most of our monies are positive interest bank debt, whether euro, US dollar, sterling, yen etc. It is a concretisation of the underlying meaning\hyp{}making story behind money. One reason why the economy becomes unstable if the interest rate is negative is an artifact of using only one kind of currency, positive interest bank debt a.k.a. money. Even a zero interest rate is unstable, because that zero is the average. So for small periods of time, or in small geographical regions, the effective interest rate is negative, while nearby or a little later it is positive and the entire vacuum is unstable.


A myth that most people believe is that money comes from the state. In most countries, over 95\% of the money supply is created directly by the banks. The work of Ann Pettifor\cite{pettifor-production-of-money}\index{Pettifor, Ann} and Stephanie Kelton\cite{kelton-deficit} \index{Kelton, Stephanie} shows how we could create an economy that does the job of provisioning for us all far better, if we took the current story of meaning making that has generated the reality of money supplied today and adopted another story that may well be far better at creating the reality we need to rise to the climate emergency\index{climate!emergency}.


For example, we already know that you might want to decide whether to plant pine trees or oak trees. If you do a net present value calculation using a range of different currencies; for example, any money, or a currency based on different fundamentals such as the Terra\index{complementary currencies!Terra} or Swiss WIR\index{complementary currencies!WIR} complementary currency, running since 1934 (code: CHW), your decision may be different in each case, even though you use the same economic formula.


To build an economy\index{economy} that does the job of provisioning\index{provisioning} for all, I believe that making the same leap physicists\index{physicists} made a century ago to seeing the choice of currency as an active dynamic vacuum, and not a neutral inert one, is essential. 


Each currency\index{currency} is simply the reality shaped by the different stories behind each currency. The story of money is positive interest bank debt, the story of the WIR is zero interest peer-to-peer business transactions, and the story of the Terra is representing in a human economy key elements of nature's economy. The conclusion, from a purely financial net present value calculation, is pine trees if you use money as the medium for the calculation, and oak trees if you use the Terra.


There are three excellent books to read for far more detail on how to recognise which currencies go with which capitals\index{capitals}, and the minimally diverse set of currencies needed to build an economy that does the job of sustainably provisioning for all over multiple generations: two by Bernard Lietaer et al.\cite{lietaer-money-and-sustainability,lietaer-rethinking}\index{Lietaer, Bernard} and the doctoral dissertation of Leander Bindewald\cite{leander-phd}\index{Bindewald, Leander}


I (Jack and Graham) see sufficiently strong indications of just how powerful each currency is in shaping the reality that emerges to believe that, if we had had an economy based on an appropriate number of complementary currencies\index{complementary currencies} (including money\index{money}), very few of the business choices leading to the climate emergency of today would have been taken. 


This claim has such far-reaching consequences of such power that it warrants repeating. If our economy was based on an appropriate basket of currencies, not just the single currency of money in different national flavours, and our economic theory recognised and could calculate exactly how each currency shaped the economic reality that you experience, we would never have gone down the decision tree that has created the reality of our climate emergency.


In 10 or 20 years time, you may well be looking at money the way physicists\index{physicists} today look at the vacuum\index{vacuum}. As something so far from being a neutral, inert background, that it's hard to believe physicists could ever have imagined the vacuum was not playing an active role in shaping reality.


Jack and I sincerely hope that this will get sufficient attention so that the experiments capable of falsifying this hypothesis actually get run in the very near future. They'll probably need to be phenomenological simulations, rather than real experiments, given the challenge of finding enough other earthlike planets in a climate emergency to run even a minimally robust double-blind experiment!


Are you an entrepreneur, or leading a mature business? How might your business plan change completely if you use the same assumptions and a fundamentally different currency in your spreadsheet? How is the meaning\hyp{}making story that has created the reality of money, i.e., positive interest bank debt, shaping your business reality? What does this mean about how you could better incorporate your business? Read on about the FairShares Commons incorporation, and ecosystem FairShares Commons\index{FairShares Commons} companies, in Chapter~\ref{chapter:ownership}.




\subsection{Institution: capitalism}
\index{capitalism|(}
Adam Smith\footnote{Adam Smith (1723-1790) was a philosopher and considered by many to be the founder of economics. But Abdul Islahi makes a convincing case that the founders of economics were actually Islamic economists writing during a very fruitful period between the 8th and 11th centuries\cite{islahi-history-islamic}. In his important book, \emph{History of Islamic Economic Thought: Contributions of Muslim Scholars to Economic Thought and Analysis}, Islahi convincingly documents that many of Smith’s ideas, and just about all the concepts found in today’s economics textbooks, were first developed by Islamic scholars. } wrote a story\cite{smith-wealth} describing a world that had never existed, but that he believed would be a significant step forward for humanity. A world where men and women are free to take care of their own interests and the greater good, with security because they are free of debt, injustice, (no precariat!); and where everything is coordinated and balanced by God for the greater good\cite{graeber-debt}. 


This story has played a very powerful role in creating the reality you experience, and especially some of the institutions that the story has concretised into.


We’ve lost key elements of his story in today’s capitalism\index{capitalism}, and other elements have been reshaped by our other stories to a point where Adam Smith would not recognise his story. For example, as you will read in Section~\ref{section:money-vs-currency}, the very money that we use to measure almost everything of value, and to facilitate the flow of any value from where it's abundant to where it is scarce, is itself a story of debt\index{debt}. Almost all the world’s monies are different flavours of positive interest bank debt. Today's capitalism concretises the very opposite of his world without debt.


If you read both of his books, \emph{The Theory of Moral Sentiments}\cite{smith-moral} and the \emph{The Wealth of Nations}\cite{smith-wealth} that he wrote 25 years later, he would far more likely recognise his capitalism in the regenerative capitalism I describe in section~\ref{section:regenerative-capitalism}.


The dominant variant of capitalism today, and what most people think is all that capitalism can be, can be simply defined as: the means of production are ownable property, owned and governed by persons, both human and non-human legal persons, and either directly or indirectly via businesses, investment funds, and governments. Then problems come adding all the other meaning\hyp{}making stories around this, leading to a ridiculously small percentage of people having controlling governance power. 


Paul Mason's\index{Mason, Paul} book\cite{mason-post-capitalism} \emph{Post-Capitalism} contains a lot of excellent content, and is well worth reading. 


Much writing and talking is about what we do not want, without naming what we do want, now how to build it; and suggests that there is little of value in capitalism that we ought to retain. I believe that, just as quantum mechanics included what worked in classical mechanics, and transcended it by defining clearly the limits, and what worked beyond those limits, we can quickly develop the regenerative economy we need by recognising and retaining what is of value in capitalism, as well as defining clearly what is fundamentally false, and where the boundaries are. 


Regenerative capitalism\index{capitalism!regenerative} includes all capitals\index{capitals}, all currencies\index{currency}, up to and including the capitals and currencies of our planet's ecosystem. (Section~\ref{section:regenerative-capitalism}.)


I found, writing this section, that there was no definition of capitalism that everyone would agree with. Looking at the writings of different people over time, each of their capitalisms changed when they looked at it. Capitalism 30 years ago was different to today, and it differs from country to country. Typical of thought form P5\index{thought forms (28)}, and any inherently emergent story. 


So here is our attempt to lay out enough building blocks to span all the different stories\index{stories} of capitalism, and their concretisation, as you experience them today in your reality. From these you will recognise and categorise what you are likely to hear talking to neoclassical economists and reading their books. In the rest of this subsection, I will do my best to point out the story versus the concretisation, and the opening for economics to make the same jump that physics made from classical to quantum mechanics.


You can think of capitalism, as we experience it today, as an entire bookshelf of meaning\hyp{}making stories\index{meaning-making stories} fighting for supremacy. It is often very hard to recognise clearly what is a factual and actual versus a concretised story. This multiplicity of stories battling for supremacy is why defining capitalism is impossible. Whatever it was yesterday, it is something different today, and yet again different tomorrow. Opening the door into the next stage of capitalism requires each of us to grow our capacity in the 28 thought forms so that we can recognise the difference between the specific concretisation, and the valuable essence, of each story.


Capitalism at work is extraordinarily dynamic and innovative. Capitalism is almost as dynamic and innovative as nature is. This is definitely something to retain. To rise to the adaptive challenges of our climate crisis and development goals, we need all the innovation and dynamism we can muster. But, of course, in a form that fits within nature's boundaries.


As you read above, the heart of capitalism is property. The meaning\hyp{}making story that created the capitalism we experience says that the best way to maximise the value of something is for it to be owned by someone whose self-interest lies in maximising that value. This story of property says that if only you own something can the best decision be taken.


In Section~\ref{section:slavery} you will read compelling evidence that this is patently false in at least some situations, times, and places. In fact, you have already experienced in your reality at least one situation (you as property; I hope because you have never been another’s property, but maybe because you have) where I expect it is perfectly clear to you that property law\index{property!law} is the best way of destroying value\index{value}. The doorway into a regenerative economy that will work lies in different meaning\hyp{}making stories\index{meaning-making stories} around property and freedom. 


I'm not in any way saying here that we should eliminate property law in all cases, everywhere; what I am saying is that we need to look carefully for just where property law gives us the outcomes we need to deal with the adaptive challenges we face, when this freedom gives us what we need. And maybe we will recognise that we also need to invent something that integrates the two, in the same way that quantum physics\index{physics!quantum} integrated the concepts of particle and wave into one.


The current concretisation of that dynamism and innovation is, however, harmfully tied into property\index{property} as the underpinning story of capitalism. This means that most of the innovation and dynamism are treated as property owned by a few, who charge rent to the many to access that. There are elements of property that we need to retain, elements of property to lose, and a new integration that goes beyond owning and non-owning. These are all stories, and there's nothing to stop us from integrating their essence into a completely new story of regenerative capitalism\index{capitalism!regenerative}.


Zero marginal cost\index{marginal cost, zero} is one of the goals of capitalism, whether explicit or implicitly stated. As soon as we have zero marginal cost, the distinction between owning and not owning, private property and a commons, and capitalism and socialism\index{socialism}, disappear. These complementary pairs\index{complementary pairs} become integrated, just as you read above about quantum physics\index{physics!quantum}.


Maximising profit, and a drive towards zero marginal cost, is another deemed building block of capitalism today. This has some validity, and much that we should keep,  because it is a door into a future regenerative economy that does the job of provisioning for all. Maximising profit just needs to be broadened to take into account all capitals. Striving for zero marginal cost of capital growth for all capitals will lead to regeneration of all of them and a good, viable standard of living for us all, and in fact everything else that is alive.


The assumption that all individuals are naturally acquisitive, into bargaining, and oriented around their own self-interest is central to the modern concept of capitalism. But as you will see in Table~\ref{table:needs-top} on Page~\pageref{table:needs-top}  our top human needs are pretty much balanced across doing things for others and doing things for ourselves. 


The door into creating a regenerative economy lies in recognising that all of these needs are part of what energises and motivates human beings to get stuck in, be productive, and create wealth. In a regenerative economy, each need is part of the description that economics gives and works with in an integrated way, recognising that each individual has a unique weighting for each need. In a regenerative economy, the individual who is strongly acquisitive and self-interested is as valuable as another altruistic and other-interested individual.


Capitalism is also built around power\index{power} over, almost exclusively via the power of lots of money over little money. There is some balancing through politics, and a growing power from mass action to balance, but the power of big-money is still overwhelming. The power of big-money is also beginning to subvert mass action, subtly harnessing it in the interests of big-money.


There is nothing inherently wrong with power\textemdash as any physicist will tell you, power is just the ability to move something heavy. If we are going to move all the heavy things in order to tackle our climate crisis and the 17 SDGs\index{Sustainable Development Goals, UN 17}, we are going to need power. This is the door into the regenerative economy. Instead of an either / or acceptance of power or rejection of power, instead of either accepting or rejecting monetary power, we expand the role of power to include all powers, especially from all capitals\index{capitals}, into a power balance.


The final consequence of capitalism as concretised into our economic practices today is that the capitalist economy cannot do the job of provisioning for us all that it was invented to do. By failing to adequately provision for all, we have all the evidence we need to step away from the meaning\hyp{}making story of capitalism today. This is not just the door to the regenerative economy you all need, it's a great big blooming rocket strapped to your back.


The founders of today's neoclassical economics\index{economics!neoclassical} had a very scientific orientation, and expected that future generations would do just what has happened in physics\index{physics}. Run experiments, gather data, and use hard data to finetune theories. Until the theory could no longer be tuned to include the data, and then to come up with fundamentally better theories. I imagine that most would be horrified at what has happened since then. I believe they would agree with my seeing economics as less the study of what an economy is, and more the study of what you can say about an economy. Just as physics\index{physics} is the study of what you can say about the physical world, and less the study of what the physical world actually is. They would be activists for updating economics to better capture what we can say about economies, and away from declaring what an economy is.


To build the regenerative economy\index{economy!regenerative}, a big capitalism multiplying all capitals that humanity so desperately needs to repay natural capitals and repair the harm we have done to the life-carrying capacity of our planet and society, we must urgently go beyond simplistic either / or approaches. We need the complementary pair approach that quantum mechanics applies so well to nature\index{nature}. It is time to recognise that current capitalism and socialism\index{socialism} are really in the same relationship as the complementary pairs\index{complementary pairs} of quantum mechanics\index{mechanics!quantum}; they are not restricted to only being a binary choice opposites. Some successful countries, like Norway, are successful precisely because they treat them as complementary pairs, integrating good capitalist business with good society to raise the standard of living for everybody.


It is time to harness the value of capitalism (even Marx\index{Marx, Karl} and Engels\cite{marx-engels-communist}\index{Engels, Friedrich} recognised the beneficial power that capitalism brings) for all capitals.
\index{capitalism|)}
\section{Conclusion}
Hopefully by now you are seeing how our economy has been invented by concretising our meaning\hyp{}making, and the difference between what an economy actually \emph{is} (which is inherently unknowable actuality) and what we can \emph{say} about it. 


Central to this book is the theme of complementary pairs rather than opposites. Property\index{property} and freedom\index{freedom} are not opposites, they are a complementary pair\index{complementary pairs}. Stakeholding, stewardship and ownership are a complementary pair, neither identical nor opposites. Freedom and stakeholding are also a complementary pair.