\chapter{Your organisation is its incorporation}
\addcontentsline{toc}{chapterdescription}{How can the incorporation both increase your company’s financial performance and transform it into a regenerative business that addresses our global crises? By having all stakeholders aligned in the full range of ways of benefitting from the business and enabling it to succeed. How to do this? Reincorporate your company as a FairShares Commons: split past from future, money from governance; include all stakeholders and all capitals in governance and wealth sharing. So all benefit: investors, founders, staff, customers, suppliers, communities, and our natural environment.}
\label{chapter:who-is-your-organisation-incorporation}


\begin{chapterquotation}
There is only one way to see things, until someone shows us how to look at them with different eyes.\\
\raggedleft\textemdash Pablo Picasso\index{Picasso, Pablo}
\end{chapterquotation}


Modern business, the tool that we have invented to meet our needs, is no longer good enough. That we are stuck dealing with many issues is visible in the plethora of conflicting opinions in business theory, politics, and economics, on what to do; and the growing number of fudges to keep business as usual going.


This is like the situation in physics and art a century ago. It suggests that we are asking fundamentally wrong questions, like when Rutherford\index{Rutherford, Ernest} asked where an electron is (Chapter~\ref{chapter:emergence-einstein-picasso}).


It is time for the capitalist business concept to transform. As you've seen before in this book, transforming capitalism needs transformational thinking and us to transform our meaning\hyp{}making stories, to bring together apparent opposites and integrate them into completely new complementary pairs. Just as Picasso and Einstein\index{Einstein, Albert} did.


To build a truly Adaptive Organisation\index{Adaptive Organisation} that performs regeneratively over the long term, and to mine and refine all the conflict in order to extract valuable data on how it no longer fits into its niche in society, we need a new capitalism, where all stakeholders are members of the organisation. 


Borrowing from quantum physics, \index{physics!quantum} this means reinventing the elementary particles of incorporation and how they interact. These elementary particles are the shares and the shareholders, interacting via the rights and obligations embedded in the shares they own. 




\section{Where your organisation is: incorporation}
\label{section:where-is-your-organisation-incorporation}
\index{organisation!types of|(}


As always, I use a scale from 0 to 5 so that you can map out where your company lies on the stakeholder and incorporation axis. Each is a concrete meaning\hyp{}making story category, developed in consultation with Rory Ridley-Duff\index{Ridley-Duff, Rory} of Sheffield-Hallam Business School. This ranking is a lens designed to shine a clear light on a progression from the company as the property of a narrow group of people, to do what they want with, through successively greater levels of freedom and inclusion of broader groups of stakeholders and needs, up to complete freedom and broad appropriate inclusion.


On this scale there are jumps. The level 5 Free / Commons is, I believe, the first incorporation form suited to Tier~2 consciousness, called Teal in Reinventing Organisations\cite{laloux-RO} and Yellow in Spiral Dynamics\cite{beck-SD}. (See Table~\ref{table:SDi} on Page~\pageref{table:SDi}.) There is a jump from Level~3 to~4 to~5. Level~3 is spiral green, Level~4 is green with a little yellow, and Level~5 is cleanly yellow or teal.


There is a huge amount of good in the progression we have already made to Level~1, 2, and~3 companies. To build the regenerative, or at least sustainable, economy we urgently need, now is the time to take the next step. I hope that this scale will inspire you to step your organisation up one or more levels. If you are using deliberately developmental practices, or dynamic governance for roles and tasks, anything less than Level~3 makes these fragile. 


If you aspire to antifragile Teal\index{Teal} for your organisation, get yourself to Level~5 incorporation, then strive for Level~5 on the other two axes along the viable valley. Any organisation below Level~4 on each access is inadequate and too fragile for the critical task we have today of creating a thriving society in a thriving natural environment. If you want to build a future-fit, regenerative, circular, or Teal organisation, go for Level~4, ideally~5 on each axis. Especially the incorporation axis.


\begin{description}
\item[5: Free Commons of productive capacity.] ~\newline
The organisation\index{organisation} here is consistently and integrally a free non-human person, across all domains. As such, it is free to act in an optimum way to benefit the entire ecosystem it is in, including having the freedom to choose for itself when it is right to change or even when it has reached the end of its life. Without this, no organisation's task and role design can progress beyond self-directing, let alone reach being an autopoietic member of an ecosystem.


This applies at all scales, from each individual through to each company. Members are free to opt out, and each member within each scale and between scales forms complementary pairs. There is a constant dance between the roles of the individual and the collective at the next scale up, between roles, needs, and offers between members with each scale and between scales. 


Stakeholders engage in the governance of a commons of productive capacity, and represent all capitals touched, using a stewardship paradigm that takes the probable needs of future generations of beneficiaries into account in any decisions. Everyone recognises that the commons is bigger and longer living than any of them. Appropriate sanctions are in place for stakeholders that fail to fulfil their fiduciary duty of care towards the Commons. 


At Level 5 growing all capitals and sharing surplus capitals equitably to all members investing any capital is deep in the practices and culture. A Level 5 company is regenerative in all ways for all capitals and stakeholders because that is the kind of company it is, not because of any regulation nor any incentive. Some Steward-owned companies will have some elements of a Level~5 company\cite{purpose-foundation}, but lack the leap out of all concepts of ownership, or the all-capitals required to naturally form regenerative ecosystems.


This is the best legal incorporation for an investor\index{investor!impact} or founder wanting to protect the long-term integrity of their intent, values, and purpose; and especially any impact, regeneration, circularity, or sustainability outcomes. Because it is a free incorporated being, bad actors cannot later seize control.


This is the first Tier 2 incorporation level, suited to a Teal or Yellow organisation.


\item[4: Governance and wealth sharing across multiple stakeholders.] ~\newline
Has multiple stakeholders\index{stakeholders}, representing all the capitals relevant to the business. All stakeholders receive an equitable benefit across multiple capitals and all invest the capital(s) of that stakeholder category in the company’s success. All capitals count as ‘skin in the game’\cite{taleb-skin}. Individuals can be in multiple stakeholder categories.


Intellectual property is treated as a commons (either closed only to members, or open to the public), curated by the company, and available for use by the originators of that IP, even if they cease to be members. You cannot be alienated from your IP.


All wealth generated is shared equitably, and in particular, any appreciation of the company’s market value as a whole is partially distributed to the holders of investor shares as an increase in the market value of their shares, and partly to the other shareholders in the form of investor shares.


Decisions in general meetings are taken using at least a weighted decision-making process that minimises the possibility of overwhelming the needs of any stakeholder group. Ideally, decisions are taken using a consent process. 


This is the highest Tier 1 incorporation, perhaps with a tinge of Teal / Yellow. The standard FairShares company is an excellent example. Some Steward-owned companies are also at this level\cite{purpose-foundation}.


\item[3: Multistakeholder Coops/For-purpose/basic Steward owned/Self-owning] %~\newline
A company falls into this Level if it is still within the meaning\hyp{}making story of a company as an ownable good, but is half-way out of that paradigm towards being a free commons with all-capitals growth and sharing. It may be structured with a trust, or similar, that has voting rights, and trustees or stewards who are selected for independence and capacity to take wise long-term decisions according to the company’s interests and values as laid out in the statutes.


The statutes and split of ownership between the stewards, the trust, and other shareholders is such that the company as a whole cannot be sold in any normal sales process. One possibility where it might in effect be sold would be if its larger purpose requires it to change how it is incorporated, and so it is sold into another incorporation that is also at least at Level~3, steward owned or self-owning. Depending on the details\cite{purpose-foundation}, it may have elements of Level~4 or Level~5.


It may be a multistakeholder cooperative, but not yet fully FairShares. Some incorporated benefit corporations may be here, not at level 1; some standard cooperatives may be here if they also have clear mandatory vetoes to the sale of the company, and changing any purpose and sale-related aspects. 


I see John Lewis as a level 3 company, along with many others, such as Robert Bosch and Zeiss, that are formally owned by a trust that serves to benefit a broader set of stakeholders than just the financial investors\index{investor!impact}. These are a vital step towards the commons. Perhaps you could consider them a hybrid of a narrow Commons and an owned company.


The Scott Bader Commonwealth Ltd \index{Scott Bader Commonwealth} is another excellent example of a Level 3, tending towards a Level 4 company. The original founder, Ernest Bader, wanted to ensure that his Quaker values continued, and so put all his shares into a trust, with all colleagues as trustees. Their trustee obligation is to act in the company’s best interest for the present and future. 


B-corp incorporated companies (not just the certification) often lie at Level~3. 


The paradoxical shadow side of these initiatives is that, because they act within the myths of Section~\ref{section:myths-of-incorporation}, especially the myth that shareholders own the company, their very laudable efforts at making business a force for good also increase the strength of the myths. A bit like a tug of war against yourself, fighting a meaning\hyp{}making story whilst staying within the story makes the story stronger.


\item[2: Cooperative, Employee-owned company.] ~\newline
One of the oldest and best alternatives to the limited company. Likely most of you have a cooperative nearby, either a worker coop (most of the shares are in the hands of the staff stakeholders) or a customer cooperative (most of the shares are in the hands of customer stakeholders). The cooperative has been, and still is, an essential, and very powerful counterweight to the limited company. More in the next section.


A disadvantage of the cooperative form is the difficulty attracting risk capital; and the risk in general meetings of having only one stakeholder type in the decision making.


\item[1: Employee ownership (Limited company, etc.)] ~\newline
You have an employee stock ownership plan, all staff behave consistently according to your clear corporate values. Sustainability and corporate social responsibility are business strategies, not tangential exercises. Companies with B-corp certification, but still incorporated as typical limited companies, can lie here. They may do superb work, and act as superb role models; but it is fragile, subject to the financial investors\index{investor} agreeing. All it takes is enough voting power in the hands of an extractive investor, and everything can go out the window in an asset-stripping spree.


\item[0: Standard limited company, or analogous organisation. ] ~\newline
Most companies, such as private limited or public listed companies; impact companies; standard non-profit organisations, non-governmental organisations are here. For up until the second half of the last century this seemed good enough, because we could not see the harm it was doing.


This legal form is unsuited to a Level 3 or above developmental organisation or self-governing organisation. Attempting to go Teal on these two axes in a Level 0 incorporation is a highly risky, fragile endeavour. 


For those into spiral dynamics (Table~\ref{table:SDi} on Page~\pageref{table:SDi}), \index{Spiral Dynamics} you cannot build a Spiral level 7 Teal\index{Teal} culture and operations on a level 3 and 4 (red-blue) foundation. 
\end{description}


\section{A share is a package deal}
Recall from Section~\ref{section:myths-of-incorporation} that business is full of myths, \index{myth} concretised meaning\hyp{}making stories believed to be absolute legal truths. 


One of these myths is that shareholders\index{shareholders} actually own the company. Rather, the shareholders own an abstraction called a share,\index{shares} a package deal of rights and obligations\cite{brown-bbaequity}. These rights and obligations are the building blocks you use in constructing a business capable of doing the job at hand; for example, to build a regenerative economy. Like any set of building blocks, you must assemble them in the right way for the job you need done, regardless of how others have assembled theirs. So if you want a regenerative business, assemble the blocks to multiply all capitals and protect all stakeholders from exploitation.


There are six primary building blocks.


\begin{description}
\item [Liquidation rights.] Shareholders come last if the company folds. Shareholders have the right to residual assets after all other obligations have been paid.
\item [Income rights.] The right to receive a share of the dividend in return for your investment, if there is a surplus that the directors decide to distribute.
\item [Appreciation rights.] If the company’s total value grows, the value of your shares may grow.
\item [Voting rights.] The right to engage in governance in a general meeting.
\item [Transfer rights.] The right to transfer by selling, or in some other way, your shares to another person.
\item [Information rights.] The right to information about the company's policies, practices, and performance.
\end{description}


There are a number of pre-packaged deals, differing in countries around the world, and each may have different subtleties and flavours. But there's nothing sacrosanct about any of these package deals. Each is just one meaning\hyp{}making story concretised in law. 


That concrete is not irreversibly solid; it is just as nebulous as the story that created it.


In most jurisdictions, there is some freedom to build your own specific package. For example, voting rights: a share can have no votes or many. In a cooperative, it's one person one vote, regardless of how many shares they have. In a traditional plc, it's one share, one vote. In France, recent legislation has made automatic what has long been possible: people who hold their shares for longer than two years get a double vote. Belgium intends introducing multiple voting rights from 2020.


Appreciation rights can be capped; for example, Enspiral in New Zealand has a 15-times cap on appreciation rights. 


The standard package deals in most countries range from trusts, through cooperatives, to private or public limited companies. The FairShares Commons \index{FairShares Commons} company we propose can be constructed sufficiently well as simply another package deal that integrates the most desirable elements of the full range into one super package. Of course, to get to a perfect FairShares Commons, company law  needs updating.


Each of the traditional package deals below, common examples of incorporation  at levels 0, 1 and 2) has advantages and disadvantages.




\begin{description}
\item[Public limited liability company] If your company has shares that anyone can buy or sell on a stock market, it is most likely this. 


Public limited liability companies are superb tools to attract and multiply large quantities of financial capital. They have superb strength at convincing strangers (and large numbers of strangers in today's crowdfunded companies), who don’t usually even first meet in person, to trust this non-human legal person with their money. 


This is huge. Trust, the deepest foundation of human society, has been externalised from real humans into a non-human institution that is a legally concretised story.


These companies have been a superb engine in raising the standard of living of a number of us reading this book. Sadly they have proven to be equally powerful tools to increase the gap in financial wealth between big investors\index{investor} and the rest because they are designed to multiply only financial capital. 


Typically in public limited liability companies, every share that you buy comes together with one vote in decisions in any general meeting of shareholders. In a one-share, one-vote scenario, a decision taken in the annual general meeting (for example, how much to pay the executives, to sell the company, or buy another one) is the opinion with the most money behind it. 


The problems we are facing today are a direct consequence of decisions taken because they met the needs of the money-majority, and were contrary to the needs of other stakeholders.


Investors are the only stakeholder category active in the company. Other stakeholders can only become active by buying shares, thereby also becoming members of the investor category. Usually, though, they won’t be able to buy enough votes to have any meaningful weight in an annual general meeting.


\item[Private limited liability company]
Many companies begin life as a private limited liability company and become a public limited liability company when they are successful enough to warrant going public via an initial public offering (IPO). 


The governance processes here are typically the same as in a public limited liability company. Investors are the only stakeholder category active in the company. Other stakeholders can only become active by buying shares, but with the same caveats as for public limited companies.


One important difference though: in most jurisdictions, new investors buying into the company need to be known and approved.


\item[Employee-owned limited liability company]
This is a restriction on the limited liability company, one where the voting shares of the company are in the employees’ hands. Even better is if the shares are in a trust that all employees are members of.


The big benefit of an employee-owned company is that all staff can now align their needs with those of any investors\index{investor} and the company as a whole. If the employees take significant risks, such as accepting a 50\% salary cut during an economic downturn, they know that they will all share the benefits in capital gain and dividends when the economy picks up. Employee-owned companies have proven significantly more resilient in the face of setbacks than investor-owned companies.


\item[Worker cooperative]
The next kind of company, going back to the Rochdale Pioneers, is the worker cooperative. Just like an employee-owned limited liability company, a worker cooperative has both limited liability and the shares in the hands of the staff.  


One difference here compared to the Level 4 FairShares or Level 5 Free company, e.g. a FairShares Commons, is that in a coop wealth shared is with the staff only, whereas the FairShares shares wealth back to all stakeholders and investors of all capitals. 


The important differences to a limited company are a consequence of a very different underlying story. The story shaping the reality of a cooperative puts the group of people working together and the business results equally on centre stage. Governance is now one person one vote, regardless of how many shares you have. 


In this case, decisions in a general meeting get taken according to the opinion with the most people behind it. However, they are more likely to take multiple perspectives into account, and some are using the latest dialogue approaches to achieve consensus.


Employees are either the only stakeholder category active in the company, or the most powerful category. Investors can have shares, but they are usually limited to some small fraction of the total capital invested and have limited voting power.


\item[Customer cooperative]
The customer cooperative is very similar to the worker cooperative, except that now the dominant stakeholder category governing the company is the customers. Everything else written above for the worker cooperative applies here too.


\item[Social enterprise]
Social enterprises have also grown rapidly over the past two decades. Social enterprises typically use one of the above forms, although in some countries there are now specific social enterprise company structures available, such as the Community Interest Company in the UK. The significant difference in making a business a social enterprise lies in the meaning\hyp{}making story that underpins it, which puts the benefit of a target segment of society as the primary reason why the company exists. So any company can be a social enterprise, so long as the company purpose clearly state this social benefit, and legally bind the company executives’ fiduciary accountability to serving this purpose.


The stakeholders with the power to govern the company will then be defined by whichever form of company is chosen. Choosing the right set of stakeholders is a very important part of making sure that the social enterprise remains true to its values and objective(s).


\item[Charity]
Depending on the country, a charity may use a specifically defined charitable legal form, or one of the legal forms above. In either case, the company is structured to concretise the reality shaped by a story as being of service to some disadvantaged stakeholder or other form of public benefit\footnote{\url{https://www.gov.uk/guidance/public-benefit-rules-for-charities\#about-public-benefit }}. Because they exist to directly provide benefits to targeted segments of society, charities usually benefit from special tax regulations, both for the charity and for the donors. Charities operate under trust law. Donated money either goes to a core funding of purpose or a specific trust for that group's campaign.


\item[Trust]  
A trust is much broader than, but in some ways has common ground with, a charity, in that it exists to serve some specific non-business objective. A trust is completely constrained to serve whatever need it was created to serve, and no other; and to stay completely within the values and principles defined by the trust founders. A trust is governed by trustees who are elected or appointed and must vote in accordance with the principles of the trust. A trust may be legally incorporated as a charity, for example, the National Trust in the UK is a registered charity. 


\end{description}


All company forms are mythical beasts that we have invented. We, through our individual stories and the stories that we've invented to underpin these companies, have created the reality that we experience today\footnote{For German speakers it's worth reading Jo Aschenbrenner's book~\cite{aschenbrenner-purpose}. \index{Aschenbrenner, Jo}}.


Even though these are all merely concretisations of stories, over time we’ve lost sight of the story. And now we believe the concretisation is the only possible actuality, rather than our invention.\index{organisation!types of|)}




\section{Reinventing the company}
There are two essential lenses to use to see both what a company can become and what we need to do differently to incorporate Level~3, 4, or 5 companies.
\subsection{Separate past from future, money from power}
If you look at the rights above, you will see that they fall into two different types: rights anchored in the past, and rights anchored in the future of the company, covered in  Chapter~\ref{chapter:ownership}.


In the traditional package deals, the past and the future are locked together. You invest money in the past and buy the whole package of rights locked together. Even though there's no inherent connection between your right to benefit from the wealth that the company has generated in the past, and your capacity to wisely steward the company into the right future for the company.


Recall, governance rights are anchored in the company’s future. They give the shareholder the right to engage in steering the company wisely into the future. We can justifiably design a company on the basis that the stronger your connection to the company (not how much money you’ve invested), the more governance rights you ought to have\textemdash for example, anyone who has owned shares for less than six months has no voting rights, but if you’ve owned them for over a year, or a number of years, you have a multiple of voting rights.


This makes sense when you look a bit more closely at the fact that, even if there is only one single shareholder, they do not own the company in the same way they might own a bar of chocolate. Because the company is a non-human legal person just as much as the shareholder is a legal person, company law requires even a single investor\index{investor} to keep it at arm’s length. They cannot directly use the company and its assets as their own. It's time to take that really seriously.


There's also nothing fundamental in the idea that the holder of a share can only get the share by buying it. After all, in most countries the law no longer allows citizens to buy the vote. You can see how silly the idea is that only investor stakeholders get the right to vote, by thinking how silly it seems today that only male landowners once had the right to vote in England. For example, the Fonterra Co-operative Group Limited, New Zealand’s largest company and responsible for around 30\% of the world’s dairy exports, issues shares only to member-farmers, at one share per kilogram of milk solids produced annually.  


Today you earn your national vote through being a citizen in good standing and satisfying certain eligibility criteria. This is part of the foundation of your freedom of self-determination, your freedom to follow your own purpose in life, and for you to develop as is best for you as a free human being.


We can concretise this same meaning\hyp{}making story in our companies. Base voting rights on stakeholder citizenship, including the strength of the stakeholder's long-term commitment to, and dependence on, the company. Then the company is free to develop as is best for itself as a free-living non-human being.


In contrast to governance, the financial rights, (liquidation, income, and appreciation), are anchored in the past. Depending on how much wealth the company generated in the past, you get some percentage.


A hard separation of the future from the past is at the core of a FairShares Commons\index{FairShares Commons} and any other free company. 


\begin{itemize}
\item The people who steer the company into the future all have a strong commitment to the company’s future, and fully recognise it as a living being and a fully free non-human legal entity. 
\item The people who benefit today from the wealth that the company has generated over the past are those stakeholders who have contributed to that wealth generation through any investment of any kind of capital.
\end{itemize}
\subsection{Multiple capitals and returns on investments}
\index{capitals!multiple|(}
Financial capital is not the only capital, and the investors of financial capital are not the only stakeholders investing with risk. Depending on which meaning\hyp{}making story you use, in other words, which frame of reference you use to give meaning to the risk of your investment, you may see an investment of money as being a lesser risk than your investment of ten years of your life as a founder of a company.


After all, if a regular investor gets none of their money back on one investment, they are likely to get it back many times over on another; they can earn money in multiple ways to make good on their loss. On the other hand, until everyone owns a Tardis with a capacity for time travel and to regenerate themselves, there is no way that you will ever get back 10 years of your life. The human capital investment of years of life, by founders and other people working for the company, may be a higher-risk investment in their frame of reference than the investment of money is in the investor’s\index{investor} frame of reference.


Equally, nature has invested significant resources in our economy, which will take millions of years to regenerate naturally. The risk of natural capital investments to human life, especially the investment of waste absorption capacity, is now a major risk to all life on earth. Were businesses and our economy designed around all the risks of all stakeholders across all their capitals, we would make different decisions.


The foundation of an incorporated business is risk-reward. Any risk investment is rewarded, in proportion to the risk, with returns. We are now in a world where we must recognise the investment of all kinds of capital, and the risk that each investor\index{investor} is taking, on an equitable footing. Risk\index{risk} is defined by the meaning\hyp{}making stories, which means working with the interpretation of risk of each stakeholder investing in each kind of capital. 


The reward may be in exactly the same capital and currency invested, or there may be some floating exchange rate mechanism between capitals and currencies. The financial wealth that’s generated because of the investment of all the capitals needed for a business’s activities is part of what must be equitably shared across all stakeholders, in a proportion that balances the relative perception of risk. The financial wealth generated by a business includes both the surplus that the company can generate and the appreciation of the company's value.


The current allocation just to the investors of financial capital, and none to the investors of any other kind of capital, is part of what gives us the dysfunctional rent-seeking we have today.\index{capitals!multiple|)}


\section{FairShares companies}
\index{FairShares companies|(}
\label{section:fairshares-companies}
Few companies can start, scale, and deliver results sustainably over the long term without the investment of external financial capital. To be attractive to financial capital, a company must balance the risk that investors\index{investor} take locking their wealth into the company for many years by offering the possibility of a greater reward. (Recall from Section~\ref{section:ergodicity} on ergodicity\cite{peters-ergodicity-economics} that this is not simple probability.) 


The company must also be  trusted, which means offering investors the right to information, and votes in the general meetings as well. If you give someone some of the power to steer the company into the future, and enough of the wealth that you hope to generate, they may believe they can make the gamble pay off on average. 


Peter Drucker, one of the few people in the past century who deeply understood what businesses are for, pointed clearly at why a business needs to be a FairShares Commons.\index{FairShares Commons} He stresses customer centricity, that every business exists to deliver something that customers are prepared to offer more for than the total delivered cost.  In Procter \& Gamble\index{Procter and Gamble} during my (Graham’s) time, the phrase 
\begin{quote} 
the consumer is boss
\end{quote} 
was regularly invoked in business decisions. This made it relatively easy to use consumer research data, not the HIPPO (highest-paid person’s opinion) as the basis of business decisions. 


The different CEOs running the company during my tenure regularly repeated this message, making clear that we should think of them as de facto reporting into Procter \& Gamble's consumers.


However, there were many decisions where that was clearly not the case. Decisions around the long-term sustainability of our products were quite clearly too heavily driven by meeting the quarterly expectations of analysts and shareholders. As were some decisions around buying other companies, retrenchment, or selling off parts of P\&G, discontinuing or starting a product line.


Customer centricity in many companies can only drive choices within small bubbles, constrained by the immediate investor needs, without any direct dialogue between customers and investors. Any company that truly wants customer centricity must retain the power of a customer cooperative to have the customer’s voice in the largest decisions taken in a general meeting, directly talking with the investors and staff. 


Claiming customer centricity\index{customer centricity} as core to your company is extremely fragile if customers lack adequate direct voice and power on the board and at the general meeting level. (The idea that consumers have sufficient voice and power through withholding purchase fails today, and I doubt if it ever really worked.) 


This is equally true for any other stakeholder group you claim to be part of your purpose, whatever your organisation. If you are bringing clean wood-burning stoves into Kenya, do you have people using your stoves on your board and in your AGM, with voting rights?


Equally, many companies say that their staff are their greatest asset. And yet, when you look at large scale decisions affecting staff, many treat them as a disposable, and they even show up on the balance sheet as a liability. Very few companies keep track of the asset value of their real human capital. 


If a company is going to truly regard staff as part of the company’s asset base, the staff need to have an equitable power to govern the company, and an equitable share of the wealth generated, compared to other voting stakeholders. The company needs to have both the strengths of an investor-centric\index{investor} limited company and a worker cooperative or employee-owned company.


These days companies are also competing with each other for talent, looking for suppliers that will share business risk with them, and asking their end customers to play a central role in building the business’s success. For example, without the core value creation role that the end users play in companies like Facebook\index{Facebook} or Google,\index{Google} those companies would be worth a minuscule fraction of their current value.


Today, companies are more and more dependent on the non-financial capital that is invested. The FairShares company developed by Rory Ridley-Duff\index{Ridley-Duff, Rory}, Cliff Southcombe\index{Southcombe, Cliff}, and colleagues\cite{duff-fairshares}, and my integration with the Free Company to form the FairShares Commons\index{FairShares Commons}, recognises the risk that all investors of all capitals are taking. The FairShares company is designed to offer all investors of all capitals an equitable reward for their risks, and an equitable power in the large-scale governance decisions taken in the general meetings on behalf of the company.


\begin{longstoryblock}
I (Graham) gave a talk in 2018 on the FairShares company at a conference at the RSA in London marking 10 years since Lehman and Northern Rock triggered the 2008 financial crisis, using LTSGlobal (2013) and Evolutesix as examples. One participant was a senior civil servant, who came up to me afterwards to thank me for what I'd said and the hope that my talk had given him: 
\emph{ 
Graham, I am often in talks that begin the way your talk did, pointing out one or more of the problems we are facing, and the way that companies are a major part of the problem. Yours is the first one that did not end by asking us, civil servants and politicians, to first change the law. Yours is the first to end with the message that there is nothing stopping us addressing these problems now with existing company law.
}
\end{longstoryblock}


As you read the rest of this section, keep that firmly in mind. The essence of everything I'm describing can be created in almost all jurisdictions. You may need a little bit of creativity to find the route within the company law of your country to turn the same essence into a specific incorporation. For example, the FairShares outcome that can be achieved via one single FairShares company in English law may need to be a number of different companies in another country's law. 


If you want to build a FairShares company, you will find it easiest if you first work with somebody who has experience in defining business strategy and structures to draft a wireframe statute for the company. Then go to a lawyer who you believe is capable of the out-of-the-box thinking and meaning\hyp{}making needed to find a legal way of concretising that within the company law of your country. 


The essence of a FairShares company is very simple. Its members come from all stakeholder groups with a meaningful investment of any capital and all have an equitable share of the rights to


\begin{itemize}
\item steer the company into the future, i.e. governance rights
\item the wealth generated by the company in the past.
\end{itemize}


The FairShares company applies Picasso's\index{Picasso, Pablo} and Einstein's\index{Einstein, Albert} lenses\index{lens} because it recognises that it's our meaning\hyp{}making stories that narrowly see companies as having to be just one thing or another. What a business actually is, and what it can become, is free of most of the limitations that our stories impose.


For example, why should a public listed company and a trust be opposites? Surely to do the kind of regenerated business that we need today, all businesses must have the structural integrity of a trust, and attract all kinds of capital to accept the risk, by offering the potential to regenerate by enough to more than offset the risk. Quantum physics\index{physics!quantum} and Cubism\index{Cubism} recognise that any entity in actuality can be simultaneously composed, in the form of one complementary pair, of what we view in our limited reality, and the language we can use, as exclusive opposites. 


The FairShares company already offers massive benefits in dealing with today's challenges than any traditional incorporation. (If you turn each of the dials to 100\% and all the others to 0\%, the FairShares company becomes one of the traditional companies. There's nothing really new in here, other than cooking the ingredients into a much tastier stew.)


\index{FairShares companies|)}


\section{FairShares Commons companies}
\label{section:fairshares-commons}\index{FairShares Commons|(}
Integrate all previously opposing elements, such as high impact or high profit, into one complementary pairing: high impact and high profit, and you need an enabling tool like the FairShares Commons.


In Chapter~\ref{chapter:ownership} you read about the different approaches to company ownership and governance. Over the past few centuries, case studies of highly successful yet idiosyncratic businesses have demonstrated that each element of the FairShares Commons leads to superior business outcomes across a wide range of metrics, including profit.


Put them all together and we get the ultimate regenerative adaptive business, maximally able to thrive profitably \emph{and} regenerate all capitals as one single purpose, a complementary pair, through good times and bad, because it can twist and turn as fast as its drivers are changing. We need the full power of financially profitable regenerative adaptive businesses if we are going to multisolve to successfully overcome our global adaptive challenges.


If Picasso \index{Picasso, Pablo} was painting a business today, he would see it as a living being delivering value to other human beings, directly or via other businesses, and he would reject the idea that any single perspective could ever be viable. Picasso would recognise the living being with innumerable inter-related inputs, outputs, and capitals. 


It would be perfectly obvious to him that no real business can ever be adequately represented by any of the standard company forms in use today. It would also be perfectly clear to him that the nature of a business, as a living being, includes an intangible, nebulous component\footnote{As in Daoism, the Dao that cannot be named}, \index{Daoism} where most of what is important cannot be represented in the statutes, or accurately measured. In Picasso's art, as in quantum physics, the inherent nebulous intangibility is embraced and worked with.


He would also recognise that a living being exists across time, not in time. For example, you are yourself across your entire history from birth to death. You can only get a full answer to the question \emph{Who am I?} if you know your life from birth to death (and maybe even before and after). 


So just as you always have time to transform yourself across the full timeline of all of who you are, so too does every business have potential to transform into a completely different kind of living meaning\hyp{}making being, if it is incorporated as a free non-human person. If we can get this done really well, soon, for enough businesses, then there is every reason to believe that we can find a way to use, as part of the solution for tomorrow, enough of the wealth generated by the companies that have become part of the problem today.




\index{stakeholders}
In Section~\ref{section:integral-organisation} you saw one way of looking at an organisation and its individual human members through the four lenses of each integral quadrant. You saw that an organisation is inherently and irreducibly both the nebulous, hidden Human Capability Hierarchy,\index{Human Capability Hierarchy} and the visible, concrete Management Accountability Hierarchy \index{Management Accountability Hierarchy}/ Functional Accountability Hierarchy. \index{Functional Accountability Hierarchy}


You also read how there is no empirical justification for our inventing an impermeable, rigid division of the human beings who form the cells of your organisation into distinct categories like employee, executive, supplier, or customer, just as quantum physics shows that actuality is not actually segmented into waves or particles. All of them are part of the organisation in all four quadrants. 


The FairShares Commons \index{FairShares Commons} gives you the foundations for antifragility by concretising the complementarity of all stakeholders\index{stakeholders} in all quadrants.


To build a regenerative, antifragile, adaptive FairShares Commons, you:


\begin{itemize}
\item  explicitly and visibly include all stakeholder categories;
\item  explicitly and visibly include all capitals (stores of value) and their flows;
\item  surrender forever to the inherently and irreducibly unknowable, nebulous nature of your organisation.
\end{itemize}


\subsection{Include all stakeholders}
Stakeholders in your organisation do not stay in categories with hard boundaries. Each shows up as a whole person, human or non-human, integrating all the stakeholder categories they belong to, into one meaning\hyp{}making living being. All are united by the common oneness of being part of the entity they have a stake in.


Recall the restaurant example of Section~\ref{section-stakeholder-institution}.
Fast forward 10 years. You have scaled to 10 restaurants across London, Paris, and Munich, on a foundation of significant venture capital investment. You hired experienced managers, you have a board, you already have your first Michelin star for the first restaurant you opened in London, and you have customers giving you glowing reviews.


Your business is what it is because of all of its stakeholders, not just you\textemdash exactly as described in the Ubuntu\index{Ubuntu} philosophy. In the beginning, this was visible on a daily basis because the few stakeholders all knew how each contributed to the whole.


But then you reincorporated as a limited company according to the term sheet the investors demanded. You set up a clear business plan, with a hard differentiation between customers, employees, suppliers, and their respective value equations. 


Compare this conventional entrepreneurial startup lens with the FairShares Commons lens: now you clearly see all the potential for value generation in impact and profit that slips through the gaps created by stakeholder divisions. Customers and suppliers cannot optimally interact with each other, nor are they motivated to do word of mouth marketing.


What's wrong? The dominant logic of today's business stories. A logic that says everybody fits into distinct categories that exclude the other and each category is in competition with every other category. 


These myths have created the reality we all experience today.


The new meaning\hyp{}making story\index{meaning-making stories} I propose is accepting that every stakeholder shapes the living being that your business is. It will thrive better if you eliminate any barrier to them investing their time and energy into it, in a partnership across all stakeholders. 


The more interaction you have across all people in your business, the more that human energy is turned into results (Section~\ref{section:interactivity}). And the FairShares Commons\index{FairShares Commons} gives you this across all stakeholders, not just staff.


Now get into your space elevator, and up outside the atmosphere. Look down at our blue planet Earth, and everything that needs to be regenerated to get each living system that human life depends on fully regenerated. 


Imagine what such a regenerative business ecosystem\index{ecosystem} could achieve if it reached maximum interactivity because it no longer imposed hard boundaries between stakeholders? If the organisation’s \index{organisation} visible legal form accurately represented the true nebulous, permeable, soft nature of the boundaries between stakeholder categories? Imagine if all stakeholders\index{stakeholders} were able to invest their unique energy and capitals to your business, united by their common ground of thriving? 


Then each business activity is in the best position to multisolve our challenges.




\subsection{Include all capitals and currencies}
The wealthier your company is, the more capacity it has to deliver the output that society needs at large scales, and to adapt to the changing demands of its niche. Wealth\index{wealth} here includes far more than financial wealth. It means all of your company’s capitals, and its ability to access those capitals so that they are put to work.


The different stakeholders of each company bring in different capitals, each vital for that company to succeed. The companies of today need to have all of them, and they need to be designed so that all their values (not just financial capital) are multiplied, and so each of them has the appropriate currencies (not just money). 


The FairShares Commons company is designed to make it easy to have all capitals and all currencies fully active because each is represented with voice and power at the general meeting / share level. It is an ideal building block for companies in the regenerative, all-capitals, all-currencies Economy of the Free \index{Economy of the Free} of Chapter~\ref{chapter:economy-of-the-free}.


Whilst I believe that the FairShares Commons is the best, there are other approaches addressing the flaw in the current dominant logic that have also emerged and grown over the past 12 years since I started working on this, including:


\begin{description}
\item[Purpose Foundation] \index{Purpose Foundation} In Berlin, this group spans active investment, consulting, and engaging with politicians to change the law. Their approach centres on removing any option of selling a company as the means to protect its long-term purpose, and splitting voting from financial rights. 
\item[B-corp certification] \index{B-corporations} provided by the B-Labs, applicable to for-profit corporations. Any FairShares Commons company is highly likely to satisfy the requirements for certification.
\item[Benefit corporations]\index{benefit corporations} A type of incorporation that legally entrenches multiple stakeholder needs into the company objects (i.e. its various purposes). This includes some, but not all, of the FairShares Commons elements, and may depend still on investor goodwill in voting.
\item[Conscious Capitalism] \index{capitalism!conscious} A movement and organisation promoting business as a force for good, based on the conscious capitalism credo\cite{mackey-conscious}.
\end{description}


\index{FairShares Commons|)}


\section{In closing}
The question that I have used since 2008 to drive the creation of much of what is in this book is: 


\begin{quote}
In this dysfunctional system that we have, what is keeping it stable in its dysfunctionality? What is it that has defeated all the valiant attempts to make business a force for good?
\end{quote}


This question led inexorably to the FairShares Commons\index{FairShares Commons} company form, where the company is a living, meaning\hyp{}making being, a non-human legal person with all that free personhood entails, starting with my conclusion in 2010 that the whole concept of a company as an ownable good is fatally flawed. That concept is a barrier holding businesses and ourselves back in our time of need.


This integration of the Free company, or commons company, and FairShares, is something that I believe is at least an essential next step towards the kind of businesses that are fit for the future. The kind of businesses that Einstein\index{Einstein, Albert} and Picasso\index{Picasso, Pablo} would build if they were in business today, and the Economy of the Free they would envision.