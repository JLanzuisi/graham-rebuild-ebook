\chapter{Building a general theory of economies}
\addcontentsline{toc}{chapterdescription}{We need a general theory of economies, one that is falsifiable, predicts well the outcomes of any intervention, fully incorporates or aligns with all we know and will learn about physics, chemistry, human nature, motivations, the Cognitive Developmental Framework, and all else. In particular, an economics based on all 8 billion unique realities of our global population, and the non-ergodic nature of these realities and economics. Such an economics is then always relative to everything else. So Einstein’s General Relativity may well give us good ideas on how to formulate it. This chapter suggests one way: accept that all decisions are rational in each person’s local reality, then focus on how to connect one reality with another. Enter a decision spacetime.}
\addcontentsline{toc}{chapterdescription}{\pagebreak}
\label{chapter:gen-th-economies-build}


\begin{comment}
Jack comment: our effort to do this parallels the 19th century movement by utilitarians (e.g., Jeremy Bentham) to actualize interpersonal comparisons of utility (i.e., satisfaction). This idea would be worth expanding on in our next book. And perhaps we can use the term interpersonal in one way or another? Lacked the right lenses to look through, didn’t have the comparisons with cubism and general relativity, nor modern understanding of adult development, and so could not see clearly enough what they were pointing at.
\end{comment}


You can skip this appendix without losing anything if you want to know how to build a regenerative business, or regenerate yourself. 


Jack and I decided to include this for those of you wanting to dive deep into a potential future alternative to current economics; we hope it will provoke a radically new box to rethink economics into, by both highlighting deeper reasons why we are in the mess we’re in, and what we must do to get out. 


Keep reading it if you get energy from it; otherwise skip it now and return to it on a cold winter evening. 


The foundation of a general theory of economies, \index{general theory of economies}  inspired by Einstein’s General Theory of Relativity, began by asking what we see looking at our economy through the central lens of this book: each of us experiences our own unique reality. This yields the basis for a central theory of economies:\index{general relativity}


\begin{quote}
everyone always takes rational decisions. \index{rationality}
\end{quote}


These rational decisions are rational according to the unique inner reality and resulting frame of reference in the moment the decision is taken. So, just as the essence of general relativity is a way of connecting spacetime frames of reference from point to point, the essence of a general theory of economies is a way to connect one reality at one moment in time in one person to that of themselves at a different time and place,  or to another person at some place and time. In other words, a way of connecting one decision frame of reference \index{frames of reference} with another, an economics aligned with Otto Laske's CDF. \index{Einstein, Albert} \index{Laske, Otto}


So how can we connect your decision frame of reference now with your decision frame of reference later, and the decision frame of reference of someone else?


We can only compare decisions and their rationality if we can transport realities and frames of reference \index{frames of reference} (thought form P6 of Table~\ref{table:28DTF}) from one decision point in spacetime and person to another. (You'll see where this is going in a moment, unless you're familiar with general relativity, in which case you may already see where we are going!)


This general theory of economies\index{general theory of economies}  differs from the usage in J.M. Keynes’ \emph{The General Theory of Employment, Interest, and Money}\cite{keynes-general}, which is also referred to as \emph{The General Theory}; although in his Collected Works he alludes to Einstein’s work as an inspiration. Keynes \index{Keynes, J.M} might view the general theory of economies here as a natural consequence of bringing together cutting\hyp{}edge understanding of individual adult development (Size of Person, inner reality, etc.) and \index{Size of Person} social development (e.g. of cultures, as described by Spiral Dynamics) with at least some of the key concepts that led him to write his book\cite{galbraith-keynes, keynes-collected-writing}. \index{Spiral Dynamics} 


This approach makes clear that economics\index{economics} is the study of what we can \emph{say} about decisions and how they affect provisioning, not about what \emph{is}. This broadens economics beyond human society, and certainly beyond any one worldview, such as the European, American, or Chinese business models. This is very much the same as the realisation in physics that physics is only about what we can say about how the world is, not the study of what actually is.


This also highlights the complementarity between economics and business theory. Economics\index{economics} is about all decisions and all the different things and meaning\hyp{}making that shape them. Business theory is about how to take decisions, including those on how to execute in order to actually move and combine resources to create value, and you will see how business theory lies within a general theory of economies.




Whilst Keynes and his successors could never aspire to calculate macroeconomic outcomes bottom\hyp{}up using a large number of individual decisions makers, each with a different inner reality and frame of reference, including all recursive feedback loops, today we can, using the same supercomputers used for simulating weather and climate. 


Perhaps the phenomenological models that form much of economics\index{economics}  today, which were the best that could be used before today’s supercomputers, can soon be replaced by bottom\hyp{}up calculations based on billions of people, each with their own unique experienced reality and frames of reference, constantly changing along their paths through life. All we need now are people brighter than I (Jack and Graham) am to build such a general theory of economies.


\section*{How does general relativity describe gravity?}
To use general relativity as a metaphor, to guide us towards an hypothesis for a general theory of economies, we need to understand just enough of the design principles of general relativity.


General relativity\index{general relativity} is captured in one of the most beautiful and symmetric equations in all physics:
\begin{equation}
        G_{\mu\nu} = T_{\mu\nu}
\end{equation}
$G_{\mu\nu}$ (also written as $\mathbf{G}$), the Einstein tensor, \index{Einstein tensor} describes how spacetime\index{spacetime} is curved, and
$T_{\mu\nu}$ (also written as $\mathbf{T}$), the stress\hyp{}energy tensor, describes how much mass, energy, \index{energy} and physical stress there is at any point in spacetime.


The physicist Archibald Wheeler\index{Wheeler, Archibald} summarised the beauty of this equation: \begin{quote}
matter tells spacetime how to curve, and spacetime tells matter how to move.
\end{quote}


There is no gravitational force anywhere in this equation. It says instead that matter is always moving in straight lines inside a curved spacetime. Curved spacetime and matter actually exists, gravitational force is nothing other than a simplifying mental crutch humans have invented so that we can make approximate predictions using simple equations. Wheeler's summary makes clear that what we see as the Earth moving in a circle around the sun is, in actuality, the Earth moving in a straight line through a spacetime that has been closed in on itself by the sun’s mass.


Spacetime itself is the circle.


Newton\index{Newton, Isaac} thought that there were three completely distinct elements: space, time, and force of gravity.\index{gravity} Einstein\index{Einstein, Albert} realised that special relativity (forces and moving objects) didn’t fit Newton's concept of gravity, and therefore Newton’s gravity must be fundamentally flawed, even though it had worked well enough so far. 


Einstein's intuitive leap was to realise that the effects attributed to the force of gravity could just as easily be the effects of curved spacetime in the absence of any force whatsoever. 


The Einstein tensor, $G_{\mu\nu}$, unpacks into a fiendishly complex set of equations, that most physicists only get to see in their third or fourth year at university, and begin to solve in the simplest of cases. It describes, at each node in spacetime, how to carry the vectors (like pointers pointing) from one node to the next.


Imagine you are standing at the South Pole. In every direction you turn, you are looking north. Think of a cluster of arrows, flat on the Earth's surface, at the South Pole, fanning out in a complete circle, all pointing north. Now you want to transport each of these arrows along its own line of latitude towards the North Pole. Each, as they move towards the North Pole, must stay flat on the earth's surface. By the time they get to the Equator, when you compare one arrow to its neighbour, they are all parallel to each other now, not fanning out hedgehog style. By the time they get to the North Pole they are all pointing into each other, not away from each other.


That is what $G_{\mu\nu}$ does. It describes how to transport pointers from one node to other nodes in spacetime, constrained by the necessity of staying in spacetime along the entire journey. These pointers are a frame of reference. So if you think of the geometry of spacetime as infinitely many nodes, and the connections between the nodes, with directions pointing forwards in time and space on each three spatial and one time dimensions, $G_{\mu\nu}$ tells you how to transport these pointers, or frames of reference, inside the curve of spacetime. 


So spacetime\index{spacetime} is closed in on itself into a circle halfway around these pointers, which will be pointing in the opposite direction. This gives us a clue on how to transport meaning\hyp{}making frames of reference from one decision to the next.


This means staying within the two dimensions of the Earth's surface, which are curved and closed in on themselves. If you were to walk from the South Pole due north to the North Pole in a straight line, and when you got there just keep going in the same straight line, you'd arrive back at the South Pole. You could keep going forever, as you would be moving in a never\hyp{}ending straight line in a curved space.


The stress\hyp{}energy tensor, $T_{\mu\nu}$, captures everything that actually exists at each point in spacetime. How much mass, energy, stress\footnote{Stress is defined in physics by looking at an infinitesimally small box in spacetime and is composed of the combination of forces that are stretching or squashing along one axis, and the forces that are twisting clockwise and anticlockwise around the axis.}, and momentum; pointing in which direction; is present. 


Putting Wheeler's words into symbols now, $T_{\mu\nu}$ tells $G_{\mu\nu}$ what to be, and $G_{\mu\nu}$ tells $T_{\mu\nu}$ how to move.\index{Wheeler, Archibald}


What might a theory of economies look like that is based on geometry instead of force? 
We know that physics was fundamentally transformed when Einstein \index{Einstein, Albert} realised that everything attributed to gravitational forces could be better described by a force\hyp{}free geometrical theory, let's look seriously at a geometrical general theory of economies, and see where it might get us.


Geometry\index{geometry} is about all the nodes, or points, and how adjacent nodes connect to each other. By connect, I mean how a Frame of Reference at one node is transported to the next node, i.e., changes as it is moved. In a theory based on force, the geometry of spacetime (the Frame of Reference) is just an arbitrary mathematical construct to use as the basis for your calculations, and stays the same at each node. 


In a geometric theory, spacetime\index{spacetime} is no longer a mere arbitrary mathematical choice for calculating with. The shape of spacetime is the origin of the physical effects we see; such as the GPS satellites orbiting the Earth in a stable way. 




\section*{G in economies}


What might $G_{\mu\nu}$ represent in a general theory of economies?\index{general theory of economies} We need to look for some equivalent to spacetime, with curvature replacing economic forces. Then, many economic ‘forces’ in current economics thinking will emerge as fictitious concepts, analogous to sunrise and sunset being convenient fictions easing daily language. It’s far easier to talk about enjoying a sundowner at sunset than an earth rotator at the distal extreme of the diurnal rotation!


So we need to define the nodes of an economy, how the nodes are linked, and what is transported from one node to the next.


I propose that $G_{\mu\nu}$ represents the space of all possible decisions, across all physical spacetime. At each point in this decision spacetime, $G_{\mu\nu}$ carries the pointers, or frames of reference in a decision at that point.


This is building on centuries of work on what economics\index{economics} can validly say about the economy: that at its core it is a sequence of decisions, choices between options.


Any of us who is taking a decision will be at a specific node in decision spacetime. We will take a rational decision according to the rationality of exactly that node in decision spacetime.


If we were to take the same decision a little later, we would be taking that decision at a different node in decision spacetime, where the decision tensor (reality, or frame of reference pointers, i.e., meaning\hyp{}making) might have changed direction. They might even, like the Earth moving around the sun, have moved halfway around a circle and lead us to take the exact opposite decision. Both decisions are completely and 100\% rational, even though each opposes the other.


Looking at decisions as always being 100\% rational, but taken in a curved decision spacetime, we now have a way to make better sense of the relationship between all the different perspectives in economics. We just need to now find a way of formulating decision spacetime accurately, and especially how $G_{\mu\nu}$ transports the frame of reference defining rationality at one node to the next.


Now you can see what is actually happening if an economist struggles to grasp decisions taken by other people. Neoclassical economics \index{economics!neoclassical} assumes decision spacetime has a Cartesian geometry\footnote{Keynes \index{Keynes, J.M} actually has a paragraph on the flaws in economics due to this assumption in his draft \emph{General Theory} \index{The General Theory (Keynes)} manuscript\cite{galbraith-keynes, keynes-collected-writing}}. In this, if you transport the frame of reference defining rational from one point, one person, or one moment to another, the frame of reference continues to point in the same direction. So what is a rational decision at one node in decision spacetime\index{spacetime} is always the same rational decision at all nodes in spacetime.


Therefore, if a different decision is taken, it must be because either the person is irrational, or under the influence of some force that overcomes a free decision, in an economics\index{economics} based on Cartesian geometry and forces.


Replacing this with the idea that decision space-time itself is curved, i.e., everyone’s Frame of Reference is unique, so two people’s decisions can only be compared after transporting the Frame of Reference from one node to the other, needs to be thoroughly explored and tested to destruction. 


There may, of course, be some even bigger abstract space; or perhaps more research will show that there is no way of describing economies on the basis of the geometry of decision spacetime. Finding out will require rigorous scientific research, gathering hard data capable of falsifying our hypothesis unambiguously. But, whatever the outcome, we must find a way to integrate the understanding of reality and economics. 


Value is in this space and forms part of the frame of reference. In a curved decision spacetime, value changes as you move from one node in decision spacetime to another. $G_{\mu\nu}$ gives you all you need to know about how decision spacetime is curved, and therefore how value changes (is transported) as you move through decision spacetime.


I am at a node in my decision spacetime, using the decision frame of reference of that node, which includes any additional forces, to take a rational decision. As I move forwards in time my decision frame of reference will shift, pointing in a slightly different direction. 


Some of this change is due to my meaning\hyp{}making stories\index{meaning-making stories}  changing as I internalise experiences from earlier decisions. This is clearly not ergodic.\index{ergodicity}


This change in my frame of reference \index{frames of reference} is captured by the curvature in the path I move along over the year, and $\mathbf{G}$ describes how that path curves. So $\mathbf{G}$ transports rationality of now to rationality in the future or past, however far I want to go, enabling me to understand the rationality of previous decisions even if they appear irrational in my current reality.


Each point, or node, in decision spacetime is abstract, and is independent of any entity taking a decision. 


These entities may be individual human beings, families, multinationals, or even global governance bodies. If each moves along their natural path in decision spacetime, every decision they take will always be the most rational decision available at that point, and with the information they have.


All this makes $\mathbf{G}$ a phase space of all possible decisions, meaning\hyp{}making stories, including values, principles, internalised culture, etc. These give the “how much” value on the $\mathbf{G}$ side is attributed to the things and all possible spacetime positions.










\section*{T in economies}
What might $T_{\mu\nu}$ represent in a general theory of economies? We need to look for what actually is, what moves, what forces exist, and what contributes any degree of shaping to the geometry of $G_{\mu\nu}$.


I propose that the key elements of $T_{\mu\nu}$ are things (food, houses, gold, knowledge, etc.); and potentially forces (e.g. a threat of violence if you fail to take a specific choice); that might have value attributed on the $\mathbf{G}$ side or not by the meaning\hyp{}making stories active at a given node.


There may be more elements of $T_{\mu\nu}$ that are not part of this category. I hope there are, and maybe you will be one of those able to identify them and flesh out what a general theory of economies fully includes.


So $T_{\mu\nu}$ describes everything that becomes a resource\index{resources} (resources are things with the meaning\hyp{}making that attributes value to them), power, politics, memes, feelings, movement, etc.\textemdash whole swathes across all physical and social sciences.


$T_{\mu\nu}$ also describes what is actually done as a consequence of decisions. All human behaviour lies here: the work that we do, individually and organised together in businesses, etc. All organisational theory and human behaviour must be included in this general theory of economies, and it ought to describe all of how organisations \index{organisation} actually work in different kinds of contexts, in different ways of organising, and embedded in different overarching meaning\hyp{}making cultures.
\section*{An economy is $\mathbf{G} = \mathbf{T}$}
$T_{\mu\nu}$ tells $G_{\mu\nu}$ how to curve. It shapes the meaning\hyp{}making stories, values, etc., creating the shape, the curvature of decision spacetime $G_{\mu\nu}$.


$G_{\mu\nu}$, the space of all meaning\hyp{}making stories and consequent decisions, then tells all the things how they should move (linking supply and demand), their usefulness or value, and how they ought to change.


In this general theory of economies, the decisions (buy and sell, pricing, how to incorporate, what is waste versus resource after recycling, etc.) are always locally rational at each point in decision spacetime. 


It's only possible to construct a theory of rational decision\hyp{}making if that theory fully includes everything impacting each individual's locally and internally constructed reality, and describes accurately how that reality is to be transported faithfully from one node in decision spacetime to the next ($G_{\mu\nu}$).


Compare this geometric general theory of economies to two criticisms often levelled at neoclassical economics~\cite{keen-debunking, earl-econocracy}. \index{economics!neoclassical}


\begin{enumerate}
\item The assumption that everyone is rational is not true.
\item The lack of pluralism, and lack of seamless integration with all other relevant disciplines, ranging from the physical boundaries placed by laws such as energy conservation and entropy, through to the latest cutting\hyp{}edge insights from the social sciences and neurobiology.\index{neurobiology} 
\end{enumerate}


In this general theory of economies, \index{general theory of economies}we show how the neoclassical economics assumption of rationality is retained, by showing that any discussion of rationality \index{rationality} can only be done after describing in sufficient detail the frame of reference, defining what is and is not rational, used by the decision\hyp{}making entity. 


We show that decision frames of reference are, and hence rationality is, dependent on exactly where in decision spacetime the decision is located. Reality\index{reality} is curved: it can only be approximated in a sufficiently small and locally ergodic approximation; all other decision evaluations must use a curved transport of the frames of reference.


This general theory of economies embraces all pluralistic aspects that impact on an economy, either on the $T_{\mu\nu}$ side shaping decision spacetime, or on the $G_{\mu\nu}$ creating the decisions telling things what they are, what value they have, and how to move or transform from one place to another, from lower value to higher value or from higher to lower value.


For example, the elasticity of supply and demand in neoclassical economics\index{economics!neoclassical} (the re\hyp{}establishment of equilibrium after some change in supply or demand leads price to exert a force on the other) becomes a force\hyp{}free geometric effect in decision spacetime itself: it is stretched or compressed, and will describe exactly the outcomes we see without any need for market forces. This stretching / compressing is caused by the stress part of $T_{\mu\nu}$.


I believe that replacing the concept of market forces, and all the algebra of forces in current economic thinking, with curvature in decision spacetime, as well as things in $\mathbf{T}$, will give economics the same leap in power to understand and predict that physics gained when Einstein recognised that there was no such thing as a gravitational force, despite all that that illusion enabled.\index{Einstein, Albert} 


In this geometric theory, all theories of value have their place. The labour theory of value is just a local approximation to an area of decision spacetime with meaning\hyp{}making stories in which labour is the dominant source of value, a utility theory of value is just a local approximation to another area of spacetime with meaning\hyp{}making stories around utility, and any intrinsic value is also part decision spacetime’s shape.


None of these theories of value can ever be right or wrong; more, they are better seen as local approximations in a small enough domain of decision spacetime where the curvature is negligible compared to the outcome of decisions. 


Supply and demand\index{supply and demand} are central to economics, and neoclassical economics is based on looking at them as two fundamental forces. You achieve an equitable price when the two forces are exactly in balance. In a general theory of economies, these disappear as forces. Rather, supply and demand are two different kinds of mass that shape decision spacetime, creating the meaning attributed to abundance, scarcity, and each item being at some price point.


Both ergodic\index{ergodicity} and non\hyp{}ergodic aspects of an economy may be described by a geometric general theory of economies because the decision spacetime includes what physicists would call phase space. This decision spacetime extends across all time, from the first decision to the last decision in society (and maybe even in the entire universe). All paths are present, and include the full impact of history, of times of stability and times of revolution, path\hyp{}dependent non\hyp{}ergodic cumulative decisions, etc.


This geometric approach \index{geometry} to constructing economics might deal with all concepts of competition and collaboration, from a single monopoly and no collaboration, through to perfect competition or collaboration, and the entire continuum of actual competition\hyp{}collaboration between the extreme limits. 


Neoclassical economics, where each person and decision are independent of each other and independent of time, emerges as the Cartesian approximation to competition in a geometric general theory, i.e., an approximation that’s only valid where each point in decision spacetime is independent of each other point, and there is no curvature.\index{economics!neoclassical} 


Also, the emerging class of prosumers is well described here. This is another complementary pair that must be part of any general theory of economies.


Tension and conflict\index{conflict} play a central role in the general theory of economies. They are in $T_{\mu\nu}$ and shape how the rational frame of reference \index{frames of reference} is transported by $G_{\mu\nu}$ through time (for example, one specific decision\hyp{}making entity, whether a human or corporate entity) as they change their meaning\hyp{}making stories by mining and refining what the tension is telling them about how to make their local, internally constructed reality a better and better match to actuality.


The value of tension\index{tension} and conflict between people to steadily increase the viability of the decisions they are taking, especially in an organisation that requires them to collaborate and turn those decisions into valuable output,  is now visible.


This general theory of economies\index{general theory of economies} also captures cleanly how different individuals in different parts of an organisation will take decisions that are quite different from each other in exactly the same situation and looking at exactly the same data, and yet each is taking a fully rational decision. Each is at a node some distance apart in decision spacetime. Of course, immediately after a strategy discussion and alignment, they may each be at nodes that are very close together, but because decision spacetime is curved, as they move forward through time they may move to nodes that grow further and further apart. So the choices that each makes later on, looking at the same set of options, will grow further and further apart.


Whether any specifics of general relativity are present in a general theory of economies is going to require significant research spanning disciplines from pure mathematics through to social sciences and neurobiology. I have used $G_{\mu\nu}$ and $T_{\mu\nu}$ as a metaphor to provoke a new perspective; we still need to determine if it is indeed a tensor, some other holor, or something else completely that we need. 


That will require scientific research, rigorous enough to falsify everything false in this approach. This will take the kind of massive simulations I (Graham) did in physics, known as Monte Carlo simulations, to have a workable approximation to the path integral of each person (Section~\ref{sec:who-am-i}) taking into account everything non-ergodic in each person’s decision making and the economy as a whole (Section~\ref{section:ergodicity}). The more accurate the simulation needs to be, the further into the past and future it needs to do the path\hyp{}integral of each person represented in the model, the more variation it needs to have in identity and life influences, the more relationships with others, and the more people it needs to simulate. It may need to run multiple times, creating an ensemble of configurations, leading to a probability distribution of economic outcomes. This ensemble also will include in the simulation a recursive awareness of such simulations.


These ensembles can then be used to test the theory, by comparing the outcomes from the ensemble average with historical data; and only once we have tested to falsification all we can, are we able to use what is left to begin tentatively predicting the future.


Hopefully this approach can catalyse a useful general theory of economies. 


\section*{What about neoclassical economics?}
\index{economics!neoclassical|(}
Neoclassical economics requires the kind of fudges that physicists\index{physicists} take as an absolutely sure sign that they need a new theory. 


We know it is time for a new theory of economies\index{general theory of economies} because neoclassical economics does poorly when trying to describe and predict the emerging zero marginal cost economy, gift economy, care economy, emotional labour economy, etc. However, in some cases it seems a good enough approximation to be useful.


In general relativity,\index{general relativity} if you look in a small enough region of spacetime to form an approximate inertial frame of reference, you can't tell the difference between Newton's gravity\index{gravity} and Einstein's gravity. How small that box needs to be depends on how tightly spacetime is curving. If you're in the middle of nowhere, where $G_{\mu\nu}$ is flat, the approximate inertial frame may span millions of years and millions of kilometres in each direction. If you are close to a very high density of $T_{\mu\nu}$, like a black hole, \index{black hole} the approximate inertial frame of reference may be smaller than your little toe.


I expect that we will find the same to be true with neoclassical economics. There will be some small enough domain in decision spacetime where $G_{\mu\nu}$ is flat across large numbers of people, and across long periods of time, and you have an approximate decision frame of reference that looks just like neoclassical economics.


However, if you are close to a concentration of power, where there is a steep power gradient from one point in decision spacetime to the next; or, if you are on the boundary between two cultures, where there is a steep gradient of meaning making, then the approximate neoclassical frame of reference may be so small that it is no bigger than one person over one second.


Steep meaning\hyp{}making gradients can occur across physical space, for example between two people in a marketplace with very different cultural backgrounds or personalities, or two neighbouring countries with very different cultures. It also occurs across time, during epochal shifts, such as we have today.


If this kind of geometrical general theory of economies\index{general theory of economies} is a useful description of how an economy actually works, it will describe the decisions being taken now and over the next 10 years as the rapid changes in $T_{\mu\nu}$ shape the very tight curvature in decision spacetime $G_{\mu\nu}$, and vice-versa.


These power \index{power} differentials lie in the meaning\hyp{}making of each individual. For example, I might threaten to retrench somebody from my company, unless they do what I say. If that person is completely dependent on their monthly paycheck to survive, they will see an existential threat and will do exactly what I say.


However, if that person already has their own startup underway, just ready to launch, and being retrenched will give them what is missing in time, commitment, and seed money, they will make meaning of being fired in an overwhelmingly positive way and will then do exactly the opposite of what I want, so that I fire them.


The curve in decision spacetime may be so steep that their decision changes from 10:00AM to 10:01AM.


Equally, who I am, my meaning\hyp{}making stories, my relative position in the company, and much more, can also make that a very tight curvature.


So to fully capture the rationality of two different choices separated in time or made by two different people you need to understand how to transport the rational decision\hyp{}making frame of reference from one point in decision spacetime to the other. That is what $G_{\mu\nu}$ does.


The curvature of our economic space, or decision spacetime, then tells us which choice to make on resource allocation. This includes what decisions to take in politics, relationships, and all other kinds of intangible value.


So I propose that any full theory of economies must be equally applicable across both the tangible and the intangible, and across all time, including explaining how the economic value changes across time. If not, it is at best an approximate phenomenological model, not a theory; and at worst a fantasy we force onto society as if it were truth.


In Newton's triad of independent space, time, and force, the units you use to measure each are independent and have no inherent meaning.\index{Newton, Isaac} 


Relativity\index{relativity}  and quantum physics, however, make clear that there are natural units. For example, the speed of light is the natural unit linking space and time. The speed of light in nature is one: the unit of space equals the unit of time. It is only our uninformed choice of metres and seconds that gives us a big speed of light.


Many numbers used by engineers to figure out how to build something have nothing to do with physics, but rather are artefacts of the units that we invented before we understood quantum physics and relativity.\index{quantum physics} 


Equally some units in economics will turn out to be 1, because they are inherently natural; anything else is an artefact of an uninformed approach. For example, aspects of pricing, such as the exchange rates between different national currencies, are artefacts. These are now part of curvature instead.\index{economics!neoclassical}




\section*{Value and pricing in a general theory}
Value\index{value} is at least partly, if not wholly, attributed by the meaning\hyp{}making frame of reference \index{frames of reference} specific to a human meaning maker, in some emotional state, at some point in time, at a place, in a context. This value lies in $\mathbf{G}$, where changes in value yield the curvature of $\mathbf{G}$. Even if a decision is being taken by a computer program, or even an artificial intelligence, the meaning\hyp{}making (including emotional), of the programmer or the dataset used to train the AI is irreducibly part of the value.


The general theory of economies\index{economics!neoclassical} gives us a mechanism to take the drivers of value\cite{mason-post-capitalism} on the $\mathbf{T}$ side and bring them all together into the rational decision on the $\mathbf{G}$ side at each node in decision spacetime.


So whilst neoclassical economics is based on an inherent substance to value in an object or service, in this proposed general theory of economies,\index{general theory of economies} value comes from all of the different kinds of meaning\hyp{}making that anybody might have; all their values, principles, beliefs and religions; now, in the past, or at any time in the future, and across all situations.


This then creates the curvature in $\mathbf{G}$, including the momentary value in the local frame of reference that anything has at a specific point in decision spacetime. Value can be attributed to something from any and all kinds of meaning\hyp{}making, which means that value can have a highly convoluted geometry from one person to the next. 


Value can be like the scissors, paper, stone game, except with infinitely many elements, not just three. Just as stone beats the scissors, paper beats stone, and scissors beat paper, so too with value. There can be no universal Cartesian theory of value, not universal rational decision, when value can have a large or even infinite number of axes that cannot be put onto one single Cartesian geometry.\index{geometry} 


This integrates all current approaches to value in economics into one theory of value. Utility is clearly part of this, but equally, emotional labour and the intangible values (trust, love, reciprocity) of emotional labour shape $\mathbf{G}$ into intangible value geometries. The emerging realm of green, blue, sustainable, etc. economics, Sharia economics, and everything else shows up as additional dimensions in a multi\hyp{}dimensional within the much larger decision spacetime.


From this you can also see that the neoclassical economic concept of value\index{value} is too small to base a theory of an economy on. At best, it is useful for a very small, local approximation to a theory. $\mathbf{G}$ encompasses all kinds of decisions, at each point in decision spacetime, and deciding what value something has is one class of decisions in the entire decision spacetime.


It also shows that pricing, which shows up on the $\mathbf{T}$ side, is a local artefact of the currency you use (for example, positive interest bank debt, a.k.a. money) to parametrise the underlying curvature of decision spacetime. In a full theory of an economy,\index{economy} the relative ranking in price of different things will depend on which currency you use. In practical terms, this means that planting an oak forest will be of higher value than planting a pine forest if you use the Terra as currency; or vice versa if you use money as the currency; and depending on which of the many other aspects of $\mathbf{G}$ value you decide to use to define the rank order.\index{currency} 


This also means that on the $\mathbf{T}$ side we need at least as many currencies as there are kinds of value. Just as you cannot rank scissors, paper, stone, in a linear Cartesian basis, you cannot parametrise all kinds of value with money alone.


Our definition of property\index{property} as a situation where one single entity, or a few close entities, has the sole decision authority over a thing can now be represented cleanly on the $\mathbf{T}$ side. Property now shows up in the maths as a force in $\mathbf{T}$, acting on a thing, pushing it off the curve in $\mathbf{G}$ that it would naturally follow were it free.


It’s now visible how the consequences on price, value, rent\hyp{}seeking, etc. are created by the relationship between laws and institutions (like property) and individual meaning\hyp{}making frames of reference. 


It's only now that we have enough understanding of human beings to even begin defining a general theory of economies.\index{general theory of economies} Up until now, all we have been able to do is look at various alternate Cartesian one\hyp{}dimensional approximations, so a utility theory of value and a labour theory of value are seen as competing alternatives, rather than two complementary axes in a curved decision spacetime.


Physicists \index{physicists}have a very simple model, the Ising model, that will be useful for simple decisions like buy / not buy. In it every node is either buy or not buy. In the absence of any effect from the $\mathbf{T}$ side (e.g., a meme), the decision to buy or not to buy is completely arbitrary, and  $\mathbf{G}$ has a potentially high curvature from node to node. As soon as there is some effect in  $\mathbf{T}$, this smooths the curvature, aligning decisions. Akin to a magnetic field that makes magnets all point North, the meme makes the decision point to buy, or not buy, as the case may be. The Ising model may allow us to calculate in all situations, from equilibrium to complete disequilibrium in times of rapid change, (or as physicists would say, a phase transition,) where buy or not buy becomes sensitive to slight changes in the meme, or even Cartesian rationality.




\section*{Freedom and property in a general theory}
The more freedom\index{freedom}  that an individual has to move exactly along the curve in decision spacetime, the better they are able to choose according to the decision frame of reference at exactly the point in decision spacetime they are naturally at. This is very much the same as saying that the Earth will continue to move in a straight line through curved spacetime around the sun until acted on by a force, such as a moon\hyp{}sized object striking it. 


Equally, an individual who is free to follow their own natural straight line path through time, to faithfully follow the curve their changing meaning\hyp{}making gives them, will always make a rational decision that is right for them. However, if they are not free to choose, and constrained in some way by being property or exposed to some force, they will be forced to make a ‘wrong’ decision.


\section*{Equilibrium in a general theory}
\index{equilibrium!general theory|(}
The general theory of economies \index{general theory of economies} embraces different types of equilibrium and non\hyp{}equilibrium. You can have an equilibrium in information, an equilibrium in flows of value, and much more.


Consider knowledge, and the non\hyp{}equilibrium path between equilibria. Imagine that there is a crash coming, as there was in the mid\hyp{}2000s. If no one has any information about the impending crash, we have a local equilibrium in knowledge and in meaning\hyp{}making. As soon as the first economist predicts the impending crash, we no longer have an equilibrium in knowledge.


This economist now begins talking to people about the impending crash. These people start talking to other people, and sooner or later the press publishes it broadly. Each person who hears this will move to a different point in decision spacetime, depending on whether they believe the prediction, on their hardwired nature, and on all of their meaning\hyp{}making stories. Some traders will believe the prediction, others won’t. Each will use their own meaning\hyp{}making stories to generate decisions about buying and selling.


At every step in the journey, each change in knowledge for each person moves them to a different point in decision spacetime and changes the nature and probability of the impending crash. Possibly the probability will reduce to zero, or just reduced in severity to a negligible dip.


The general theory of economies gives us the tools needed to recursively iterate along the path. We can transport the frame of reference \index{frames of reference} for a rational decision in the starting equilibrium, where no one knows, through the point where one economist predicts, to the point where a small number of people know, and the economist then recalculates their prediction based on those people knowing. They then disseminate that prediction more broadly, recalculate the prediction based on that number of people knowing and each of them talking to each other and spreading their own personal reality of the probability they see of the crash happening. And ad infinitum until we reach the end equilibrium, where everyone knows, and everyone knowing has been incorporated back into an adjusted prediction of the crash, infinitely many times, to reach a new equilibrium.


The general theory of economies allows us to do this path\hyp{}dependent iteration, and even take the average of the integral of all possible paths, and all the other intriguing things that physicists do to construct theories that predict the non\hyp{}equilibrium physics we need to have smartphones.


At the same time, there may always be types of non\hyp{}equilibrium on the $\mathbf{T}$ side, in the concentration and flow of things of value, tangible and intangible. There will always be concentrations and fluctuations. 


Equilibrium is a paradox at the heart of a current concept of business, and the economy: a business, and an economy, is healthy if and only if it is growing. But by definition something that is growing, where interest rates are not zero, is not in a complete equilibrium. And since money is built on positive interest bank debt, a zero interest equilibrium economy based only on money is unstable\cite{graeber-debt} and so at best a momentary equilibrium. \index{non\hyp{}equilibrium} 


In a way, this is much like the dynamic interplay in our atmosphere. We have repeating weather patterns, which over large enough scales we call climate: from microclimates through to our global climate. We also have weather that, in some mountainous regions, can change from one extreme to another in minutes. 


There is no equilibrium in the atmosphere; at best, we have approximate local equilibria.


There is a conflation, in much of modern economics\index{economics}, of equilibrium as a useful but limited approximation to make certain kinds of calculations possible, and the economy actually being in equilibrium; and even worse, that equilibrium is something to strive towards. In this general theory of economies, it's clear that equilibrium is no more desired a state than non\hyp{}equilibrium.\index{non\hyp{}equilibrium}  There is no benefit steering an economy towards equilibrium.


Rather, the general theory of economies shows how well, and when, equilibrium is a useful approximation, over how large a region in decision spacetime, and how an economy functions, regardless of how deep the economy is in non\hyp{}equilibrium.


In physics \index{physics} you can only talk with confidence about a calculation based on an approximation if you have understood sufficiently well what is lost from the actual physics of the world by making that simplification, and if you can demonstrate that what is lost makes a negligible difference to your prediction of what will happen. 


For example, the assumption that the air is in equilibrium is invalid at microscopic scales. The blue sky you see when you look up on a sunny day is created by the microscopic density fluctuations of the air scattering blue light far more strongly than red light. You would not have a blue sky if the air was in perfect equilibrium at all scales.


Because a general theory of economies\index{general theory of economies} includes all of nature, it includes the notion that an economy is only in equilibrium if all of nature is in equilibrium in its interaction across time with the rest of the economy. Anything done in one part to pull manufacturing and trade towards equilibrium, but in a way that takes our natural environment further from equilibrium, takes the economy as a whole further from equilibrium. \index{equilibrium!general theory|)}
\section*{General theory of organisations}
\label{section:general-theory-organisations}
\index{organisation!general theory|(} 


A general theory of economies must include a description of business and other organisations, just as the general theory of relativity\index{general relativity}  stretches from extremely small to large mass objects. We need a general theory of organisations that scales seamlessly from one individual organising themselves and their tasks, all the way through to global ecosystems of organisations collaborating with each other as an entire ecosystem. It must be seamlessly part of the general theory of economies.


Decision spacetime and meaning\hyp{}making stories link the day\hyp{}to\hyp{}day practice of business and the general theory of economies.


As Peter Drucker\index{Drucker, Peter}  says, an executive is a production line, with decisions as the output. An executive board is a decision factory. A business then delivers results by adding human energies to the decisions, turning them to action.


An organisation\index{organisation} is a decision\hyp{}making entity that bundles together multiple individual decision\hyp{}making entities. A bit like the universe:  you or I are specks of dust moving through decision spacetime, an organisation is a planet formed out of megatonnes of dust moving as one, and a multinational is a galaxy.


So a general theory of economies\index{general theory of economies}  includes all aspects of organisations and defines an organisation as its decisions across successive points in decision spacetime, the actions taken to execute those decisions with excellence, and the results achieved. All of this as a bundle spanning whatever region of spacetime the organisation occupies for its existence. In time, this is from first startup idea to final death; in physical space, however big an area of the Earth it impacts; and in social space, all the people it impacts.


This scales seamlessly from one individual up to our entire global economy.


Calling this subset of \emph{the general theory of organisations}, it then states that your organisation occupies a certain amount of decision spacetime, with a curvature created by all the organisation’s physical things and meaning\hyp{}making stories, from each individual's personal meaning\hyp{}making stories through all scales up to those of the entire organisation; and those of all other entities sufficiently big and close to shape your organisation’s decision spacetime.


In a general theory of organisations, people and organisations will move in straight lines along their inherent ‘free’ curved decision spacetime, \index{spacetime} unless prevented by an external force. We want to move in a straight line, and can only do so if we have the freedom to. 


If we are in some sense lacking freedom, \index{freedom} if someone has put walls across our natural path through spacetime that we keep crashing into and bouncing off, then we will move contrary to our nature. That will cost us energy,\index{energy}  and may well reduce the capacity of individuals and businesses or any other organisation to take the right decision.


We now have a way of integrating everything that impacts how organisations take decisions. Anticipating the four integral quadrants described in Section~\ref{section:integral-organisation}, everything is either an existing structure and force, in the lower right quadrant, or a meaning\hyp{}making story in the individual upper left or organisational lower left quadrants. All on the $\textbf{T}$ side creates the shape on the $\textbf{G}$ side of decisions, meaning, and perceived value.


So if the outcomes of decisions are no longer what we need them to be, we need to identify everything on the $\textbf{T}$ side that is currently shaping the $\textbf{G}$ decision spacetime. 


Doing this via regulations is clumsy, because it doesn't change the underlying natural meaning\hyp{}making, nor the natural decision; instead, regulation acts as a force imposed. Of course, to some extent this does change decision spacetime, because meaning\hyp{}making now includes the meaning attributed to obeying or disobeying regulations. 


The rise in regulation; in the amount needed by the welfare state; failing banks, and those banks too big to fail; the rise in the precariat; wherever you look you see more and bigger ad\hyp{}hoc forces introduced to stop our current paradigm from collapsing. The kind of ad\hyp{}hoc introduction of patches that, in physics, signal clearly that the theory is no theory. Just a patchwork of approximations to a theory.


Iva Vurdelja, a lecturer in Marquette University Graduate School of Management, and at Loyola University, uses the thought forms of Table~\ref{table:28DTF} to guide her students to deeper insights into how patching our current system cannot deliver what we need. CSR, for example, is a patch on a fundamentally broken system. \index{Vurdelja, Iva} 


So we are getting a glimmer of how we may be able to construct one single general theory that spans everything, from a single individual deciding in the supermarket whether to buy Ariel/Tide or Persil for their laundry, through the global economy and the impact of all our global bodies on our global economy, and beyond to nature's economy.


Instead of working through clumsy applications of increasingly ad\hyp{}hoc forces of regulation, we now can work with the essence of what has enabled human beings to thrive. Our capacity for meaning\hyp{}making. Which includes implicitly some kinds of regulation, but now in complementarity with other sources shaping $\mathbf{G}$.


We are always, individually and collectively, writing and rewriting our meaning\hyp{}making stories. 


The emerging climate emergency\index{climate!emergency}  is driving a rapid rewriting. By making this process explicit, and by putting in only those structures, processes, and relationships needed for our meaning\hyp{}making stories to get successively closer to the actual world we live in, individuals and business will easily take the decisions that are right for life on earth to thrive because they will be the natural decisions to take, not forced decisions from regulation. They will be decisions taken because \emph{that’s just who I am, and what I love doing.}


This will also lead to far more multisolving\cite{sawin-multisolving}\index{multisolving} (Section~\ref{section:fairshares-commons}), the best strategy we have to solve the multiplicity of problems causing our climate emergency, and all our other injustices.


The role of decisions on organisation design and operation, and hence the final consequences of an organisation design and how it is operated, can now be integrated into one general theory. 


Whichever individual or team has decision authority over the organisation design takes that decision according to the local shape in their $\textbf{G}$ space, which is their individual and collective meaning\hyp{}making stories, along with the bigger meaning\hyp{}making stories of the different cultures and worldviews in which they are embedded. They then create structures and processes in $\textbf{T}$ that modify the meaning\hyp{}making stories of everyone else in the organisation and act as forces preventing people in the organisation from taking their natural decision.


This shows how meaning\hyp{}making determines the organisation design, which then both act as a force on everyone else; and how this force may then modify the meaning\hyp{}making stories of some in the organisation. Others may choose to not adapt their meaning\hyp{}making stories. This may be exactly the right choice, or not, for each specific individual.


This also shows why there cannot ever be one single approach to organisation design, best in all contexts, for all worldviews, and for all individuals. Why the best approach is a do\hyp{}it\hyp{}yourself approach, drawing from a multiplicity of good practices, optimised for the here and now of any specific organisation’s drivers.


Since this general theory of economies\index{general theory of economies}  spans all scales, from each individual through to our global economy, we may well only complete it once we have integrated everything that makes each of us human beings into it. This includes integrating all of our sciences: biological, neurobiological, psychological, social, anything that plays a role in our lives.


It may well be decades before we sufficiently grasp how everything is interrelated and how all these are nested, interlocking, open systems in constant transformation and movement. 


How well we can shift from our current closed disciplines into such an open systems approach to each one depends on how fluid we are in all 28 thought forms \index{thought forms (28)} of Chapter~\ref{chapter:who-am-i-sense}, and how well we can surface and adapt the meaning\hyp{}making stories defining our self\hyp{}identity (Chapter~\ref{chapter:who-am-i-meaning}).\index{organisation!general theory} 




\section*{Building and scientifically testing a general theory of economies}
\index{general theory of economies} 
This is the hard part! If it had been easy, we would already have such a theory.


This general theory of economies is inherently pluralist.\index{pluralism}  It automatically includes political economy, which used to be economics until the mid-19th century. It includes the changes in decision spacetime through the 19th century, as neoclassical economics divorced itself from political economy.


So the builders and testers can only be fully multidisciplinary teams, filled with generative tensions. This will not be easy for anyone involved, because their source of identity and self\hyp{}worth will initially be strongly challenged; and so all the team members must master the adaptive way of Part~\ref{part:you} or similar, and be well skilled in a wide range of other disciplines such as yoga, meditation, and more. And, of course, where everyone fully grasps the distinction between science and scientism (Section~\ref{section:cargo-cults}) so that they engage in the brutal rigour needed.


We will have two complementary paths to building and testing such a theory. Inside\hyp{}out and outside\hyp{}in. Inside\hyp{}out, we begin with hypotheses about each individual’s reality, put them all together in a computer simulation, and then predict outcomes at all scales, from individual to global. Outside\hyp{}in, we look at the decisions first, at all scales, \index{scaling} and use those to infer the reality within which such a decision is rational. 


Both together give us a way of testing our hypotheses scientifically, iteratively leading to a theory. 


This is much like the way physics\index{physics}  works, with teams of theorists and experimentalists in a brutal, rigorous, collaborative tension to uncover what we can reliably say about how the world works. 


\begin{quote}
Well\hyp{}regarded future economists\index{economists} will, first and foremost, have a thorough understanding of adult development, a very strong understanding and connection with society as it is; and they will be adult development practitioners too. 
\end{quote}


Those eminent physicists and philosophers Queen put it so well when they sang \emph{Is this the real life? Is it just fantasy?} (Brian May has an excellent grasp of physics; in fact he put his PhD in astrophysics on hold in 1974 for Queen, completing it in 2008.)


This book is all about the difference between reality\index{reality}  and actuality,\index{actuality}  how the reality you experience is always the best approximate model of actuality that you have available to you at the time, somewhere between fantasy and actuality. You are always doing the best you can within the reality you are experiencing in the moment.


So I have no doubt that every economist was doing the very best they could in their momentary reality; according to the frame of reference they had available to use, at the point in decision spacetime they were at. Just as Newton\index{Newton, Isaac}  could never have created anything other than his description of gravity, because he didn’t have available to him any geometry that could possibly have described the motion of the planets and stars and apples, so too were the originators of today's economics doing the best they could.


However, we are now in a global society with a global interdependency across all scales of local and global economies, and we have already seriously overshot the capacity of our natural environment to support human life in the near and long term future. We know enough to know that sticking dogmatically to what are, at best, local approximations, is now threatening us all.


To construct a general theory of economies, we must identify everything necessary to generate a valid understanding of our economies that is capable of making predictions we can reliably use to navigate humanity through the emergencies of the coming decades.


We need to understand what all the physical and nonphysical things in an economy are, including those that are metaphorically equivalent to the mass, energy, momentum, and stress of general relativity's stress\hyp{}energy tensor $T_{\mu\nu}$. We need to understand what the full multidimensional space of economies is, equivalent to Einstein's tensor \index{Keynes, J.M} $G_{\mu\nu}$.\index{Einstein tensor} 


This will confront some economists,\index{economists}  and many organisations developing or putting economic thinking into practice, with an adaptive challenge. Most experts and leaders only learn how to rise to technical challenges, but not adaptive ones. These need the tools described in Chapters~\ref{chapter:who-am-i-base} to~\ref{chapter:who-am-i-one}. They are amongst the most powerful tools available to recognise your meaning\hyp{}making stories, how they create the reality you experience, and then coordinate across all of your inner tensions and drives as you iteratively bring your meaning\hyp{}making stories into ever greater alignment with actuality. Which is what you need to rise to an adaptive challenge.


The adaptive challenge is present for anyone whose source of self\hyp{}identity and meaning lies in their work or discipline. Many neoclassical economists identify themselves with neoclassical economics, which means that anything pointing at a fundamental change in economics is an existential threat to their self\hyp{}identity. Identifying where neoclassical economics is wrong, in order to identify what works better, is an adaptive challenge.\index{economists!neoclassical} 


Contrast this with physicists,\index{physicists}  who love proving their colleagues wrong. Those who source their identity from their discipline source it from the bedrock of science: science is falsifying, not knowing. The only reason why everybody who comes up with another claim that Einstein\index{Einstein, Albert}  is wrong is treated with caution is that, after a century of falsification attempts, no one has managed to do better than Einstein in the areas where general relativity works. General relativity\index{general relativity}  occupies the pedestal ‘theory’, rather than ‘approximate model’, because it is the simplest description found \emph{so far}, across the largest swathe of actuality, of what we can \emph{say}.


Some cosmologists are working as hard as they can to prove general relativity wrong because we know that the universe is doing things that look berserk within general relativity. Physicists know that general relativity cannot be the full theory, because it breaks down right at the beginning of the universe (infinite curvature in $\mathbf{G}$), and looks inelegant at the current rate of the universe’s expansion. However, the two alternative categories: modifying general relativity; or the multiverse; are widely deemed inelegant and distasteful. (Beauty has so far proven a reliable guide to tell the difference between a viable theory and a model in physics.)


So physicists\index{physicists}  know that there is another breakthrough coming. Space may not survive as something independent, time may not survive; both may even be emergent phenomena coming out of something much deeper. Physicists\index{physicists}  all know that to survive, let alone thrive, they must be open to go where the theory takes them, not clinging to an old idea, risking being steamrollered flat as the new idea comes in.


I (Jack and Graham) really hope that this appendix can play a role in catalysing the shift we need in economics; and that many others will take on the task of developing and testing a general theory of economies. We intend our forthcoming book to cover what emerges. 


As a last thought, maybe what general relativity says about the apple that fell on Newton's head (during the plague pandemic that hammered society then) can help us. It makes clear that gravity did not pull the apple down onto Newton's head, rather the Earth had been pushing Newton\index{Newton, Isaac}  and the apple tree upwards until, at some point, the stalk became too weak to continue pulling the apple up, and broke. At that point, the Earth pushed Newton's head into the (relatively) stationary apple.