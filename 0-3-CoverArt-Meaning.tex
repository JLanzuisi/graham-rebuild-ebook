\chapter{The cover tells the story}
\vspace{-2cm}
\begin{center} 
by Nikyta Guleria
\end{center}
\vspace{2cm}


I designed the cover to tell the story of the book, a story Graham has been telling since long before I began working with him, six years ago. A story that still gives me hope, a story of how we can realistically build a new world today, a practical world that works for everyone and all life on spaceship Earth. 


A new world filled with constant, sustainable, even regenerative motion, like the waves and tides of our oceans and atmosphere that we can see every day; the seasons, that we see over months; and the waves shaping our landscape, that we can only see in geological records. The constant regenerative motion shaping everything, connecting everything, moving energy and resources to where life needs it. 


So the image of the new world is dominated by colourful waves representing building blocks that are still to be defined; hope, flow, celebrations, the visible and the hidden, opportunities and threats, emotions, adventures, successes, failures, and everything in between and neither; and much more. They represent the opportunity we have \emph{right now} to ride the tide Brutus points at in Shakespeare’s Julius Caesar: \begin{quote}There is a tide in the affairs of men \newline Which, taken at the flood, leads on to fortune.\end{quote}


This new world, though, doesn’t exist yet. We cannot yet know what it will look like, nor can we know all the elements we will need to build it. But we do know the regenerative, life-giving essence it must have, and that to see it we need to look through new lenses, just as Picasso and the other founders of cubism did. Look for the eyes and faces in the new world, representing the capacity we most need now: to see what others cannot yet see, to imagine possibilities others cannot yet imagine, and to build what others cannot yet build. 


As we rebuild, we also need to dismantle, step by step, the grey old world dominating the lives of most of us still. A world that fails almost everyone on the planet today; and even worse, sets up the next generations for misery, by wasting,  in a few decades, the critical resources our descendants ought to still have access to. A world that is holding us, and the next generations, back from thriving.


This world’s systems and structures systemically exclude, or discriminate against, people who would love to contribute to a thriving society; sometimes driving them into homelessness or worse. They are psychologically dangerous, and even physically dangerous; workplace practices and cultures that, at best, waste lives and our potential to create a better world, and at worst end in burn-out, depression, and even death. 


Step by step we need to dismantle our linear use\hyp{}and\hyp{}dump mentality and practices (depicted in piles of garbage and demolitions), dismantle everything putting toxic waste into nature and our communities (oil well, factory chimneys and fumes) replacing them with circular, fully regenerative practices. 


We live in a fading,  graying old world, on a wall blocking us from seeing, let alone breaking through to, the vibrant, regenerative world we ought all to thrive in. 


We can rebuild everything we have today, so that we are all freed up to be our best, unique selves, and all thrive together.