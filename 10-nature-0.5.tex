\chapter{You are your nature}
\addcontentsline{toc}{chapterdescription}{Your hard-wired nature is the basis for your thoughts and meaning\hyp{}making. Increase your subtlety working with, not against your nature: your personal energy economy, needs, feelings (vs. judgements dressed up as feelings), your cognitive biases, and your relationship with money and power.}
%\addcontentsline{toc}{chapterdescription}{\pagebreak}
\label{chapter:who-am-i-nature}


\begin{chapterquotation}
The meaning of life is to find your gift. The purpose of life is to give it away. \\
\raggedleft\textemdash Pablo Picasso\index{Picasso, Pablo} 


\centering
God grant me the serenity to accept the things I cannot change, the courage to change the things I can, and the wisdom to know the difference. \\
\raggedleft\textemdash Reinhold Niebuhr\label{quote:niebuhr}
\end{chapterquotation}


\section{Your nature just is; or is it?}
The serenity prayer of the American theologian Reinhold Niebuhr (1892 – 1971) \index{Niebuhr, Reinhold} captures the nebulous balancing act of getting to know all the elements of who you are and growing yourself into who you can become.


The first step in your process of experiencing your reality depends on your hardwiring. You cannot change most of this, and those small bits that you may be able to change may take an entire lifetime for very little change.


Unchangeable aspects of you are all those directly anchored in your genetics; other aspects of you affected by your epigenetics may or may not be changeable. 


You have limited room to change anything anchored in your physical body either. Especially your physical brain. For example, the psychiatrist and nuclear brain imaging specialist Daniel Amen\index{Amen, Daniel} speaks\footnote{See his TEDxOrangeCoast talk.} and writes~\cite{amen-neuroscience} of how your physical brain determines the reality you experience. 


If you are dealing with OCD, depression, anger, panic attacks, etc.\textemdash much that is usually dealt with via psychotherapy or medication\textemdash it may well be your brain’s wiring that is the cause. All your behaviour is merely the symptom of the cause. 


Which is why behavioural interventions can easily be cruel, because the behavioural symptoms can never be wished away. There will always be some symptom of the root cause. For example, you may have had an innocuous fall when young, one that no one even remembers, that caused physical trauma to your brain, which now shapes the meaning making you’re capable of, causing your behaviour.


Two people with identical behaviours, or sharing the same visible characteristics, may have fundamentally different brain activities behind them. And so they require fundamentally different kinds of coaching, scaffolding at work, etc. to change their behaviours.


Be very aware of this with anyone trying to help you from a purely behavioural paradigm; especially if you are trying to help yourself by bullying yourself into new behaviours! 


We recommend you focus yourself on what you absolutely can do, which is to get to know the difference between your nature and your meaning\hyp{}making stories, and then modify your meaning\hyp{}making stories to work \emph{with} your nature in an ever more subtle, compassionate, and productive way. And check out emerging approaches, e.g., in neuroscience on rehabilitating your brain after damage.


If your nature has given you exceptional natural talents in responding immediately and creatively to unplannable crises, don't waste your time in trying to become good at planning predictable processes. If your nature\index{nature}  has given you exceptional talent and energy at connecting with people, don't try to become an analyst sitting alone with vast tables of data.
\section{Your personal energy economy}
\label{section:PEE}\index{economy!personal energy} \index{economy!personal energy|(}
There are some activities that you will find naturally give you energy,\index{energy}  and some activities that you will find invariably drain you of energy. Look at these carefully over the course of at least a week, and you will get useful clues to what I like to call your personal energy economy.


Manage your personal energy economy scrupulously, and you will find that you are happier and more productive. I have found the following approach, based on Jung's work\index{Jung, Carl}, very effective because it is so simple that everyone can remember it. This is a language about how to maximise flow; to understand yourself and each other through talking in a non-judgemental way. This is not at all intended to be an accurate measurement of your nature!


To get a good enough idea of your natural energy economy, think of the different kinds of tasks that give you or take your energy across the following four categories:


\begin{description}
\item[Exploring] If you get energy from always doing new things, from creating on a blank sheet of paper, you are naturally an explorer. Most likely you lose energy if you need to use a process more than once or twice, or go on holiday to the same destination for the third time. You have a high need for significance, to see the big picture, and want your work to matter.


\item[Vibrant] If you get energy\index{energy} from meeting people, from talking about what excites you and what excites them, and you love seeing people getting energised with shiny eyes after talking to you, you most likely have a vibrant energy. Most likely you lose energy if you need to pay careful attention to detail or draw conclusions from deep analysis. You have a high need for movement, to see growth, and want diversity in experience.


\item[Context] If you get energy when a plan of action is thought through in advance, so that everyone can see what needs to be done, when, by whom, then you are likely to be naturally high on context. You have a high need for connection, to be grounded, and want to understand everything that is and needs to happen.


\item[Systems] If you get energy by spending time alone, working with facts and data to get to the right answer, you'll score high on systems in your natural energy profile. You likely lose energy spending lots of time with large groups of highly diverse people, especially if you are required to move them using your charisma and working intuitively with what's happening in the moment. You have a high need for certainty, clarity on what needs doing, and want to avoid mistakes.
\end{description}


Caution: never think of yourself or others with sentences like \emph{I \emph{am} an Explorer / Vibrant / Context / Systems person / type}. Only use these as a language to identify clearly and talk about the characteristics of activities that give you energy, compared to those that you need to pump energy into. Read more on why most type indicators are seldom helpful, and can be harmful, in section~\ref{section:against-typologies}. 


Any well-functioning organisation needs people across all four energy dimensions. None of them is inherently better or worse than any others. Whatever your natural energy profile is, if you try to be someone else, you will end up less happy and less productive. If you mistakenly see your natural energy economy as stories that can be changed, you are likely to end up frustrated and disillusioned because you haven't changed, despite all your effort.


As with any economy, you can only be successful when you work with, not against, its true nature.


In Table~\ref{table:simple-PEE} is a simple way of getting an initial idea of your natural energy profile. Take this just as a starting point, and each time you find yourself doing something that gives you energy, or something that drains your energy, add that data to this profile to get a better handle on your natural personal energy economy. 


\begin{table}[htbp]
        \centering
        \begin{tabular}{ L{0.70\textwidth}  L{0.10\textwidth}}
                \toprule
                \textbf{Explorer} \hfill \textit{total:}  &  \\ 
                Vision, Ideas, Big picture    &  \\ 
                Dynamic, bubbly, intuitive    &  \\ 
                You get energy starting things    &  \\ 
                Decision making: Intuition, reflection first    &  \\ 
                \addlinespace[.5em]
%         
                \textbf{Vibrancy} \hfill \textit{total:} & \\ 
                Variety, Growth, Participation     &  \\ 
                Passionate, extroverted, excited     &  \\ 
                Great at meeting new people    &  \\ 
                Decisions making: through dialogue with others    &  \\ 
                \addlinespace[.5em]        
%         
                \textbf{Context} \hfill \textit{total:} & \\ 
                Timing, Delivery, Perception    &  \\ 
                Compassionate, team player, sensory    &  \\ 
                You get energy by getting things done with others    &  \\ 
                Decision making: guided by one of more senses    &  \\ 
                \addlinespace[.5em]
%                 
                \textbf{Systems} \hfill \textit{total:} & \\ 
                Process, Data, Quality    &  \\ 
                Detailed, orderly, introverted    &  \\ 
                You get energy completing things   &  \\ 
                Decision making: Analytical, data first    &  \\
                \bottomrule
        \end{tabular} 
\caption[Personal energy economy]{Quick test of your natural personal energy economy. Distribute 100 points across all 16 lines under the headings, in the way that you think best matches the kinds of activities that give or drain you of energy. Give more points to activities that give you energy, and fewer points to energies that don't, and no points to activities that drain you of energy. Ignore any effect from what you have learnt to do well; focus on what naturally gives you energy. Then add up your points in the four lines under each energy to give your final score for that natural energy type.}
\label{table:simple-PEE}
\end{table}


The higher your score in the four energies, the more that these activities are natural energy givers for you. You could call your highest energy your sun energy, because the more time you spend doing activities of this type, the more energy pours into you. (Double check that you have avoided giving high points to something you have had to learn to do well and efficiently, but that drains your energy.)


If you have one (or maybe two) that are significantly lower than the others, and you know that spending more than 10\% of your time doing these kinds of activity leaves you drained, you can call this your battery energy.


The other one or two neither give you energy nor drain you of energy, and we call these your moon energies.


Some people score very highly on one and very low on another; and some score more or less the same across all four types of energies. None of these is better or worse. It just means that your natural energy is different to somebody else's, and your task from now on is to continuously work on putting yourself into environments where you can play to the strengths of your natural energies.


For example, I (Graham) am naturally very high on explorer energy, extremely low on systems, and moderately low on context. I got huge amounts of energy in the very early activities of creating this book, but the final activities of proofreading, polishing, and checking all the resources exhausts me. I quite gladly spent a month's holiday doing the early exploration and creation of the book, and returned feeling refreshed and energised. As we moved into the book’s later stage, I relied more heavily on Jack having a higher systems and context natural energy.


Take this to heart. Focus on maximising your natural energy, and finding other people whose energy\index{energy}  complements yours to take care of the activities that drain you. They will be very grateful because they will find that precisely these activities that you get energy from other ones that drain them.\index{economy!personal energy|)} 
\section{Your needs}
\label{section:needs}\index{needs|(} 
Your natural human needs are linked to your natural energies. In Table~\ref{table:needs-top} you’ll find a list of the needs that show up as most important to people across a wide range of studies\footnote{The needs, feelings, judgements etc. tables here have been compiled from a wide range of sources over the past decade, many from the NVC community\cite{CNVC, rosenberg-nvc, little-total-honesty} started by Marshall Rosenberg. Go there for more good stuff.}. 


What I find striking, looking at these needs, is how poorly designed our workplaces are at meeting them. Altruism,\index{altruism} for example, has been clearly recognised as a fundamental human need. It's a superpower of humanity, enabling us to thrive better by bonding around shared stories. Yet how many workplaces lack any systematised opportunity to thrive by meeting people’s need for altruism?


Perhaps the most important human need is the need for hope.\index{hope}  How well does our economy\index{economy} do at meeting your need for hope? How well does your organisation do at meeting the everyone’s need for hope?


\begin{table}[ht]
        \centering
        \begin{tabular}{ R{0.30\textwidth} L{0.30\textwidth}}
                \toprule
                \textbf{Primary general needs} & \textbf{Primary Work Needs } \\
                \addlinespace[.5em]
                Hope & \\
                Health & Purpose \\
                Security & Autonomy \\
                Community & Mastery \\
                Fairness & Relationships \\
                Bonding & \\
                Altruism & \\
                Playfulness & \\
                Celebration & \\
                \bottomrule
        \end{tabular} 
        \caption[Top human needs]{In this table we list the top human needs identified in biological and social science research. The work needs are also general needs, but are met through some kind of work.}
        \label{table:needs-top}
\end{table}


In Table~\ref{table:needs-all} you will find a comprehensive table of human needs. I find this table very useful, and I hope you will find it useful in understanding your hardwiring, and how that shapes the reality you experience. 


It is especially useful to make better sense and meaning of how you and your colleagues show up at work. Almost everyone will agree with you that these are valid human needs, even if they have a different order of importance to you.


Money is nowhere in the table of human needs. 


Money is merely one tool we have at our disposal to meet the needs listed in these tables. And it is seldom the only tool that can meet these needs. As this book shows, we have removed much of the power that we have to use business to solve the global challenges we're facing, because we've designed our businesses and our society so heavily around the one tool called money\index{money}  that many mistake it for a need.


We all differ from each other in some way in how important each need is to us. Some needs in the table will have such a low importance to you that whether it is met or not makes no difference to you. Other needs will have such a high importance to you that, if those needs are not met, you will experience extremely strong feelings. Sufficiently strong that you will either be almost irresistibly driven to act to get your need met, or will need all your strength to not act.


\begin{longstoryblock}
One of my (Graham) highest needs is freedom. This is very closely related to my natural explorer energy profile. If I do not have enough freedom,\index{freedom}  I will break out sooner rather than later, whatever the cost, to meet my need for freedom. Over the course of my life I have become a little better at enduring situations where my need for freedom is not met, but I will never ever be able to do that naturally. It will always cost me energy.\index{energy} 


And one of my needs (Jack) is self-expression. My self has always been non-binary and amorphous. When I was younger, I cursed God for my long sleepless nights, nights that seemed like days, days that seemed like months, for placing me in \emph{“the dark woods, the right road lost”}\cite{dante-inferno}, not knowing who I was or who I was supposed to be. But now I’m thankful for being non-binary, for I marvel at life’s wonderful diversity, I listen to the rain, I understand what it means be male and female, to be accepted, rejected and hated, to be despaired and fulfilled.  To fulfill my life’s potential, to be a full human being and to help others do so, it is important for me to self-express, whatever my self is.
\end{longstoryblock}


Many needs are strongly driven by your hardwired nature, and failing to get these needs met sufficiently well will at best leave you less productive and happy than you could be, and at worst will take you into dire consequences like burnout.


To identify which needs are most important to you, keep track over the course of a few weeks: 1) when you experience feelings in the table of positive feelings (Table~\ref{table:feelings-positive}), these are reliable indicators that one of your or more of your important needs is being met; and, 2) when you experience one of the feelings in the table of negative feelings (Table~\ref{table:feelings-negative}); again, this is a reliable indicator that one or more of your important needs is not being met. In both cases, note down which feelings you experience, and which needs your feelings are telling you about.


Keep doing this and over time you will build up a good understanding of which of your needs are really important to you. Some are part of your hardwiring, and will remain important throughout your life. These needs, whilst part of your hardwiring, are age-dependent, and your personal ranking and weighting will change throughout the course of your life.


Even better, some needs are primarily anchored in your stories,\index{stories}  and your stage of development.\index{meaning-making!six stages of}  These will change as you grow.


\begin{small}
        \begin{longtable}{L{0.33\textwidth}  L{0.33\textwidth}  L{0.22\textwidth} }
                \caption{Comprehensive table of needs. \label{table:needs-all}} \\[-.5em]
                \toprule
                achievement & connection & mutuality \\
                best use of energy and time & acceptance & nurturing \\
                knowledge of self & affection & respect \\
                self-development & appreciation & self-respect \\
                spirituality & belonging & safety\\
                & company & security\\
                autonomy & cooperation & stability\\
                choice & communication & support \\
                choose dreams & closeness & to know \\
                choose goals & community & to be known \\
                choose means to achieve goals & companionship & to see \\
                freedom & compassion & to be seen \\
                independence & consideration & to be heard \\
                space & consistency & to understand \\
                spontaneity & empathy & be understood \\
                to listen & inclusion & trust \\
                & interdependence & warmth \\
                meaning & intimacy & honesty \\
                awareness & to love & authenticity\\
                challenge & to be loved & integrity \\
                clarity & & presence \\
                co-create & celebration & physical wellbeing \\
                competence & appreciate life & air \\
                consciousness & appreciate accomplishments & breath \\
                contribution & celebration of life & comfort \\
                creativity & commemorate grief, loss & emotional safety\\
                discovery & share joy & exercise\\
                contribute to others’ well-being & share pain & light \\
                efficacy & & material security \\
                effectiveness & peace & movement \\
                growth & beauty & nourishment \\
                hope & communion & physical contact \\
                learning & ease & protection \\
                mourning & equality & reconciliation \\
                participation & harmony & rest/sleep \\
                purpose & inspiration & right temperature \\
                self-expression & order & safety \\
                self-worth & play & sexual expression \\
                stimulation & joy & shelter \\
                to matter & humour & touch\\
                to make a difference & charge one's batteries & water \\
                understanding & fun & \\
                & letting go & \\
                & recreation & \\
                \bottomrule
        \end{longtable}
\end{small}
\index{needs|)} 


\section{Your emotions and feelings}
\index{feelings|(} 
Your emotions and feelings are part of  the data that you use to make sense and meaning. They are hard data, coming from a superbly designed instrument, your mind and body, as a way of evaluating potential opportunities to meet, and potential threats to, your needs. Feelings and emotions are your best guide, telling you what needs are important to you, and if they are met or unmet.


You have a few fundamental feelings; the rest are emotions that you construct from your meaning\hyp{}making. For simplicity I typically call both feelings, but sometimes the distinction is important. You can read more about how you create most of your emotions through your meaning\hyp{}making stories in Lisa Feldman Barrett’s book\cite{barrett-emotions}\index{Barrett, Lisa Feldman}. In short, many of the feelings that many believe are universal actuality, experienced by all the same way, are more each person’s unique inner reality, dependant on your unique life path. 


Unfortunately, we seldom develop the vocabulary to name the precise emotion or feeling we have. If we can't name it, we can't use it for its prime purpose: to identify clearly which specific need is met or not met, and therefore what to do about it. Imagine how different your life would be, and even how different economics would be, if everyone had been taught how to use this at school! 


It’s never too late; start today putting into practice all the words in Tables~\ref{table:feelings-positive} (positive, energising feelings) and~\ref{table:feelings-negative} (negative, energy-draining feelings) below. 




\begin{small}
        \begin{longtable}{ p{0.17\textwidth} p{0.17\textwidth}  p{0.17\textwidth}  p{0.14\textwidth}  p{0.17\textwidth} }
                \caption{Feelings anchored in positivity.}\label{table:feelings-positive}\\[-.5em]
                \toprule
                affectionate & hopeful & gleeful & delighted & composed \\
                compassionate & expectant & intense & glad & centred \\
                friendly & encouraged & invigorated & happy & content \\
                loving & free & keyed up & jubilant & cool \\
                open hearted & glorious & lively & mirthful & expansive \\
                sensitive & optimistic & passionate & merry & fulfilled \\
                sympathetic & confident & perky & overjoyed & zestful\\
                tender & cheerful & surprised & pleased & mellow \\
                warm & empowered & upbeat & tickled & quiet \\
                engaged & open & vibrant & exhilarated & relaxed \\
                absorbed & proud & helpful & blissful & relieved \\
                alert & safe & grateful & breathless & satisfied \\
                curious & secure & appreciative & elated & serene \\
                engrossed & excited & filled & enthralled & still \\
                enchanted & adventurous & gratified & exuberant & tranquil \\
                entranced & amazed & joyous & radiant & trusting \\
                fascinated & animated & moved & rapturous & refreshed \\
                intense & ardent & thankful & splendid & enlivened \\
                interested & aroused & touched & thrilled & nurtured \\
                intrigued & astonished & inspired & ecstatic & peaceful \\
                involved & dazzled & alive & peaceful & pleasant \\
                inquisitive & eager & amazed & at ease & rejuvenated \\
                glowing & ebullient & awed & at peace & renewed \\
                spellbound & effervescent & motivated & calm & rested \\
                stimulated & energetic & wonder & carefree & restored \\
                & enthusiastic & joyful & clear headed & revived \\
                & exultant & amused & comfortable & wide-awake \\
                & giddy & buoyant & complacent & good-humoured \\
                \bottomrule
        \end{longtable}
\end{small}


\begin{small}
        \begin{longtable}{ p{0.15\textwidth} p{0.15\textwidth}  p{0.17\textwidth}  p{0.15\textwidth}  p{0.15\textwidth} }
                \caption{Feelings anchored in negativity.}\label{table:feelings-negative}\\[-.5em]
                \toprule
                afraid & aroused & repulsed & baffled & sad \\
                apprehensive & angry & sceptical & bewildered & blue \\
                concerned & cross & disquiet & dazed & depressed \\
                dread & enraged & agitated & hesitant & dejected \\
                fearful & furious & alarmed & lost & despair \\
                foreboding & hot & discontented & mystified & despondent \\
                frightened & incensed & disconcerted & perplexed & disappointed \\
                mistrustful & indignant & dissatisfied & puzzled & discouraged \\
                on edge & irate & disturbed & torn & disheartened \\
                panicked & livid & ill at ease & disconnected & downcast \\
                petrified & mad & muddled & alienated & downhearted \\
                scared & mean & perturbed & aloof & empty \\
                suspicious & nettled & upset & apathetic & forlorn \\
                terrified & outraged & rattled & bored & gloomy \\
                wary & resentful & restless & cold & guarded \\
                worried & upset & shocked & detached & hopeless \\
                annoyed & aversion & startled & disenchanted & melancholic \\
                aggravated & animosity & surprised & distant & miserable \\
                bitter & appalled & troubled & distracted & morose \\
                dismayed & contempt & turbulent & indifferent & mournful \\
                disgruntled & disgusted & turmoil & numb & solitary \\
                displeased & dislike & uncomfortable & passive & sorrowful \\
                exasperated & bitter & uneasy & preoccupied & sorry \\
                frustrated & hate & unnerved & removed & regretful \\
                impatient & horrible & unsettled & tepid & remorseful \\
                irritated & horrified & upset & unconcerned & unhappy \\
                irked & hostile & confused & uninterested & woeful \\
                pessimistic & repelled & ambivalent & withdrawn & wretched \\
                vexed & embarrassed & bereaved & up tight & ashamed \\
                devastated & vulnerable & chagrined & grief & bowled over \\
                flustered & heartbroken & lonely & guilty & hurt \\
                fragile & mortified & broken hearted & heavy hearted & self-conscious \\
                tense & helpless & fatigue & anxious & in despair \\
                beat & cranky & insecure & burnt out & distressed \\
                leery & dead beat & distraught & reluctant & depleted \\
                edgy & reserved & dull & fidgety & sensitive \\
                exhausted & frazzled & shaky & lethargic & harried \\
                shaken & listless & unsteady & sleepy & irritable \\
                yearning & spiritless & jittery & envious & Tired \\
                nervous & jealous & weary & over-excited & longing \\
                worn out & overwhelmed & nostalgic & pain & restless \\
                pining & agony & stressed out & wistful & anguished \\
                \bottomrule
        \end{longtable}
\end{small}


%\vfill \pagebreak  %make space for table to go onto next page


\index{feelings|)} 
\subsection{Judgements dressed up as feelings}
In normal speech, we all often talk about our judgements\index{judgements|(}  and our feelings\index{feelings} as if they were the same. For example, you might say 


\begin{quote}
I feel insulted.
\end{quote}


That is a judgement, an evaluation, you have made, not a feeling that you have. 


By calling it a feeling, you will find it far harder to recognise which of your stories has led you to evaluate somebody else's words or actions as an insult. Insult is a meaning that your stories are making, that you then attribute to the other person, even when the other person intended to insult you. What better way of winning than refusing to make the meaning of insult that the other person wants you to make, and instead making meaning from compassion for the other person's weakness, or simply laughing out loud because you've made a humorous meaning.


\begin{longstoryblock}
I (Graham) experienced this clearly when a former direct report got into an arm-wrestling match with me. Recognising that they were deliberately trying to provoke me into a fight to prove that I was justifiably their boss, I made meaning of what was happening on that basis. So I did not get angry, did not enter into a fight, and thereby demonstrated paradoxically that I was justifiably their line manager.


It has taken me quite some time to develop this capacity. I don't yet have it always, everywhere, with everyone.
\end{longstoryblock}


The better you get at distinguishing cleanly between true feelings and judgements dressed up as feelings, the better you will be able to know yourself, understand other people, and collaborate with them.


%\vfill \pagebreak  %make space for table to go onto next page


\begin{longtable}{ p{0.22\textwidth} p{0.18\textwidth}  p{0.19\textwidth}  }
        \caption[Judgements dressed up as feelings]{Judgements that can be incorrectly dressed up as feelings. \label{table:judgements}} \\[-.5em]
        \toprule
        \textbf{Accusation} & \textbf{Attack} & \textbf{Domination} \\
        accused & aggressed & beaten\\
        blamed & attacked & boxed in\\
        caught out & crushed & bullied \\
        dragged in & harassed & coerced \\
        hurt & humiliated & cornered \\
        let down & insulted & interrupted \\
        pressured & intimidated & mothered \\
        misunderstood & mistreated & obliged to \\
        neglected & offended & persecuted \\
        overloaded & provoked & put down \\
        & trapped & used  \\
        made to feel guilty & threatened & stifled \\
        \addlinespace[.5em]
        \textbf{Denigration} & \textbf{Deception} & \textbf{Rejection} \\
        abused & betrayed & abandoned \\
        diminished & cheated & excluded \\
        distrusted & exploited & misunderstood \\
        dumb & grassed on & neglected \\
        horrible & hard done by & rejected \\
        patronised & let down & unappreciated \\
        ridiculed & manipulated & unheard \\
        stupid & trapped & unimportant \\
        unworthy & & unseen \\
        sullied & & unsupported \\
        taken for granted & & unwanted  \\
        \bottomrule
\end{longtable} \index{judgements|)} 
\section{Your biases}
\label{section:biases}
\index{biases}
This section on biases is hard to allocate uniquely to any chapter. Your biases are a mix of your hard-wired nature, your meaning\hyp{}making stories, and your fluidity in transformational thought forms.  However, because they are deep and hard to change, I have decided to put them into this chapter.


We all have our own unique mix of biases but there are some biases that we almost all share, because they have a central role to play in being human. Robert Cialdini\index{Cialdini, Robert} has done an excellent job describing these in his book on influence\cite{cialdini-influence}.


In this section I summarise some of the most common biases. Keep an eye on yourself, and check whether one of them is shaping the reality that you experience and leading you to take decisions that are not in your best interests. Learn how to harness your biases to serve your interests, those of people you care for, and the needs of your organisation. Learn how to subtly protect yourself and others from your own biases, and those of others.


Warren Buffett\index{Buffett, Warren} and Charlie Munger, \index{Munger, Charlie} the founders and partners of Berkshire Hathaway,\index{Berkshire Hathaway}  one of the most successful investment companies, attribute their success, in part, to rigorously checking their investment decisions against these common biases.


You can think of these biases as both lenses\index{lens}  that you look through and frames of reference to evaluate against. As lenses, they create the reality that you see or experience. As frames of reference, they decide for you which decision to take. When those decisions have at least some rightness, these biases are helpful to us. When these decisions are harmful, and we often don't even realise that until way too late, sometimes never, our biases are not our friends.


To thrive as human beings we need these biases. They are all the Type~1 thinking of Daniel Kahneman\cite{kahneman-thinking},\index{Kahneman, Daniel}  which gives us quick decisions and cost very little energy or time. For most of human history, and in any technical challenge, this has been enough. However, in today's nebulous world filled with adaptive challenges, we need to be more  than ever before on our guard against our cognitive biases kicking in inappropriately.


One excellent clue that your cognitive biases may be kicking in inappropriately is if something just doesn't feel right in your gut. 


Then stop, and use Type~2 thinking. The other thing to do is to remember all these biases and force yourself in all decisions to check whether one or more of them is active before you take the decision.


\subsection{Hope-seeking bias}
\index{biases!hope}
Hope\index{hope}  is one of our most important needs. So when times are tough, when we cannot see a better future ahead of us, and hope has deserted us, then we are easy pickings for anyone able to sell us hope. Our hope-seeking bias is both what enables us to overcome overwhelming odds, and prevents us from pivoting into a different direction when we ought to.


\begin{longstoryblock}
Over the past 15 years, I (Jack and Graham) have often felt very little hope in me. The hope that I have has kept me going on the one hand, because I've continued to deeply believe in what I'm doing and that what I'm doing is a necessary part of creating a viable regenerative future for us all. And, my hope-seeking bias on the other hand sometimes led me astray, seeking for sources of hope in others. 


This has led me to occasionally spend time with people whose display of energy and confidence fooled me into believing that they knew better than me. Invariably I very quickly discovered that there are no gurus truly worth following. At least, none of those who declare themselves to be gurus and seek followers. I learnt that my bias towards seeking hope was sometimes leading me astray.
\end{longstoryblock}


Jack and I have written this book with the intention of generating and amplifying hope in all of you reading this book. We certainly believe that everything in this book is either what we need to be able to rise to the global challenges facing everyone, or it is at least a useful step towards what we need. We are consciously tapping into your bias to look for reasons to hope in order to accelerate the transformation of business that we believe is necessary.


Equally, your hope bias can easily be tapped into by the unscrupulous seller of expensive fake remedies. If you have lost hope, for example, of ever finding a way out of the pain you are in, anyone with charisma and the right words to trigger feelings of hope can convince you to give them what they ask for. 


Throughout history there have been the sellers of snake oil, offering hope of a cure to those with an incurable disease, tapping into all your biases in order to get you to buy what they're selling. 


In times of crisis, whether it's just your own internal crisis, or the crises everyone on the planet is facing, whenever you no longer feel hope, and find yourself leaping at anything that triggers feelings of hope in you, notice that jump in feeling hopeful and pause for a moment. 


Ask yourself what has actually triggered that feeling of hope;\index{hope} are being manipulated through any biases in this section; and: if all you had were the naked facts, without any of the emotional catalysts used by the seller of hope, would your feeling of hope have jumped just as much?


\subsection{Comparison bias}
\index{biases!comparison}
This is at the heart of so much marketing, and influences many of your purchase decisions. Ever wondered why international coffee chains offer the range of products they do, and manage to sell at the prices they sell at? Comparison bias is deep in almost all of us, and leads us to decide what is the best option based on the comparisons we are making in the moment, not on what is really best for us.


Comparison bias happens most easily in people who have little or no fluidity in recognising the frame of reference they are using, thought form C6.


Comparison bias is kicking in when you automatically decide that the most expensive item must be the best quality, or the best for you.


Comparison bias kicked in every time I (Graham) bought anything expensive, like my motorbike. Eleven years ago I took the decision, over the course of many months, to buy a new motorbike worth €15,000. At the dealer, looking through the list of all the optional extras, I ticked off many of the boxes. Each added €200 here, €300 there. By the time I'd finished, I had committed myself to another few thousand euros.


If I had decided first, without any knowledge of the motorbike’s base price, whether I truly benefited enough from any optional extra for it to be worth the price, there are some that I wouldn’t have chosen. But because €300 looked small to me compared to €15,000, my comparison bias kicked in and took the decision for me.


In business, comparison bias is often called the sunk cost fallacy. You take the decision to keep working on a project, or to keep your startup going, because you look at the incremental cost of doing just the next step compared to the far larger cost that you have already sunk into the project or your startup, instead of comparing it to everything else you could spend that money on now. 


Because it looks smaller, you decide to add that extra bit of money and keep going. 


\subsection{Reciprocation bias}
\label{section:reciprocity-bias}
\index{biases!reciprocation}
Reciprocation bias has enabled human beings to become the dominant life form on the planet because it powers our long-term collaboration. The very first humans thrived because if one hunter brought home an antelope, he would share it with everybody, knowing that at some point in the distant future, when he came back empty-handed, somebody else would share what they had brought home with him.


So most of the time reciprocation bias serves us extremely well.


As our society has grown in complexity, reciprocation bias has evolved with us. Now, we remember people who did us a favour many years ago and continue to honour the obligation to do them a favour.


Reciprocation bias is, even if we know we are being conned into something, almost irresistible. Every time I collected my motorbike from a service, the garage owner would always give me something small that he thought would be useful to me. I knew that these were never of any great cost to him, but reciprocation bias left me feeling obliged to return the favour by remaining a loyal client. I saw my relationship with him to be one of loyalty and obligation, not a mere transactional business relationship.


By getting a free gift, regardless of value, our reciprocation bias kicks in and compels most of us to give back more than the unsolicited gift.


Think about how often you will cross over the road if your radar detects any possibility of a charity collector intending to give you something small in order to manipulate you into giving them what they ask for. (This is why I seldom accept food samples in the supermarket. I know how easily that leads me to buy something I would have never otherwise bought.)


Reciprocation bias is also at the heart of effective negotiation. If one party has made a concession, reciprocation bias creates internal pressure in the other party to offer an equally valuable concession. Again, you can see that without reciprocation bias we would never be able to collaborate, and we would never be able to start up organisations nor function well with colleagues that we have little in common with.


Dealing with reciprocation bias is easy to describe, and very hard to put into practice. First, listen to your gut instinct. If your gut is telling you that somebody has given you something to manipulate you, it might be right. Maybe a narcissist. Then, realise that the enemy is not the other person, but rather your own meaning\hyp{}making story compelling you to reciprocate when given a gift. Because the other person has given you something to manipulate you into giving back, not out of true reciprocal generosity, your story telling you that you have to give back to them is now invalid, at least in this deliberately exploitative context.


Use these situations, and the ground pattern experiments described in Section~\ref{section:ground-pattern}, to create a set of experiences that rewrite your meaning\hyp{}making story to one that now says that you reciprocate only when somebody has given in generous, collaborative, reciprocity, not if somebody has given manipulatively.


\subsection{Commitment bias}
\index{biases!commitment}
Commitment is a central dimension of trust.\index{trust}  If somebody else says that they will do something, and they reliably then do it, you will have a far higher level of trust in them than if they regularly say they will do something, and fail to. So commitment bias is a very necessary and powerful story for humans to live by. We experience internal pressure from our stories to stick with what we have previously committed to, and external pressure from our family and colleagues to stick with what we've committed to. This pressure can become very powerful, including people cutting off friendship and ostracising you from the group.


The problem is, we make commitments now with the best information that we have, and then get trapped by our commitment bias to stick with that commitment even when new information emerges which, had we known it at the beginning, would have led us to making a different commitment. Without commitment bias, we would have no hesitation in making a new choice that was better.


Commitment bias is why companies large and small fail. A certain project becomes the company’s holy grail, or perhaps just one manager’s, and money continues  pouring into it long after there is evidence that it will never pay off.


Commitment bias is why some people continue to invest time and their emotions into friendships that have been one-sided from the start.


\begin{longstoryblock}
After I (Graham) slipped and hit my right hip on a stepping stone, I realised that I was feeling emotionally far worse than the pain in my hip warranted, and asked myself why. I first said, because I'm blaming myself for taking a risk when I can little afford it. Why did I take the risk? Because I was in a hurry to get back. Why was I in a hurry to get back? Because I had taken far longer than planned. Why had I taken far longer than planned? Because I had needed to do some strenuous climbing down a near-vertical drop, after going seriously off-route. Why had I gone off route? Because once I had started going off-route to try to reach a new path by going forwards, my commitment to keep going was stronger than the wise voice in me saying that I ought to turn back to where I was last on a path.
\end{longstoryblock}
\subsection{Consistency bias}
\index{biases!consistency}
Consistency bias is another dimension of trust, and can be friend and foe. Consistency means that you do the same thing each time you're in the same situation. Other people can then trust you to behave in predictable ways in the future, once they realise that you subscribe to the stories of consistency. Consistency bias is another reason why executives can easily lead their companies into collapse. We've always done it this way, and we will continue to do things this way.


As we mentioned in Section~\ref{section:emergent-strategies}, Kodak\index{Kodak}  invented digital photography, but failed to change its focus from chemical to digital as fast as consumers changed their behaviours. Commitment bias and consistency bias together led Kodak to continue to strive towards ever better chemical photographic film long after the electronics world had clearly recognised that digital photography was the way of the future. Because the electronics companies had no chemical photography, they were able to make an unbiased decision.


Consistency bias is also why it is so hard for you to change your stories, or to grow from one stage to the next. You experience huge internal distress, and external pressure, every time you begin changing one of the meaning\hyp{}making stories that run you. This is especially true as you try to move from S3 to S4. Remember, S3 is called the socialised mind, because this is the category of meaning\hyp{}making stories that is the very best at creating large, collaborative, stable and reciprocal collective relationships. No early tribe nor modern community can survive for long without enough people at S3. Keep this in mind when you begin running experiments to create the experiences you need to rewrite your stories; keep an eye open for when consistency bias keeps pulling you back into your old stories.


Consistency bias is also one of the greatest barriers to acting in time to prevent the impending climate crisis. It keeps us using limited companies, even though they cause problems. Consistency bias creates internal distress at the changes in attitude and behaviour needed. Part of our intent in writing this book is to make it easier for all of us to recognise when our meaning\hyp{}making stories are holding us back from making the right choices, and doing what we know needs doing. 


The better you get at chipping away at your consistency and commitment biases, the easier you will be able to decide and act appropriately in the adaptive challenge that our climate crisis is.


And, the better you'll get at enabling other people to take the right action despite their consistency bias.


By keeping a careful eye on which story is compelling you towards consistency, you will develop the capacity to recognise when it is a foolish consistency and when it is a wise consistency. A foolish consistency is likely present when you are about to do something to be consistent, even though your intuition is telling you that you ought not. 
\subsection{Social proof bias}
\label{section:social-proof-bias}
\index{biases!social proof}
Society plays a huge role in shaping us and our decisions\textemdash as Ubuntu\index{Ubuntu} says \begin{quote} I am because we are. \end{quote} The category of meaning\hyp{}making stories\index{meaning-making stories}  we call S3 fully recognises the power of truly belonging to a group of people who help each other in a reciprocal way. A consequence is that we tend to believe whatever the general opinion of others is, is true. Others can be millions who have a certain opinion, or simply one other person.


\begin{longstoryblock}
I (Graham) have lived in a number of different countries with widely different cultural and behavioural expectations. When I moved from Italy to Japan, I was extremely attentive to what I saw the Japanese doing, and to the best of my ability mimicked what I saw people around me doing. That's how we learn as a child what the right thing is to do, and is often the best way to learn how to behave according to the norms and expectations of any culture that you have just landed in. Far better, in my mind, than the many examples of tourists I've seen having no idea of what offence their behaviour has triggered in the meaning\hyp{}making stories of the host culture.
\end{longstoryblock}


You can harness social proof bias to support you in transforming those stories that no longer serve you well. For example, Robert O'Connor \index{O’Connor, Robert} ran an experiment with young socially withdrawn children. (I wish that I had been part of his experimental group. I usually was, and to this day tend still to be, the shy person standing on the edge of the group.) O'Connor\index{O’Connor, Robert} made a range of short films each showing a solitary child watching a group of children doing something together, and then joining in, to everyone's enjoyment. He found that simply showing the film once transformed their behaviour, and the transformed behaviour remained when he returned six weeks later.


Put this into practice when you run experiments for your meaning\hyp{}making stories\index{meaning-making stories} to have the challenging experiences they need to rewrite themselves. The more that your stories see that other stories of how to behave are more successful, the more the old stories will weaken and new ones will grow in their place.


\begin{longstoryblock}
I (Graham) know that social proof bias, along with consistency and commitment bias, are part of why I have continued for the past 12 years to focus on developing the approach described in this book. Having committed to this, quitting my career with Procter \& Gamble,\index{Procter and Gamble}  and having invested heavily, social proof is a big part of what keeps me going. I have a group of people around me who believe in what I'm doing. This keeps me going, despite other evidence and people telling me that I should give up and go back into a normal job. I believe that social proof bias here is helpful, as in many startups and disruptive innovations, because it carries you through that stage where the established majority still dominates.
\end{longstoryblock}


Equally, social proof is what keeps cults going, and holds us back from acting on our climate crisis despite overwhelming evidence. 


Social proof is often far more powerful than any factual proof. If we have enough people around us who are saying the same thing, and acting in a certain way, social proof keeps us acting and behaving in the same way. 


It says that the more people there are, the more that that way of acting or that opinion must be correct. And so in the face of change, consistency bias traps people where they are, and they tend to gather only with people who believe what they believe and act in the way they act so that social proof supports them.


Social proof bias is the reason why, in our modern cities, large groups of people can see a crime happening and not intervene. 


The best-known example of this is the murder of Catherine Genovese in New York, whose killer attacked her three separate times over a 35 minute period. She finally died from the stab wounds, despite screaming for help, with 38 witnesses. 


Because each of them was seeing their inaction as proof that they personally needed to do nothing, nobody did anything. Had one begun doing something, social proof would then compelled another to do something, until all were coming to her aid. 


This is why the first to act, for example, the first follower in a startup is crucial for the success of any founder's endeavour.


If you are ever in a situation where you need help from a large group of people, for example as a project initiator or start-up founder, you are far better approaching specific individuals and asking them to do something specific for you then sending out a general request for support to the universe.




\subsection{Uncertainty avoidance}
\index{biases!uncertainty avoidance}
Uncertainty avoidance\index{uncertainty!avoidance}  is a standard measure across cultures and individual human natures. Most people find uncertainty to be highly discomforting, and to be avoided. Uncertainty avoidance is anchored in both your nature\index{nature}  and your meaning\hyp{}making stories.\index{meaning-making stories}  
People who have a very high explorer energy almost always have a low uncertainty avoidance, and often an affinity for uncertainty. 


\begin{longstoryblock}
I (Graham) experience this. I would never have begun studying theoretical physics, moved into management with Procter \& Gamble,\index{Procter and Gamble}  and then begun developing everything that is described in this book, if I had a high uncertainty avoidance. Nor would I have moved to so many different countries. However, my affinity to uncertainty has come at a cost. Living now in Brussels, none of my school friends are anywhere near me. Every time I've changed countries, my relationship with everyone I had got to know in the previous country shifted to no more than occasional birthday wishes on Facebook.
\end{longstoryblock}


Get to know how strong your uncertainty avoidance tendency is. The higher it is, the more stress you will experience when facing the inherently uncertain, nebulous challenges that face us in today's world. An adaptive challenge brings with it even more uncertainty\index{uncertainty}  than the most uncertain technical challenge because we are required to embrace a journey across a completely uncertain and nebulous void to becoming someone with a new self-identity.\index{self-identity} 


Nothing triggers our uncertainty avoidance bias more strongly than the journey across the void from who we are now, which has given us all the success we have enjoyed up until now, to the person we need to become. After all, we don't even know what that person might be like, how they might act, how they might feel, or even if we can get across the void to become that person. You need to recognise your uncertainty avoidance tendency and figure out how to counteract it through experiments if you are to successfully rise to the adaptive challenges you're facing.
\subsection{Reward vs. loss bias}
\label{section:reward-loss}
\index{biases!reward-loss}
As Warren Buffett\index{Buffett, Warren}  and Charlie Munger \index{Munger, Charlie} say, 


\begin{quote}
Show me somebody's incentives, and I'll tell you how they will behave.
\end{quote}


The difference between our reward bias and our loss bias is one of the reasons why human beings do not take the rational economic decisions assumed in neoclassical economics. In one study a group of students filled in a survey about their preferences for chocolate bars. A while later some of the students were invited to attend a different research programme, and at the end were all rewarded with a Mars bar. This group was carefully selected to have a 50\% preference for Snickers, and a 50\% preference for Mars bars. As they left the room, they were given the opportunity of exchanging their Mars bar for a Snickers bar. Very few of the people who preferred Snickers were able to overcome their loss avoidance bias to give back the Mars bar in order to receive a Snickers in return.


\begin{longstoryblock}
I (Graham) see the far stronger power of loss avoidance bias in many of my decisions. This is one of the biases that feeds into the sunk cost fallacy. If I change my path now, after everything I have invested in it, I will lose everything that I've bought with that investment of time and money. And so loss avoidance is one thing that keeps me going down this path and investing more, even if it might be wise to stop, accept all the losses, and move on to something completely different.


Growing up in Africa, one of the earliest stories that I can remember reading, or perhaps it was even my father reading the story to me, was a story about how to catch a monkey. Take a pumpkin, cut a small hole in its side, hollow it out, and fill it with seeds. And then tie the pumpkin to a tree. If the hole is small enough, when the monkey puts its hand into the hole and grabs a fistful of seeds, its fist will be too big to come back out of the hole again. The pressure to avoid losing these precious life-giving seeds is so strong that the monkey will often stay trapped until caught or even killed.
\end{longstoryblock}


Loss avoidance is often very useful, because if we lose something valuable to us, it may not be replaced. Reciprocity bias has power because it needs to overcome loss avoidance bias to get us to give things away. However, loss avoidance bias is harmful when it traps us into foolish decisions. Like staying with something that we know, even if it's not what we really want, because the pain of losing what we have feels more powerful than the anticipation of what we might get.


\subsection{Liking bias}
\index{biases!liking}
Liking bias says that if you like something or someone, you tend to minimise, or not even see, their faults. Equally, if you dislike something or someone, you tend to see their faults even more clearly and magnify then.


This is especially relevant in a work context, especially for young managers and start-up founders. The tendency to hire or promote people that you like, because you are not seeing sufficiently clearly whether their strengths and weaknesses truly make a good fit to your company and the tasks that they need to do. 


Liking bias is at its most powerful for people who are at S3, where their meaning\hyp{}making stories are part of the socialised mind category. A central theme of S3 stories is the necessity to be like everybody else in the group, and to be liked by everybody in the group, in order to be accepted.


Managers and leaders need to be constantly on their guard against liking bias. Whenever you hire somebody into your organisation, or promote that person, take enough time to gather information on whether that person is truly a good fit for the tasks, and for the organisation's culture. Keep your eyes open for when liking bias might be hiding relevant information from you.


Liking bias is also a very powerful driver for your success in your career, getting helped, or staying out of jail. For example, in one study of jail sentences, it was found that defendants who were rated as attractive were twice as likely to walk free from the courtroom as those rated as unattractive. In another study of hiring decisions, it was found that how well-groomed the applicant was was actually a stronger driver for the hiring decision than the job qualifications, even though the interviewers believed that the job qualification was the dominant frame of reference\index{frames of reference}  they were using to take a decision.


People can manipulate you by using your liking bias simply by indicating that they like you. As soon as somebody makes it clear that they like you, reciprocation bias may kick in, along with social proof, so that you start to like them. And then you're a sucker for whatever they want to do with you.


If you have just met someone, especially in a sales situation, keep an eye on how strongly you like them. If you have any indication that you've developed a stronger liking for this person then you typically do for somebody like that, then be very wary about whether they are deliberately manipulating you into liking them, knowing that your liking bias will then prevent you from seeing all the negatives. For example, why you should not buy what they are trying to sell you, because it's not very good.
\subsection{Authority bias}
\index{biases!authority}
Authority bias comes in a light and a strong version. The light version kicks in when somebody tells you to do something and gives you a reason for doing this. Authority bias, coupled with social proof bias, makes it far more likely that you would do what they tell you, or give them what they are asking for, regardless of whether the reason has any validity whatsoever.


Simply by using a sentence that has some kind of reason and logic to it, your brain makes a quick decision on whether it might be worth the effort of Type~2 thinking to really look at the reason, think through whether it has any relevance to the request, and whether it holds water or not. All too often your brain decides to just go with the bias so fast that you don't even have the opportunity to decide consciously.


Be cautious when somebody gives you a reason why you should do what they ask. Much of the time the reason may well be valid, but sometimes it's not. 


The strong form of authority bias is how you react to any authority figure. I know for myself, somebody who has the credentials or dress or behaviour of authority triggers my authority bias in one of two ways. Sometimes, I suspend judgement, and tend to do what they say because they are an authority figure. At other times, I tend towards doing the opposite simply because they are an authority figure. Neither is sure to be wise.


The best defence you have against undue authority bias is to stop yourself whenever you are interacting with somebody who either is, or is pretending to be, an authority. Ask yourself, if exactly the same information was being given to me by somebody I had never met before, who looked like me, and who dressed like me, would I do what they're saying? Even more strongly, if somebody I had formal authority over was giving me this message, would I do what they're saying?


The second thing to do is to ask whether this person truly is an authority, and are they sufficiently expert. Ask yourself how this authority is rewarded. And then ask yourself whether you can really trust them to be acting in your best interests. 


And apply this to anything you do today because your younger self said to.
\subsection{Superiority-Inferiority bias}
\label{section:dunning-kruger}
\index{biases!superiority-inferiority}
The Dunning-Kruger effect\cite{dunning-kruger-original,dunning-2,dunning-beyond,dunning-book} plays a huge role in organisations. Combine the Dunning-Kruger effect\index{Dunning-Kruger effect}  with some of the other biases above, such as authority bias and uncertainty avoidance bias, and you have a perfect storm. Throughout this book, as you read about how various myths\index{myth}  about incorporation and the economy\index{economy}  have become concretised in the reality you experience, imagine how often the Dunning-Kruger effect\index{Dunning-Kruger effect}  has created the crises of our reality today.


The Dunning-Kruger effect is behind many of the myths of superiority and inferiority that pervade your life. Put simply, it is that almost everybody thinks they are average. 


This is accurate if your skills, competencies or capacities are indeed average. It is an unhelpful distortion if they are significantly below or significantly above average. And it is disastrous if somebody whose skills, competencies, and capacities lie way above average is reporting to somebody who is way below average. It is toxic for the junior person, and toxic for the organisation as a whole, because choices will be made that are wholly inadequate for the challenges being faced.


You can only accurately evaluate how competent you are if you are sufficiently competent in the skill in the first place. If you are not competent enough, you don't know what you don't know, and you don't even have a frame of reference good enough to evaluate your competence. So you will typically overestimate your confidence. 


Reinforced by the other biases, and even more so if you have personality traits that tend towards confidence, optimism, or narcissism.


Vice versa, people who are really good often evaluate themselves as inferior, about as good as everybody else, because they are so good they have a highly precise and evolved frame of reference, which they use to evaluate accurately the remaining gap between themself and the very best in the field.


Way too often in expert fields, someone who is not competent enough to recognise their own incompetence also has very high self-confidence and the ability to project that self-confidence. They will invariably convince those around them that they are the best and end up in positions of authority. They are even able to convince people better than them! 


Anyone who is too incompetent to be able to accurately evaluate their competence, will make error after error, blame those errors on people around them and on circumstances, without ever realising that the problem lies with their own competence. Combine this with authority bias, and especially an expert stage of self-identity, and dumb decisions will be made.


The toughest thing to do is recognise clearly what is happening, if you are working in such an inverted competency hierarchy and judging yourself as inferior. The best way to target this is to simply be aware of the Dunning-Kruger effect, raising the standard of competence of everybody\cite{dunning-beyond}, and to replace any kind of management accountability hierarchy with a human capacity hierarchy (defined in Section~\ref{section:integral-organisation}), where positions are filled using a Sociocratic consent process (Chapter~\ref{chapter:who-is-your-organisation-tasks}).




\subsection{Scarcity bias}
\index{biases!scarcity}
So much of what I (Graham) have done has been driven by scarcity bias, sometimes where the actual scarcity of something led me to see it as far more desirable than it actually was for me, or where something actually was scarce, and because of that my focus narrowed down onto only that part of my reality, ignoring everything else that was plentiful. Either way, scarcity bias has been behind many of the mistakes I've made in my life.


\begin{longstoryblock}
Scarcity bias was part of an accident I had 15 years ago on my motorbike on the N\"{u}rburgring in Germany. I'd ridden over from Brussels to spend a couple of hours on the ring during the open sessions at the end of every day. There was about enough time to get ten laps in, and so as usual when the eighth lap arrived I pulled off the ring. My rule always was to never ride the final lap or two, because that's when tiredness and overconfidence come together to dramatically increase the risk of an accident. Not just in me, but in everybody else on the ring.


However, on that day they were four Ducatis ahead of me and I could see that they were about to head off for a lap. I thought, in a week's time is the Motorrad training weekend on the ring. These are the last two laps of practice that I can get in, so I'd better get these two laps of practice in so that I'm just that little bit better prepared for the training session. The lens of scarcity bias made these final two laps seem so incredibly desirable that I saw nothing else. I was not seeing clearly the abundance of eight laps that I already had ridden that day, the many laps that I would ride in a week’s time, and a lifetime of laps ahead of me.


Halfway around the ring, on the Exm{\"u}hle 90° right-hander, I banked over just a little bit further than the bike could hold, and the wheels slid out in a classic lowsider. 


If I had to have an accident, that could not have been a better accident to have. I was also unbelievably lucky in where I had the accident; the ambulance crew, parked 15m away, was with me within 20 seconds. They were already starting the engine and entering the track before I'd finished hitting the barriers; and, because it's a steeply banked curve, by the time my body hit the barriers I'd already scrubbed off much of my speed; and because I hit the barriers with my body parallel, the impact was spread across my entire body.


I walked away from that accident with nothing broken, no lasting damage except to my ego, and was back on a borrowed motorbike the following weekend to take part in the training session.


That accident was also really good for me in two other ways. First, it really brought home to me the abundance of innovation potential that I had ahead of me in my life, and the role I could choose to play in the crisis that humanity was entering. 


It brought home to me that, whilst I had the potential for a good career in Procter \& Gamble, I would never be truly proud of myself at the end of that career. I realised I was staying at P\&G because of scarcity bias.\index{Procter and Gamble}  Scarcity bias led me to think that there was very little that I could do that would earn money, so I'd better stay with what I had even though it was not what I really wanted to do, nor was it anywhere close to the biggest difference I could make in the world. It had very little to do with what I had chosen as my core purpose in life, which was to make a difference towards the climate crisis and what we now call the sustainable development goals (SDGs).


That accident may well have been the best thing that happened to me, and to everyone who has found value in what I've done over the past 11 years, and perhaps for you reading this book now, because it broke the spell that scarcity bias had had on me.
\end{longstoryblock}


Scarcity bias, like all the other biases, has been and still is a vital tool enabling us to thrive. 


When it's right, we can make quick decisions, spending very little time, by using Type~1 thinking. Very often, scarce items are more valuable than plentiful items. 


Facebook, Google, \index{Facebook} and all the others know that the scarcest resource in the world today is your attention. \index{Google} They will use every trick in the book to convince you that you absolutely need to look at the next post because it is scarce. FOMO, Fear Of Missing Out, is an example of scarcity bias.


Scarcity bias also kicks in with non-material values, such as freedom.\index{freedom}  If you have a freedom that is valuable to you, and that freedom is even slightly curtailed, you are very likely to overreact because of scarcity bias. You may well recognise this in yourself as you were growing up, or if you have a teenage child. 


And so we react more intensively than warranted. Whenever you find yourself reacting way more intensely than is reasonable, you have been hijacked by one of the biases.


If you find yourself wanting something very intensely, check whether scarcity bias has kicked in. Auctioneers are very good at setting the stage for scarcity bias to drive your behaviour. A few years ago many countries around the world held auctions for access to the 4G spectrum. In some countries these auctions were carefully designed to make full use of these biases. The result was that the large cell phone providers paid far more than they ever had before for access to the new bands.


\subsection{Dealing subtly and compassionately with biases}
The challenge dealing with all these biases is that they compel you to act through your emotions. You are seldom aware in time of the Type~1 thinking, or of your hidden meaning\hyp{}making story. And because strong emotions tend to shut down cognitive, Type~2 thinking, just knowing about these biases from reading this book will help you in 1\% of cases.


Your biggest defence against you manipulating yourself, or somebody else manipulating you through triggering the stories behind your biases, is to get really good at slamming on the brakes whenever you feel strong emotions driving you towards a specific decision or action.


At that point, the Adaptive Way\index{Adaptive Way}  pattern called Psycho~1 is a powerful defence. In Psycho~1, you simply ask yourself, 


\begin{quote}
if I did not feel the way I do, what other decision would I take? What else would I do?
\end{quote} 


In the case of scarcity bias being the story that is triggering disproportionately intense feelings, ask yourself what you want this for. Is it something that will truly benefit you if you possess it? Where the benefit is bigger than whatever it will cost you to get? If not, think seriously about walking away, especially if there is any intuition telling you that you are in the hands of a master at using your biases to get you to comply with their needs.


Recent research into food cravings, and other kinds of addictions, show that physical needs for food, or physical addiction, is less important than we thought. When you feel a strong craving for chocolate it's unlikely that your body is telling you that it needs something in the chocolate. It's far more likely that you had two fleeting thoughts in quick succession: something triggered a fleeting thought of chocolate, immediately followed by your conscience saying no, and before you became aware of either of those thoughts, a bias kicked in and triggered an intense emotional craving for chocolate. Your bias did register both fleeting thoughts and concluded, for example, that if chocolate was something that was forbidden or restricted, it was scarce and valuable, so you had to be forced to get some. Now!


This is similar to willpower.\index{willpower} Old, less valid research suggested that willpower was some kind of finite resource. That on average you had enough willpower in your tank to take only five tough choices each day. So, if you were sitting in a business meeting, where you needed to take five tough decisions, not to attempt to use any willpower to resist eating a plate of biscuits in front of you. We now know that this research is wrong. Willpower is not like a fuel tank that gets used up, it's an emotion influenced by scarcity bias.


As soon as you feel the stress of resisting the plate of biscuits, or the stress of being in conflict around a tough decision, your brain narrows in on, say, calmness as a scarce resource. This triggers the overreaction in your feelings that makes it so difficult to resist the temptation to eat the biscuits. Once you recognise that willpower is a feeling triggered by your meaning\hyp{}making stories; is neither a fuel tank that gets drained nor a muscle that gets exhausted; you have the power to recognise that: either somebody else is deliberately triggering your biases to get the decision they want; or something deep in you is triggering your biases. 


Step away from those feelings by using psycho~1, use the space you get to remember that willpower is a feeling, and broaden your awareness from the narrow focus on what you're feeling now and the single decision in front of you, to the coming week, year, or even the rest of your life. Actively look at everything that is abundant in your present and future. (I'm glad I can occasionally do this; I'm working hard, though, to become consistent.)