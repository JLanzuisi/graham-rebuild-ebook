\newcommand{\book}{graham}

%----------------------------------------------------------------------------------------
%	REQUIRED PACKAGES AND MISC CONFIGURATIONS
%----------------------------------------------------------------------------------------

% TODO remove the [demo], it's just because images are missing
\usepackage{graphicx} % Required for including images
\graphicspath{Images} % Specifies where to look for included images (trailing slash required)

% Table packages
\usepackage{booktabs} % Required for better horizontal rules in tables
\usepackage{longtable} % Required for tables that span multiple pages automatically

% Graham's packages
\usepackage{layout}

\usepackage[utf8]{inputenc} % Required for inputting international characters
\usepackage[T1]{fontenc} % Output font encoding for international characters
\usepackage{csquotes}
%\usepackage{CJKutf8}  %For Kanji
\usepackage{textcomp}
\usepackage{eurosym}
\let\texteuro\euro  %makes the euro character typed in give the same euro symbol as \euro
\usepackage[UKenglish]{babel}
\usepackage{hyphenat} %attempt to better hyphenate compund words using \hyp{}

\usepackage{etoolbox} % Required for conditional logic

\usepackage{xpatch} % Allows easy modification of macros

\usepackage{emptypage} % This package removes headers and footers on empty pages between chapters


\usepackage[hang, flushmargin]{footmisc} % Required for modifying the footers to be flush against the left margin
\renewcommand{\footnoterule}{\noindent\rule{1.5cm}{1pt}}
\usepackage{fmtcount} % Allows converting numbers to words, e.g. 2 -> two

\usepackage{changepage} % Required for temporarily indenting text blocks

\usepackage{pdfpages} % Required for adding full page images

\usepackage{makeidx}  %required to make an index automatically
\usepackage{comment}

%----------------------------------------------------------------------------------------
%	MARGINS
%----------------------------------------------------------------------------------------


\usepackage[
paperheight=9in, paperwidth=6in,
twoside,
outermargin=1.0cm,
hscale=0.90, 
vscale=0.80,  
bindingoffset=2.0cm,
headheight=16pt,
vcentering,  
showcrop 
]{geometry}

%----------------------------------------------------------------------------------------
%	BIBLIOGRAPHY
%----------------------------------------------------------------------------------------

\usepackage[backend=biber, 
bibstyle=numeric, 
citestyle=numeric-comp, %to give 1-5 ranges by Jhonny
sorting=none]{biblatex} % Use the biber backend which requires compiling with the biber engine

% Graham version to use numeric-comp 
% from https://tex.stackexchange.com/questions/114987/biblatex-supercite-with-square-brackets-and-grouped  
\DeclareCiteCommand{\supercite}[\mkbibsuperscript]
{\usebibmacro{cite:init}%
	\let\multicitedelim=\supercitedelim
	\iffieldundef{prenote}
	{}
	{\BibliographyWarning{Ignoring prenote argument}}%
	\iffieldundef{postnote}
	{}
	{\BibliographyWarning{Ignoring postnote argument}}%
	\bibopenbracket}%
{\usebibmacro{citeindex}%
	\usebibmacro{cite:comp}}
{}
{\usebibmacro{cite:dump}\bibclosebracket}

% Force standard \cite{} citations to use the supercite style
\let\cite=\supercite %removed by Jhonny

%------------------------------------------------

% Change the order of publisher and location with a comma between them
\renewbibmacro*{publisher+location+date}{%
	\printlist{publisher}%
	\setunit*{\addcomma\space}
	\printlist{location}%
	\newunit%
}

%------------------------------------------------

% Change the formatting of the volume and number
\renewbibmacro*{volume+number+eid}{%
	\textbf{\printfield{volume}}
	\setunit*{\addnbspace}
	\printfield{number}%
	\setunit{\addcomma\space}%
	\printfield{eid}%
}

%------------------------------------------------

% Remove brackets around the year
\xpatchbibmacro{date+extradate}{%
	\printtext[parens]%
}{%
	\setunit{\addperiod\space}% Replace parentheses with a full stop and space
	\printtext%
}{}{}

%------------------------------------------------

% Brackets around the journal number
\DeclareFieldFormat[article]{number}{\mkbibparens{#1}}

%----------------------------------------------------------------------------------------
%	SECTIONS
%----------------------------------------------------------------------------------------

% Colon after volume/issue before pages
\renewcommand*{\bibpagespunct}{\addcolon\space}


\usepackage[newparttoc]{titlesec} % Required for modifying sections

\newcommand{\redefinepart}[1]{ % Command to change the \part styling, takes an argument for an optional part description
	\titleformat
	{\part} % Section type being modified
	[display] % Shape type, can be: hang, block, display, runin, leftmargin, rightmargin, drop, wrap, frame
	{\thispagestyle{empty}\sffamily\centering} % Format of the whole section
	{% Format of the section label
		\expandafter\ifstrequal\expandafter{\book}{robert}{\LARGE Part \thepart\\}{} % Part title
		\expandafter\ifstrequal\expandafter{\book}{graham}{
			\LARGE Part \Numberstring{part}\\ % Part title
			\vspace{-6pt}%
			\rule{0.20\textwidth}{3pt}\vspace{6pt}%
		}{} % Grey rule
	}
	{0pt} % Space between the title and label
	{\Huge} % Code before the label
	[\vspace{3\baselineskip}\parbox{\linewidth}{\rmfamily \Large #1}] % Code after the label
}

\redefinepart{} % Redefine \part{} to not output a part description (in case the user doesn't know about the custom \part commands below)

\newcommand{\partplain}[1]{ % Output a plain part with no description
	\cleardoublepage
	\redefinepart{} % Redefine \part{} to not output a part description
	\part{#1}
	\cleardoublepage
}

\newcommand{\partdescription}[2]{ % Output a part with a description
	\cleardoublepage
	\redefinepart{#2} % Redefine \part{} to output a part description
	\part{#1}
	\redefinepart{} % Redefine \part{} to not output a part description (reset to default)
	\cleardoublepage
}

%----------------------------------------------------------------------------------------
%	TABLE OF CONTENTS
%----------------------------------------------------------------------------------------

\usepackage{titletoc} % Required for manipulating the table of contents

\renewcommand{\contentsname}{Table of Contents} % Rename the table of contents chapter heading
\setcounter{tocdepth}{0} % Parts and chapters in the ToC; Use 1 or 2 for full ToC

\titlecontents{part} % Section type being modified
[0pt] % Left indentation
{\vspace{32pt}\centering\sffamily\Large} % Before code % was 6 originally.
{\bfseries PART \thecontentslabel:~\uppercase} % Formatting of numbered sections of this type
{\bfseries\uppercase} % Formatting of numberless sections of this type
{} % Formatting of the filler to the right of the heading
[\vspace{6pt}] % After code

%------------------------------------------------

\titlecontents{chapter} % Section type being modified
[0pt] % Left indentation
{\vspace{10pt}\centering\sffamily} % Before code % was 6pt
{\bfseries\thecontentslabel.~\uppercase} % Formatting of numbered sections of this type
{\bfseries\uppercase} % Formatting of numberless sections of this type
{~~\textemdash\textemdash~~\bfseries\thecontentspage} % Formatting of the filler to the right of the heading
[\vspace{6pt}] % After code

%------------------------------------------------


\titlecontents{chapterdescription} % Section type being modified
[0pt] % Left indentation
{\centering} % Before code
{} % Formatting of numbered sections of this type
{} % Formatting of numberless sections of this type
{} % Formatting of the filler to the right of the heading
[\vspace{6pt}] % After code

% If a chapter description exists, set it to a level where it doesn't appear in the PDF bookmarks
\makeatletter
\providecommand*{\toclevel@chapterdescription}{3}
\makeatother

%----------------------------------------------------------------------------------------
%	LINKS
%----------------------------------------------------------------------------------------

\usepackage{hyperref} % Required for links

\hypersetup{
	colorlinks=false,
%	urlcolor=Black, % Colour for \url and \href links
%	linkcolor=Black, % Colour for \nameref links
	hidelinks, % Hide the default boxes around links
}

%----------------------------------------------------------------------------------------
%	BOOK INFORMATION PAGE
%----------------------------------------------------------------------------------------

\newcommand{\bookinformationpage}[1]{
	\newpage
	\section*{Book information}
	\thispagestyle{empty} % Suppress headers and footers
	\begin{adjustwidth}{0.1\textwidth}{0.1\textwidth} % Left indent then right indent	
		\begin{center}
			#1
		\end{center}
	\end{adjustwidth}
}

%----------------------------------------------------------------------------------------
%	DEDICATION PAGE
%----------------------------------------------------------------------------------------

\newcommand{\dedicationpage}[1]{
	\cleardoublepage % Ensure the dedication is on an odd page
	\section*{Dedication}
	\thispagestyle{empty} % Suppress headers and footers
	
	\begin{adjustwidth}{0.2\textwidth}{0.1\textwidth} % Left indent then right indent
		\raggedleft
		\setlength{\baselineskip}{2.0\baselineskip} % Space between pagraphs
		\sffamily % Sans
		\textbf{#1} % The dedication text
	\end{adjustwidth}
}

%----------------------------------------------------------------------------------------
%	LOAD PACKAGES THAT CLASH WITH OTHERS ABOVE
%----------------------------------------------------------------------------------------
\setcounter{biburllcpenalty}{7000}
\usepackage{array, ragged2e}  %Added by Jhonny for better arrays
\newcolumntype{L}[1]{>{\RaggedRight\hspace{0pt}}p{#1}}
\newcolumntype{R}[1]{>{\RaggedLeft\hspace{0pt}}p{#1}}
\newcolumntype{C}[1]{>{\centering\hspace{0pt}}p{#1}}

\usepackage[nonumberlist,toc={false}]{glossaries-extra}  %automated glossary package
\makeglossaries
\loadglsentries{./glossary.tex}
\glsaddall

\newenvironment{SCfigure}{\begin{figure}}{\end{figure}}
\newenvironment{longstoryblocktitle}{}{}
\newenvironment{longstoryblock}{\begin{quote}}{\end{quote}}
\newenvironment{chapterquotation}{\begin{center}}{\end{center}}
\renewenvironment{quotation}{\begin{quote}}{\end{quote}}