\chapter{Acknowledgements}


\begin{chapterquotation}
Painting is stronger than me, it makes me do its bidding. \par  Everybody has the same energy potential. The average person wastes his in a dozen little ways. I bring mine to bear on one thing only: my paintings, and everything else is sacrificed to it \ldots myself included. \\
\raggedleft\textemdash Pablo Picasso
\end{chapterquotation}


\section*{Graham}
Everything that you have read in this book has come out of my understanding of what works, based on my experiences over five decades, including what I have learnt from many more able than I am. These experiences, and what I have learnt, have been shaped and informed by countless other people, people I have known in person, people that I have known through others, and people I have never met who have written books or given superb TED talks.


Staying focused on my energy potential, as Picasso writes in the quote above, has required sacrifices from me; and more importantly others have also sacrificed. Everyone who has paid a price for this book and work to exist, thank you. 


Much of this has been shaped by my growing up in South Africa, and the privileges I had growing up there, with my amazingly supporting parents. My father, Hylton, who passed away during the middle of writing this book, and my mother, Barbara, who is a lively 90 year old as I write this, and my sister Tessa, have all contributed an unknowably large contribution to me being me, and this book bringing you whatever value you have taken from it. My work today is in part driven by a desire to pay forward to the next generations all the good things that previous generations have in their turn paid forward to me. 


I was also privileged to go to excellent schools, Selborne Primary and College, and then to the  excellent Universities of Cape Town and Bielefeld. I cannot know how my life would have turned out without everything I learnt there, both about subjects and about life, from my peers and teachers. Hard to do so, but I will single out Profs David Aschman who helped me decide to go to UCT, Jean Cleymans, and Frithjof Karsch who supervised my theses. 


There are a few though who have been important contributors to this book itself. Marko Wolf has contributed many of the illustrations and refined my understanding of what we're doing and who I am in many challenging sparring sessions over the years. My co-founders of Evolutesix, Jason Maude, Robin Toller, Marko Wolf, Adrian Meyer, and Nikyta Guleria, along with Bernhard Possert, Rob Bigge, and Andrius Juknys who joined me later, have also, in numerous situations, sharpened my understanding of how to make this all work. 


Nikyta especially, along with Nour El-Din Hussein, her husband, have gone way beyond the extra mile. Niky has proven the best possible colleague, doing a superb job of encouraging me when I needed encouragement, challenging me when I needed to be challenged, and has been as committed as I am to turning to action everything in this DIY guide, over a journey that took twice as long as we’d planned.


Otto Laske's\index{Laske, Otto} seminal work on the Cognitive Developmental Framework has been the theoretical foundation for much of my own development over the past decade, and hence for Evolutesix and this book. He has built on the work of Kegan, Lahey and their group; Jaques; Adorno, Bhaskar; and many more. 


Norman Wolfe's development of his Living Organisation model was the final crux stone in the long arch of this book, holding it all together. Equally pivotal has been the FairShares approach of Prof Rory Ridley-Duff and Cliff Southcombe of Social Enterprise International, which I integrated with my Free / Commons Company approach.


Paraphrasing Newton, if you are benefitting from this book, it is because I have stood on the shoulders of giants. 


Eva Gottschlich and Jean Lin have been very generous in their support in making this book possible, for which I will always be grateful.


The people I worked with in Edgetalents, one of the companies where I demonstrated five years ago the power of integrating all three pillars, and the issues if any one was insufficiently strong to support the others, have all contributed to this book: Manuela, Thamires, Nikyta, Gustavo, Vanessa, Illuska, and Kelly.


Hannah and Sophie in London; and Manon, Antoine, Jean-Chris\-to\-phe in Brussels;  gave me insights into the worldview, hopes and fears of people just entering, at, and leaving University, and guided how this work has evolved.


There are many more that have contributed significantly yet have not mentioned by name; you'll know if you have contributed, and I thank you! However long I make this section, I will still be missing some important people.


\section*{Jack}
Writing the acknowledgements should be easy, right? The book is finished and like reaching the peak of a tall mountain, we are exhilarated by the arduous climb and the view from the summit, but this same exhilaration fogs our memory of all who have helped with this book, both directly and indirectly. So in advance I ask forgiveness from any person I have omitted; my forgetfulness abetted by the need for brevity.  


I must start with my grandfather, Lester James Reardon, who first taught me the critical importance of our environment and sustainability at a time when few people gave it much thought, and even fewer listened. He also taught me the value and joy of reading, effectively handing the keys to a vast intellectual community, past, present, and future, of which I am so glad to be a member. (And to this day, one of my life’s many pleasures, is reading a good book on a rainy night.)


I thank my parents, who welcomed me during dark times, offering me repose and comfort, and for their continued love and support. I thank my two brothers, Dave and Bobby, who have embraced their rebellious and non-binary sibling with love. I couldn’t have written this book without the comforting love of my family.


Finishing this book in Tzaneen, South Africa, surrounded by ruggedly beautiful mountains hardly affected by the passage of time, encourages recognizing my debt to countless authors past and present—too numerous to mention, who enabled me to create my own stories, so that I can give something back to help others, to help them surmount their own barriers, reach their summit, and lead better lives.  


As a teacher myself, I strive to positively affect the lives of my students both intellectually and personally; and in my own life, I have been quite fortunate to have had several gifted teachers, all responsible in some small way to enable me to have written this book.  Brother Joe Girard, my high school American Literature teacher, infused me with a life-long love affair with words, not only as interesting entities on their own, but also as interconnected tools in developing adequate meaning-making stories (even today I still carry a vocabulary notebook). My high school football (American) coach, Frank Defelice, who despite my lack of football talent (or maybe because of it) stabilized my tumultuous teenage years with wisdom and good advice. From them I realized that, before I even began studying physics, that my spacetime was curved, that others can positively change my life’s direction in completely unanticipated ways. 


At the College of the Holy Cross, Professor J.J. Holmes, in his course The History of Revolutions, made history come alive, especially at critical junctures when one epoch leads into another. We are in such an epochal moment now, and I thank Professor Holmes for the foundational tools to write this book. 


Frank Petralla, who was my first economics instructor (I couldn’t get enough of his courses), not only inspired me to become a professor but also to enroll at the University of Notre Dame (his alma mater) to earn my doctorate in economics. Teresa Ghillarducci, my dissertation advisor, patiently molded me into a competent economist. Bill Leahy, advisor, friend, professor, and co-author of my first two academic articles epitomized the consummate Notre Dame professor.  And finally, thanks to Charles Wilbur who believed in my fiction writing when no else did, and that a good economist could also be a novelist. I am happy to say that my first novel is finally finished! 


I’m thankful for attending the University of Notre Dame for my graduate studies in economics, for it laid the foundation for me to become a full-fledged pluralist, also imbibing me with a sense of justice. As just one example: After making several trips to Appalachia, organized by Teresa Ghillarducci and another favorite ND professor, Larry Marsh, I was struck by the immense contrast between the wealth underground and the lack of wealth above ground. I conducted my doctoral dissertation on working conditions in the coal mining industry; and these Appalachia trips stimulated my life-long interest in sustainable energy.  


After I received my doctorate in economics, I embarked on a typical career for economists doing everyday work and practicing everyday economics (similar to what Thomas Kuhn wrote about in \emph{The Structure of Scientific Revolutions}.  But one disparaging sentence, \emph{“How Dare You”}, changed my life. I quickly realized that educating our young people (rather than proselytizing them) so we all can create better meaning-making stories was far more rewarding than engaging in the more mundane everyday economics. Central in my endeavor was the launching of a new academic journal,  \emph{The International  Journal of Pluralism and Economics Education}, which after eleven years is still going strong. The IJPEE has played a small but significant role in  pluralizing economics, and I thank all the people at Inderscience and especially Liz Harris.


I thank my fellow editor, good friend, and kindred spirit Miriam Kennet, founding editor of the \emph{International Journal of Green Economics}; and founder of the Institute  of Green Economics (Reading, UK) whose irrepressible energy to make the world the better a place consistently fuels my own energy.  And I thank my life-long Indian friends, Prithvi Yadav, Awadh Dubey, and Sudipta Bhattacharyya, whose  passion to educate, to help the less unfortunate, and to make the world just and sustainable, is forever inspirational. 


And most important, I thank my two children, Elizabeth and Patrick, whose existence is forever interconnected with mine, whose energy and intellectual curiosity vitalizes me, empowering me to help make their world (and everyone’s) sustainable, democratic, and able to provision for all. I dedicate this book to you: may you and your children look back at this book and thank your father for co-writing it.  




\section*{Overall book}
Both of us are enormously grateful to a number of people who have contributed to making this book possible. Andy, Vincent, and the team at Quietroom have been unflagging in their encouragement to write the book in a way that it is easy to read. Anna Kierstan has achieved well the challenge of polishing all our rough edges away. 


We would both like to thank everyone at Granny Dot’s, near Tzaneen in South Africa, who took such excellent care of both of us during the final stages writing this book. The people and the place proved a vital source of inspiration.


This book was also made possible by support in cash and in kind by the following people, both through our crowdfunding campaign, and directly.


\clearpage


\subsection{Chapter Sponsors}
\begin{itemize}
\item Consorticon Group sponsored Chapter~\ref{chapter:who-is-your-organisation-human} (who hope that everyone may have the privilege of working with colleagues like the team in Consorticon). They  work in many of the ways described in this book, and especially resonate with the role of source in Section~\ref{section:source}.  
\item VME Retail and Coopexchange (and exchange for investor shares of FairShares and similar companies), long supporters of the FairShares approach, has sponsored Chapter~\ref{chapter:create-your-FSC}.
\item Vet Dynamics UK, a leading consultancy in creating regenerative veterinary practices, sponsored Chapter~\ref{chapter:who-am-i-meaning}.
\end{itemize}
\subsection{Gold}
\vspace{-3ex}
\begin{table}[h]
        \centering
\begin{tabular}{ c c c}
\toprule
Philippe Schmidig & Jess Allen & Vincent Franklin \\
\bottomrule
\end{tabular} 
\caption[Gold backers]{Gold backers, the early backers contributing over \pounds 500}
% \label{table:}
\end{table}


\subsection{Bronze}
\vspace{-3ex}
\begin{table}[h]
        \centering
\begin{tabular}{ c c c}
\toprule
Janssen Groesbeek & Daniel Godfrey & Henry Leveson-Gower \\
Eleonora Weistroffer & Jerry Koch-Gonzalez & Elizabeth Mong \\
Stephanie Bouju & Antonio Potenza & Whitney Rubison \\
Keith Bollington & Anna Plodowski & Robert Dellner \\
Alex Champandard & Cecil Schmitt & Thomas Tomison \\
Vincent De Waele & Achim Hensen & Rory Ridley-Duff\\
Laureen Golden & Louis Weinstock  & Olivier Brenninkmeijer  \\
\bottomrule
\end{tabular} 
\caption[Bronze backers]{Bronze backers, the early backers contributing over \pounds 100}
% \label{table:}
\end{table}
\clearpage